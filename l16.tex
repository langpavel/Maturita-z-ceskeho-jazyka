\chapter[Česká próza od počátků do 20.~století]{Česká próza od počátků do 20.~století a~její mnohovrstevnatost v~době meziválečné}

Na českou tvorbu mají vliv moderní umělecké směry, ovlivňují ji též
společenské události, jako 1.~světová válka, ke které se většina autorů
stavěla negativně. Oběvují se ohlasy \emph{Říjnové revoluce} v~Rusku,
vznik Česko--Slovenské republiky, projevuje se poválečná krize, nástup
fašismu v~Německu (1933), mnichovská zrada (1938), 2.~světová válka
a~okupace.

Vedle sebe působí autoři různých generací. Pokračuje tvorba
\emph{Jaráska} a~\emph{Wintera}, anarchystů a~buřičů (\emph{Šrámek,
Dyk}), vychází tvorba \emph{Olbrachta} a~\emph{Mayerové} a~začíná psát
řada mladších autorů.

V~próze se vydělují 20.~a~30.~léta. V~20.~letech vzniká próza v~nově
vzniklé republice. převažují kratší prozaické útvary -- povídky, novely,
romány  se oběvují jen vyjímečně. Autory je starší generace, tvorba se
vztahuje k~1.~světové válce. Silný vliv má poetismus a~ostatní moderní
směy. Experimentuje se s~formou děl -- silný vliv publicistiky.

Ve~30.~letech se oběvuje spor mezi příznivci sociálního realismu
a~představiteli surrealismu. Spor ovlivňoval celý tehdejší kulturní
život, vzniká \uv{syntetické umění} -- dílo má být dokonalé po stránce
obshové i~formální. Vrcholí česká próza\dots{} Na konci 30.~let začíná
převažovat baladické ladění. Zapříčinil to nástup fašismu, Mnichov
a~hospodářská krize. 

\section*{1.~světová válka v~české literatuře}
1.~světová válka ovlivnila celou řadu autorů. Stala se ideálovým
a~mravním problémem. Češi byli od počátku proti, ale bojovali za R-U proti
Rusům. Toto téma se projevuje u~autorů, kteří neměli s~válkou skušenosti.
\emph{Ivan Olbracht} napsal \dilo{Podivné přátelství herce Jesenia,}
\emph{Anna Marie Tilschová} \dilo{Haldy} -- psala romány o~umělcích,
\emph{K.~M.~Čapek -- Chod} \dilo{Jindrové,} 
\emph{M.~Mayerová} \dilo{Nejkrásnější svět.}
Autoři, kteří vycházeli z~vlastní skušenosti byli ovlivněni absurditou
války a~bezmocností. \emph{Fráňa Šrámek} \dilo{Žasnoucí voják,} \dilo{Tělo}
Vedle toho se v~dílech oběvuje i~humor, který činí tuto prózu významnou.

\subsection*{Jaromír John}
\jmeno{Jaromír John} se k~válečné problematice vyjadřuje v~sbírce
povídek \dilo{Večery na slamníku} Každá povídka má jiný charakter,
odlišuje se i~úroveň spisovnosti jazyka, hrdinové jsou různorodí,
oběvuje se vypravěč. Všechny povídky jsou spojeny prožitkem 1.~světové
války. Hrdinové touží po domově i~když se nemají na co těšit.

K~1.~světové válce se vrací i~později. Napsal protiválečnou
a~protifašistickou novelu \dilo{Zbloudilý syn,} kde je hrdinou sadista,
který se stává po válce počestným živnostníkem a~ve 30.~letech se stává
nacistou.

Po děti napsal \dilo{Vojáčka Hubáčka} a~prózu \dilo{Rajský ostrov.}

\subsection*{Jaroslav Hašek}
\jmeno{Jaroslav Hašek} je nejvýznamějším představitelem české i~světové
humanistické prózy. Psal humoresky a~satirické povídky -- 
\dilo{Trampoty pana Tenkráta,} \dilo{Můj obchod se psy a~jiné humoresky,}
satira na politické poměry o~vzniku strany \dilo{Dějiny Strany mírného
pokroku v~mezích zákona,} kam vložil i~politické programy.

Poté se oběvuje jeho \emph{Švejk}. Poprvé se oběvil ve sbírce
\dilo{Dobrý voják Švejk a~jiné historky.} \emph{Švejk} je úředně uznaný
blb, který je odhodlaný plnit příkazy císaře pána až do roztrhání těla.
\dilo{Osudy dobrého vojáka Švejka za světové války} nejsou dokončené.
Forma připomíná pikareskní román -- pásmo volně spojených scén, které
nejsou spojeny motivem cesty a~hlavním hrdinou je šibal Švejk.

\paragraph{Švejk}
Důležitější než děj jsou Švejkovi příběhy a~historky. Nejvíce děje je
v~1.~světové válce. Švejk se dostává do vězení, na policii, do blázince,
po vypuknutí války jede do války na vozíku, poduřuje a~dostane se opět
do vězení, pak se stane sluhou feltkuráta Katze, ten ho ale propije
a~prohraje v~kartách. Potom Švejk putuje k~nadporučíkovi Lukášovi. Švejk
ukradne psa a~oba jsou posláni  do Budějovic. Švejk se ale ztratí
a~dostane se na policii. O~Švejkovi jako o~osobě se nikdy nic nedovíme,
jen, že otec byl  profesor, že má bratra a~obchod se psy a~je úředně
uznán za blba. Švejk se nedá charakterizovat, není to jednoznačná osoba,
neustále kolísá hodnocení mezi úředně uznaným blbem a~chytrákem, který
se chce ostatním vysmát.  Švejk není autobiografická postava, má jen
společné rysy. Oběvuje se zde obrovská spousta postav, ale ani jedna
z~nich není vyloženě záporná, postavy jsou zvláštní. Čistě zápornými
postavami  jsou militaristé, kteří válku obdivují, odrodili se od národa
-- tajný agent, kadet Brigter, poručík Dub. Celý román je satirou na
válku, zobrazuje absurditu, vojenské mašinerie, což vyplývá z~postavy
Švejka, který provede do důsledku rozkazy a~stáve vypráví příhody, které
jsou často založeny na kontrastu mezi velkým a~malým, jazyková stránka
Švejkových historek vyznívá komicky a~groteskně. Oběvují se části, které
připomínají až odborný text, ale i~vulgarismy, knižní jazyk, hovorový
jazyk, autorská řeč\dots{} Jazyk postav vystihuje jejich
charakteristiku.

\section{Legionářská literatura}
\pojem{Legionářskou literaturu} píší autoři, kteří bojovali na východní
frontě v~českých legiích v~Rusku. Autoři zobrazují boj československých
legií, jejich vznik (od r.~1916), rozklad. Legionáři bojovali až do roku
1918, přesunutí na západní frontu.  Celá řada autorů z~různých pohledů
hodnotí boj československých legií z~různých pohledů, proto se názorově
liší.

\jmeno{Rudolf Medek} oslavuje hrdinství československých legií v~dramatu
\dilo{Plukovník Švec.} Hrdinou je plukovník, který řeší rozpor v~legiích,
ale legie se rozpadne, nakonec spáchá sebevraždu.

\delic

\jmeno{Josef Kopta} -- \dilo{Hlídač č.~47} -- hlavní hrdina je hodný
a~pasivní hlídač na tati, který je obviněn z~vraždy, kterou nespáchal. Aby
se zachránil, dělá hluchého. To ho izoluje, ale vinu to z~něho nesmylo
a~lidé si stále myslí, že je vrahem.

\delic

%DOPLNIT -- řeka-Řeka ... ???
Nejvýznamější dílo legionářské literatury napsal \jmeno{Jaroslav
Kratochvíl}. Dvoudílný román \dilo{Prameny} zobrazuje osud českého
intelektuála ve válce. Ten prožívá zajetí, složité proměny v~carském
Rusku -- dobu revoluce. Je psán zvláštní formou -- \uv{řeka.} Skládá se
z~několika příběhů v~kterých se osudy prolénají. Snaží se vyznat ve světtě
kolem nich. Název je symbolický.

%Dalšími autory legionářské literatury jsou \jmeno{Václav Kaplický} %???

\section{Expresionismus v~české literatuře}
\pojem{Expresionismus} je umělecký směr vzniklý na počátku 20.~století.
U~nás se projevuje po 1.~světové válce. Zachycuje rozumem nekontrolované
vnitřní stavy a~pojevy lidské psychiky. Vychází ze životního pocitu
reaguje na krizi evropského myšlení. Autoři mají pocit, že člověk je
odcizen od světa, odráží se rozpad hodnot.

%\jmeno{Ladislav Klíma} DOPLNIT %??? 

\jmeno{Jan Weiss} je představitelem expresionalismu, surrealismu
a~vědecko-fantastické prózy. Oběvuje se u~něj výrazná fantazie, knihy jsou
na rozhraní snu a~skutečnosti, jsou až halucivní.  \dilo{Dům o~1000
patrech} je antiutopie, ideál budoucnosti. Odehrává se v~\uv{podzemním
mrakodrapu,} příběh je kombinován s~detektivkou, hlavní hrdina je
detektiv. Zvláštní důraz je kladen na smyslové vjemy, hlavně na sluch
a~zrak.

\section{Imaginativní Próza}
\emph{Imaginativní próza} je ovlivněna moderními uměleckými směry,
hlavně poetismem a~surrealismem.

Jedním z~nejvýznamějších českých autorů je \jmeno{Vladislav Vančura}.
Patří do \emph{organizace avargardních spisovatelů}, tj. do skupiny
autorů, kteří chtějí posunout literaturu kupředu.

Jeho díla patří k~opozici soudobé literatury, velice zjednodušovala
jazyk. Podle Vančury se umění nesmí soustředit jenom na soudobou
problematiku, ale musí mít širší platnost. Vybírá si náměty, které mají
obecnou platnost, jako láska, nenávist či dobrota. Experimentuje
s~formou i~s~jazykem děl. Prolíná několik jazykových vrstev -- základem je
knižní vyjadřuvání, používá přechodníky, jmenné tvary přídavných jmen,
achaismy i~neologismy. Komentuje děj a~obrací se na čtenáře, vyjadřuje
sympatii a~antipatii.

Postavy jsou velice různého charakteru, ale všichni žijí naplno, milují
život, nevzdávají se při první překážce, jsou stavěni proti přítomnosti.

Na počátku 20.~let psal Vančura povídky ovlivněné poetismem. Román
\dilo{Pole orná a~válečná} je psán formou románu
\emph{Řeka}.\footnote{řada příběhů, které se prolínají} Rychle se
střídají scény spojené na základě asociace, jsou řazeny volně. Odsuzuje
zde nesmyslnosti války, obžalovává ji a~zdůrazňuje její absurditu. Román
končí ironicky -- tam, kde začal -- u~hrobu neznámého vojína, místního
idiota \emph{Františka Řeky}, který byl nejen vrah, ale i~zloděj.

V~lázeňském městečku Karlovy Vary se odehrává děj novely \dilo{Rozmarné
léto.} Schází se tu tři přátelé. Vedou poklidný život, který naruší
příjezd kouzelníka s~krásnou \emph{Annou}. Všichni tři se do ní
zamilují, nakonec se však všechno vrací do starých kolejí a~kouzelník
odejde. Příběh je parodií na přízemnost maloměšťáků.

Ve 30.~letech dochází k~dalšímu zjednodušování jazyka a~začíná sílit
epičnost. \dilo{Markéta Lazarová} je protipól proti Jiráskovi. Vychází
z~rodinné kroniky rodiny Vančurů z~Břemnic. Odehrává se v~18.~století, ale
nejde o~zobrazení minulosti, ale o~současnost. Polemizuje se
současností, která je pokrytecká a~lidé nejsou schopni opravdového citu.
Vančura vstupuje do děje a~komentuje jednání postav a~polemizuje se
čtenářem.

Později se jeho tvorba opět mění. Píše romány, v~kterých se snaží
zachytit vývoj celé společnosti. \dilo{Obrazy dějin národa českého} měl
být původně cyklus dějin, který měli napsat různí spisovatelé. Vzniká
jako reakce na Mnichov.

\section{Společenská próza}
Společenská próza se zabývá pravdivým zobrazením společnosti.

\jmeno{Ivan Olbracht} prolíná realismus s~romantismem -- pravdivě
zobrazuje společnost, vybírá si postavy na kraji společnosti.
V~1.~období jeho tvorby (do konce 1.~světové války) je ovlivněn generací
buřičů a~anarchistů. Hrdinové jsou individualističtí, vyřazeni ze
společnosti -- zlí samotáři. Dostali se tam vinou společnosti, ale
i~vlastní, a~mstí se proto společnosti. Roli haje psychologie postav.

\dilo{O~zlých samotářích} jsou tři povídky. Hrdinové jsou z~okraje
společnosti, všichni prošli nějakým žalářem a~mstí se za to společnosti.
První povídka, \emph{Joska, Ferko a~Paulina}, je příběh tří vyděděnců,
kteří se sejdou. Neshodnou se, a~proto se rozejdou a~zůstávají sami.
Druhá povídka, \emph{Rasík a~pes}, je příběh o~rasovi, který táhne svoji
ženu světem, všichni se ho štítí. Zkrotí zlého psa bitím a~krutostí, pes
ho sice poslouchá, ale nenávidí. Když mu zemře žena, hledá lásku. Hledá
ji u~psa, nakonec proti sobě stojí dvě rozzuřená zvířata. \emph{Bratr
Žak} drží pohromadě rodinu komediantů, je to nejstarší syn. Odchází, aby
našel angažmá. Po odchodu se rodina rozpadá. Otec se upije, sestra se
provdá, bratr zabije neviného člověka.

Ve 20.~letech píše Olbracht angažovanou prózu, ovlivněnou sociálním
realismem. Typickým dílem je \dilo{Anna proletářka.} Hlavním hrdinou je
venkovská dívka, která pracuje jako služka v~Praze. Seznámí se s~mladým
dělníkem. Postupně si začíná uvědomovat, že za svoji budoucnost musí
bojovat. Vrcholí za prosincové stávky 1920. Kompozice je nepromyšlená,
postavy jsou schematické. Důraz je kladen na psychologii postav.

V~30.~letech je na vrcholu své umělecké tvorby. Inspiroval ho pobyt na
Zakarpatské Ukrajině, kde je nádherná příroda, ale oproti tomu obrovská
bída, lidé jsou ovlivněni předsudky a~pověrami. Vrcholem je román
\dilo{Nikola Šuhaj loupežník,} který je psán formou románové balady.
Roli hraje vykreslení přírody, závěr je tragický. Je psán básnickým
jazykem, platnost je nadčasová, děj v~přesně vymezeném úseku, období
1.~světové a~po světové válce.

Hlavní postava je sutečná postava vojenského zběha \emph{Nikoly}, který
uteče z~armády, skrývá se a~loupí. Po válce chce normálně žít, ožení se
s~\emph{Eržikou}. Uteče zbojníkům a~sám se zbojníkem stane. Loupí,
krade, přidávají se k~němu i~další. Nakonec je na něj vypsána odměna.
\emph{Nikola} je nepolapitelný, lidé si o~něm říkají legendy -- bohatým
bere, chudým dává. Za \emph{Eržikou} chodí \emph{Svozil}, a~proto ho
\emph{Šuhaj} zabije. Když \emph{Šuhaj} onemocní, zločiny za něho páchají
ostatní. Začnou se ho bát, a~tak ho zabijí sekerami, ale odměnu na něj
vypsanou nedostanou.

\delic

Sociální a~společenskou prózu píše \jmeno{Marie Pujmanová}. Napsala trilogii
\dilo{Lidé na křižovatce,} \dilo{Hra s~ohněm,} \dilo{Život proti smrti}

\delic

\jmeno{Marie Mayerová} byla ovlivněna novinářstvím. Zpočátku píše
ženskou prózu o~postavení žen ve společnosti. \dilo{Havířská balada}
rodina \emph{Hudce}, který se po stávce dostal na černou listinu, odchází pak
do Německa, kde pracuje pod jiným jménem. Přichází tam jeho dívka
\emph{Milka}, ale on musí narukovat do armády. Umírají jim synové. Po
válce se vrací do Česko-Slovenska. Nemá práci, a~tak se živí žebrotou.
Dílo je rozděleno na tři části, každá část má jiného vypravěče. V~první
části vypráví \emph{Rudla Hudec}, je psána ich formou, hovorovou
češtinou. Druhá část je psána ve 3.~osobě nezaujatým, reportážním
stylem, zachycuje hejhoší období. Třetí část je forma neúplného dialogu,
vypravěčkou je \emph{Milka}, vede rozhovor s~přáteli.

\section{Katolická literatura}
Základem je hluboká víra v~boha. Oběvuje se mysticismus, konzervatismus,
řada autorů má blízko k~ruralismu. Nejdůležitější je vztah k~bohu.

\delic

\jmeno{Jaroslav Durych} je významným představitelem katolické
a~meziválečné prózy. Je to básník i~prozaik, dramatik a~publicista.
Významná je umělecká stránka jeho díla. V~jeho knihách, povídkách
a~románech se pojevuje náboženská víra a~soucit s~chudými. Symbolický
charakter vyniká v~realistických vykresleních skutečnosti prolínajících
se s~pohádkou, snem nebo pověstí. Novela \dilo{Sedmikráska} obsahuje
pohádkové prvky s~poetickou atmosférou. Je to symbolická novela, která
o~štěstí a~hledání lásky. Sedmikráska spojuje lásku a~čistotu s~chudobou.

Jeho historické prózy vycházejí z~důkladné znalosti historických
pramenů. Hlavním činitelem děje nejsou lidské masy, ale důležité jsou
výsznamné osobnosti, které nejsou ale hlavními postavami. Cílem této
prózy není vycchovávat národ. Důraz je kladen na vnitřní prožitky postav
a~mezilidské vztahy. Hlavním momentem je láska a~smrt. Připomíná barokní
pohled na svět, baroko považuje za vrchol dějin. Používá výrazně
expresionistické obazy založené na kontrastu. \dilo{Rekviem} aneb
\uv{menší valdštejnská trilogie} jsou tři povídky, které se odehrávají
těsně po Valdštejnově smrti. Oběvuje se obraz hrůz třicetileté války.
Poslední povídka \emph{Valdice} dává do kontrastu Valdštejnovu dřívější
velikost s~jeho smrtí.

\section{Demokratická literatura}
Demokratickou literaturou se rozumí tvorba spisovatelů kolem
\emph{Lidových novin} -- tzv.~pragmatická generace (\emph{Karel Čapek,
Josef Čapek, Eduard Bass, Karel Poláček, František Langer}).  Přiklání
se k~literatuře hradního křídla -- parodie, jak zobrazovat společnost,
nesmí podlamovat důvěru lidí v~budoucnost. Vyhýbali se velkým námětům,
zabývali se prostým člověkem, všedními starostmi a~životem. Nehodnotí
skutečnost, uznávají relativnost pravdy -- neexistuje pravda, je pouze
relativní, každý má svoji pravdu a~nemá nikdo nemá právo ji někomu brát.

\subsection*{Karel Čapek}
Nejznámějším představitelem české literatury a~nejvýznamějším
představitelem demokratické liteatury je \jmeno{Karel Čapek}. Všímá si
maličkostí a~vyvozuje závažné a~společenské problémy lidské tvorby.
Vystupuje proti fašismu.

Jeho dílo je ovlivněno reakcí na 1.~světovou válku, obrací se
k~utopickým námětům. Varuje před možností zneužití techniky. Obrací se
k~\emph{filosofii pragmatismu}, to mu umožňuje smiřovat společnost. Ve
30.~letech se mění s~dobou i~jeho tvorba a~ustupují smiřující závěry. 

Jako novinář píše kritiky, vyniká jako autor reportáží a~fejetonů,
založil nový typ článku -- \emph{sloupek}. Na svět má humanistický
pohled a~jeho články vycházejí i~knižně -- \dilo{Zahradníkův rok,}
\dilo{Jak se co dělá} nebo \dilo{Měl jsem psa a~kočku.} Píše cestopisné
reportáže. Všímá si v~nich staveb, ale přednost dává lidem a~společnosti
-- \dilo{Italské listy,} \dilo{Anglické listy,} \dilo{Výlet do Španěl,}
\dilo{Obrázky z~Holandska,} \dilo{Cesty na sever.} Napsal \dilo{Hovory s~TGM}, 
což jsou rozhovory s~\emph{Tomášem G.~Masarykem} (neobsahuje otázky). 
Psal \emph{apokryfy}, které vyvracejí známé i~neznámé pravdy literatury.
Měl sklon k~hovorovému vyjadřování, nadměrně používal ukazovacích
zájmen, ale měl i~obrovskou slovní zásobu a~často používal synonyma

Pro děti napsal \dilo{Desatero pohádek,} což je sbírka moderních
pohádek. Mizí z~nich zázračnosti i~když vystupují zázračné postavy. Děj
utváří obyčejní lidé a~je zasazen do současné civilizace.
(\dilo{Dášenka, čili život štěněte})

Čapek překládal z~cizích jazyků -- \dilo{Moderní francouzská filosofie},
ale nejvíce psal povídky. Společně s~bratrem \emph{Josefem} píše povídky
ovlivněné civilismem -- \dilo{Zářivé hlubiny,} \dilo{Krakonošova zahrádka.}
Povídky na oslavu člověka, techniky a~světa psal i~před 1.~světovou válkou.
% + + + + + + + + + + + + + + + + +

Později píše sám. V~reakci na válku prožívá pocit úzkosti a~pesimismu,
existenciální pocit. Nedůvěřuje možnostem lidského poznání, lidé k~sobě
nedokáží najít cestu.

Je ovlivněn filosofií pragmatismu a~realismu. Oběvuje se u~něj zájem
o~detektivky. Neotické,\footnote{noetika se zabývá otázkou lidského
poznání} krátké detektivní povídky \dilo{Povídky z~jedné kapsy} jsou
psány hovorovým stylem. Na začátku se oběvuje záhada, která se logickou
cestou nakonec vyřeší. (\emph{Rekord}) \dilo{Povídky z~druhé kapsy} jsou
justiční povídky, řeší otázku spravedlnosti, jak ztrestat vinu. (povídka
\emph{Soud}) Řada z~nich spojena s~komisařem Mejzlíkem. Projevuje se zde
humanistická láska k~lidem.

Čapkovy romány mají utopické náměty. Varuje před zneužitím techniky,
řeší společenskou problematiku a~reaguje na nástup fašismu a~soudobou
společnost. Používá hovorový styl. \dilo{Továrna na absolutno} je psána
formou fejetonu, je to satira na zneužití techniky. Román je založen na
objevu karburátoru, který dokáže beze zbytku spalovat mateiál a~generuje
velké množství energie. Velmi rychle se začíná množit a~tam, kde se
objeví se začínají dít divné věci. Při spalování se uvolňuje
\uv{absolutno.} V~tomto okamžiku začíná satira na církev, vědu, noviny,
tisk, zjišťuje se, že absolutno není jen jedno, ale každá země má své
absolutno. Začínají války, lidé se ničí navzájem. Závěr je smiřující --
v~poslední bitvě si vojáci uvědomí, že boj je zbytečný a~přestanou
bojovat. V~epilogu se zbylí vojáci sejdou na \uv{zabijačce,} všechny
karburátory byly zničeny, jen jeden zůstal v~pralese u~domorodců -- vše
se může opakovat.

\dilo{Krakatit} není klasický román. Prolínají se zde různé žánry --
pohádka, detektivka, dobrodružné žánry, nevíme, zda je to skutečnost,
nebo jen iluze ing.~Prokopa. Prokop vynalezne výbušninu s~obrovskou
výbušnou silou. Výbušnina vybuchne z~neznámého důvodu a~zraněný Prokop
prozradí tajemství výroby svému známému. Pozdě si uvědomí, že to může
být zneužito proti národu. Snaží se nalézt svého známého a~zabránit
výrobě. Nepodaří se mu to zastavit a~nastane velký výbuch. Prokop se
probere a~setká se se starcem (bůh), který mu říká, že člověk má dělat
jen malé věci, které jsou k~prospěchu lidem. Postavy mají symbolická
jména.

Společnou problematiku, zda je možné poznat pravdu o~člověku, má
\emph{neotická trilogie} -- \emph{Hordubal, Povětroň} a~\emph{Obyčejný život.} 
Liší se navzájem náměty. \dilo{Hordubal} je klasická románová balada.
Děj se odehrává na Zakarpatské Ukrajině. \emph{Huraj Hordubal} se vrací
domů. Při důlním neštěstí si všichni o~něm myslí, že je mrtev. Žena se
sblíží s~čeledínem a~jeho dcera ho nezná. Hordubal je prostý, milý
a~hodný. Je nalezen mrtev, zjistí se, že jeho srdce bylo probodnuto
jehlicí, ale až po smrti. U~soudu se ale nic nezjistí. Nakonec se ztratí
i~jeho srdce.

\dilo{Povětroň} je neotický román. Během bouřlivé noci dojde k~havárii
letadla. Pilot je odvezen do nemocnice, ale nikdo ho nezná. Nad jeho
lůžkem se sejdou tři postavy --- mladý lékař, milosrdná sestra, pacient
-- básník. Každý z~nich vypráví o~pilotovi příběh, každý do něj promítá
svůj vztah k~životu, světu, ale nikdo o~tom člověku pravdu nezjistí\dots{}

\dilo{Obyčejný život} je příběh starého železničního úředníka na penzi,
který si začíná psát zápisky ze svého života. Začíná si uvědomovat, že
se vůbec nezná. Z~toho vyplívá, že nelze poznat pravdu o~člověku,
protože se člověk nezná ani sám. Neexistuje absolutní pravda, je jen
relativní.

\dilo{Válka s~mloky} je satirický utopický román. Je satirou na fašismus,
vědu, celou kapitalistickou společnost. Byli objeveni mloci velice
podobní lidem a~velice učenliví. Oběví je několik lidí -- kapitán van
Toch, obrátí se na přítele Bondiho a~naučí je bránit se žralokům. Začíná
satira na vědu, církev, publicistiku, vykořisťování\dots{} Lidé začínají
zkoumat mloky. Mloci přejímají užitečné, ale nepřejímají city, fantazii,
umění. Mloci se začnou bouřit proti lidem, potřebují více prostoru. Mají
svoji hymnu a~svého vůdce, kvůli mělčině potápějí části zemí, některé
země s~nimi dál spolupracují a~mčky přijímají potápění země a~jiné činy.
Končí rozhovorem autora se svým svědomým, dává jediné východisko, že
mloci se nakonec postaví proti sobě. Skládá se ze tří knih. První je
psána formou vypravování, připomíná povídky, druhá je psána
publicistickým stylem -- útržky novin, kteé si schovával pan Povondra,
třetí část jsou fiktivní reportáže. Oběvuje se ironie a~satira. Je to
nejvýznamější protifašistické dílo.

%\dilo{První parta}

Čapkovo drama je ovlivněno expresionismem. Postavy jsou loutky, které
mají dokázat charakter dramatu. Témata si vybírá utopická a~alegorická.
Oběvují se smiřující závěry nebo závěry umožňující různý výklad. \emph{(Více
o~dílech \dilo{R.~U.~R.,} \dilo{Ze života hmyzu,} \dilo{Bílá nemoc}
a~\dilo{Matka} je v~sekci \ref{sec:capekdrama} na~straně~\pageref{sec:capekdrama})}

\subsection*{Josef Čapek}
\jmeno{Josef Čapek}, bratr \emph{Karla Čapka}, básník, prozaik, malíř,
dramatik a~výtvarník, psal společně se svým bratrem. \dilo{Stín
kapradin} je baladická novela, roli hraje psychologie, příroda. Jde
o~příběh dvou pytláků, kteří zabili hajného a~potom utíkají před policií.
\dilo{Kulhavý pokoutník} je zvláštní próza -- esejistická kniha, autor
medituje o~životě, vyplývá odhodlanost. Narozdíl od \emph{Stínu
kapradin} ostatní díla nemají děj, jsou spíše lyrická. Od Karla se liší
hlubším pojetím děl. Píše i~literaturu určenou dětem -- \dilo{Povídání
o~pejskovi a~kočičce.}

%\subsection*{Eduard Bass}
% DOPLNIT ...