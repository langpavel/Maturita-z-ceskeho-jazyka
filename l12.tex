\chapter[Vývoj českého divadla]{Vývoj českého divadla a~dramatu od počátku do konce 19. století}
\section{Vznik českého dramatu}
\pojem{Světské drama} vzniklo původně z~náboženských her, které se hrály
o~Velikonocích a~o~Vánocích. Nejstarší dochované české dílo je
studentská hra \dilo{Mastičkář,} která byla vložena do náboženské hry,
kdy tři Marie jdou pomazat Ježíše Krista a~cestou se zastavují u~kupce
s~vonnými mastmi. Ten se je snaží přesvědčit, že jeho mastě jsou
nejlepší.

\section{Pololidová a~lidová tvorba}
\emph{Pololidová tvorba} -- přechod mezi oficiální literaturou a~lidovou
slovesností, psána většinou o~měšťanech, byla učena nižší šlechtě
a~měšťanům. Patří sem \pojem{knížky lidového čtení}, 
\pojem{interludia}\footnote{fraška, krátká lidová hra, má pobavit}
-- vysmívá se sedlákům i~vulgárně.

\jmeno{Václav František Kocmánek} napsal 7 interludií a~básnickou skladbu --
\dilo{Lamentatio Rusticana} (Nářek měšťanů) -- psanou česky.
Její součástí je \emph{Selský otčenáš} tj.~báseň, jejíž poslední verš v~každé
sloce tvoří otčenáš.

\section{První české divadlo -- Bouda}
Prvním vlasteneckým divadlem bylo divadlo \jmena{Bouda}{divadlo!Bouda} na současném
Václavském náměstí, které pořádalo pravidelná divadelní představení pro
české obecenstvo. V~letech 1786---1789 se tu hrály česky a~německy
vlstenecké hry -- též osvícenského a~protifeudálního zaměření. Jako
damatici zde půsovili \jmeno{Václav Thám}, \jmeno{P. Šedivý},
překladatel \jmeno{Karel Ignác Thám} (z~angličtiny a~němčiny).

\section{Divadlo v~národním obrození}
Hrálo se ve \jmena{Stavovském divadle}{divadlo!Stavovské}, vznikají
kočovné divadelní společnosti, divadlo se hraje ochotnicky -- na hry měl
vliv preromantismus.

\section{Drama}
Zakladatelem moderního dramatu je \jmeno{Josef Kajetán Tyl}. Používá
realistické prvky, převažují tři druhy dramatu.

Prním typem jsou \emph{dramatické obrazy ze života}. Náměty bere ze
současného života, ukazuje vady lidského charakteru, hrdinové odkrývají
nedostatky ostatních postav, konflikt končí smírně.
Patří sem \dilo{Paní Marjánka, matka pluku,} \dilo{Paličova dcera} nebo \dilo{Chudý kejklíř.}

Dalším druhem jsou \emph{dramatické báchorky}. Realistické zobrazení se
prolíná s~pohádkovými motivy. \dilo{Strakonický dudák} -- Švanda je syn
lesní víly Rosavy, která mu očarovala dudy, se vydává do světa, kde chce
vydělat peníze, aby si mohl vzít dorotku, ale také proto,
že~nejdůležitější jsou peníze. Je to slaboch -- miluje Dorotku i~vlast,
ale snadno zradí. Teprve vlivem událostí a~lásky Dorotky a~Rosavy si
uvědomí, že nejdůležitější je domov a~láska.

\emph{Historické drama} nezobazuje vřdy minulost, jde o~hry politické 
a~aktuální. Vyzvedává úlohu lidu v~boji za svobodu.
Můžeme zde jmenovat \dilo{Kutnohorští havíři,} \dilo{Jan Hus,} \dilo{Jan Žiška} 
a~\dilo{Krvavé křtiny aneb Drahomíra a~její synové.}

\section{Drama v~období májovců a~lumírovců}
\jmeno{Jaroslav Vrchlický} psal dramata. Náměty čerpal z~antiky a~z~českých
dějin. \dilo{Hippodamie} je trilogie, která připomíná antickou hru
(mluvené slovo a~hudba).

\section{Divadlo u~nás po 1. světové válce}
\subsection*{Režiséři}
Jako režisér činohry pražského \jmena{Národního
divadla}{divadlo!Národní} působí od počátku století \jmeno{Jaroslav Kvapil}.
Snaží se sjednotit všechny divadelní složky do jednoho celku -
kontextovou scénickou hudbu, bodové osvětlení, impresionistické
nepodrobné kulisy. Od herců vyžaduje tzv. \pojem{civilní} projev, bez patosu. Na
scéně ND uvádí \emph{Shakespeara, Ibsena, Čechova} a~\emph{Dyka (Krysař, Milá 7
loupežníků, Zmoudření Dona Quijota)}. K~jeho vlastním dílům patří
\dilo{Princezna Pampeliška} a~libreto \emph{Rusalka} pro Dvořáka.

Oproti tomu režisér \jmena{Vinohradského}{divadlo!Vinohradské} a~později
i~\jmeno{Národního divadla}{divadlo!Národní} \jmeno{Karel Hugo Hilar}
prosazuje v~divadelní tvorbě expresionismus. Přichází s~novinkami jako
bylo \emph{otáčivé jeviště} nebo zadní projekce.

\subsection*{Osvobozené divadlo}
Počátky \jmena{Osvobozeného divadla}{divadlo!Osvobozené} spadají do roku
1925, kdy student filosofie \jmeno{Jiří Frejka} spolu s~přáteli
z~konzervatoře uvedl Molierovu frašku \dilo{Cirkus Dandin,} ovšem
upravenou po svém, s~dávkou studentské recese. Postupně uvádějí další
hry, zejména futuristů a~poetistů (\emph{Apollinaire, Breton, Nezval}).
O~rok později pokřtil Frejka svoji amatérskou scénu na \emph{Osvobozené
divadlo} a~to se stalo jednou ze sekcí \emph{Devětsilu}.

V~roce 1927 \emph{Frejka Osvobozené divadlo} opouští a~spolu
s~\jmena{Emilem Františkem Burianem}{Emil František Burian} zakládají
divadlo \jmena{Dada}{divadlo!Dada}. Mezitím ale do \emph{Osvobozeného
divadla} vstupuje \jmeno{Jaroslav Ježek}, který hned zpočátku zaujal
svými klavírními improvizacemi.

Kromě toho sem přicházejí \jmeno{Jiří Voskovec} s~\jmena{Janem
Werichem}{Jan Werich},  oba studenti práv na Karlově univerzitě. Než
našli stálou scénu ve \emph{Vodičkově ulici} (dnes divadlo \emph{ABC}),
vystřídali působiště v~\emph{Malostranské besedě} a~v~\emph{Nuselském divadle}. 

Proslavili se hned prvním představením \dilo{Vest Pocket Revue}.
I~základem dalších her je dadaistický slovní humor a~písňové texty, zatím
ještě proti nikomu nezaměřené. Takový je \dilo{Ostrov Dynamit,}
\dilo{Fata morgána,} \dilo{Sever proti Jihu} nebo \dilo{Golem}. Jejich
režisérem byl \jmeno{Jindřich Honzl}, který režíroval i~všechny další
hry \emph{V+W}.

V~souvislosti s~hrozbou fašismu v~Itálii a~Německu se posléze v~jejich
hrách začíná objevovat politická satira. První taková hra je
\dilo{Caesar,} jehož diktaturu prožívají hlavní hrdinové -- Římané Bulva
a~Papula. Diplomatické protesty vyvolalo uvedení hry \dilo{Osel a~stín,}
kde V+W opět za použití antického námětu kritizují fašistickou
diktaturu. V~závěru totiž použijí autentické Hitlerovy výroky.
Protihitlerovská je i~hra \dilo{Svět za mřížemi.} Diktátor, tentokrát
mexický, vystupuje i~v~\dilo{Katovi a~bláznovi.} \dilo{Balada z~hadrů}
se inspiruje životem \emph{Fran\c coise Villona}. Během této hry vznikla
i~první forbína, kdy \emph{V+W} prokládali představení krátkými výstupy.
V~nich vtipně a~kriticky reagovali na nejaktuálnější události, čímž
prokázali spoji schopnost improvizace a~kontaktu s~publikem. V~další
hře, \dilo{Rub a~líc,} \emph{V+W} coby vagabundi Krev a~Mlíko bojují
proti koncernu Noel. V~režii Martina Friče rovněž natočili filmovou
verzi -- pod jménem \dilo{Svět patří nám.} Ze současnosti byla i~hra
\dilo{Těžká Barbora.} \dilo{Pěst na oko} se opět vrací do historie
a~v~několika příbězích z~různých dob vyzdvihuje úlohu prostého člověka
v~dějinách. Všem hrám je společný charakter hlavních postav, V+W jsou vždy
dva kumpáni, kteří procházejí světem, ať už současným nebo minulým,
glosují jeho nedostatky a~bojují za jeho lepší tvář. Nedílnou součástí
her byla jejich hudební složka, písně vzniklé za spolupráce
s~\jmena{Jaroslavem Ježkem}{Jaroslav Ježek}, které ovlivnily další
písňovou tvorbu a~jsou často známější než hry, ze kterých pocházejí.

Kromě zmíněného snímku \dilo{Svět patří nám} natočili V+W další filmy,
jako \dilo{Pudr a~benzín,} \dilo{Peníze nebo život} a~\dilo{Hej Rup!}
Přestože \dilo{Hej Rup!} kritizuje kapitalistický režim a~líčí vznik
jakéhosi dělnického hnutí, nevyhýbá se ani parodii na tehdejší
\uv{budování socialismu} v~SSSR.

Vzhledem k~sílícímu politickému tlaku bylo v~listopadu \rok{1938}
divadlo uzavřeno a~\emph{V+W} emigrují do Francie a~USA, kde se
\emph{Voskovec} živí v~Hollywoodu a~jako divadelní herec. Po válce se
oba vrací domů, jen \emph{Werich} však zůstává i~po 1948. Pak se věnuje tvorbě
pro děti -- \dilo{Finfárum} -- a~spolupráci s~\jmeno{Miroslavem
Horníčkem}{Miroslav Horníček}.

\subsection*{Divadlo E. F. Buriana - D 34}
Po působení v~divadle \emph{Dada} zakládá roku 1933 herec, hudebník,
a~režisér \jmeno{Emil František Burian} divadelní scénu nazvanou \jmena{D
34}{divadlo!D 34} podle druhé poloviny sezóny 1933---1934. Narozdíl od
Osvobozeného divadla je jeho divadlo režisérské, zaměřuje se na uvádění
starších i~současných prozaických i~básnických děl, ovšem
aktualizovaných do moderní podoby. Uvádí díla \emph{Moliera,
Shakespeara, Puškina, Goetha, Klicpery,} ale i~\emph{Haškova Švejka}
nebo \emph{Nezvalova dramata}. Na motivy lidové poezie a~barokních
lidových her sestavuje hry \dilo{Vojna} a~\dilo{První} a~\dilo{Druhá
lidová suita.} V~neradostné době třicátých let se tak snaží oslavovat
český národ a~jeho dávné kořeny.

\subsection*{František Langer (1888 - 1965)}
Značnou část své tvorby věnuje dramatu i~\jmeno{František Langer}.
Povoláním byl lékař, později působí jako dramaturg ve \emph{Vinohradském
divadle}. Úspěch mu přinesly jeho veselohry, první \dilo{Periférie} je
situována mezi zloděje žijící na pražském předměstí -- v~Košířích. Řeší
tu problém zločinu, svědomí a~trestu.

Hra nazvaná \dilo{Velbloud uchem jehly} vypráví opět o~životě prostých lidí na
předměstí, tentokrát ale o~úspěchu chudé praktické dívky.

Netradičně, jako \uv{divadlo na divadle} je pojatá hra
\dilo{Dvaasedmdesátka.} Žena odsouzená na dvacet let pro vraždu manžela
zinscenuje ve věznici představení, v~níž vylíčí své životní osudy. Opět
na periférii mezi zločince se vrací hrou \dilo{Obrácení Ferdyše
Pištory,} kde se hlavní hrdina, kasař a~pasák, vrací do normálního
života.

Velké popularity dosáhla komedie \dilo{Grandhotel Nevada,} ve které se
Langer vysmívá šarlatánství a~propaguje zdravý přírodní život.

\subsection*{Karel Čapek (1890 - 1938)}
\label{sec:capekdrama}
Ve dvacátých letech se na dramatickou tvorbu zaměřuje i~slavný prozaik
\jmeno{Karel Čapek}. Nepočítáme-li starší komedii dell'arte \dilo{Lásky
hra osudná,} byl jeho prvním dramatickým dílem Loupežník. Jako protiklad
k~\emph{Mahenovu Jánošíkovi}, který pomáhal ostatním, je Čapkův
\dilo{Loupežník} romantický hrdina, bouří se proti konvencím
představovaným rodiči jeho lásky \emph{Mimi}.

Světový ohlas vzbudila hra \dilo{R. U. R.} (Rossum's Universal Robots).
Varuje v~ní před přetechnizovaností moderního světa. V~Čapkově hře se
lidstvu vymstí v~okamžiku, kdy se \emph{roboti}, které lidé vyvinuli
jako své pomocníky, postaví proti nim. Lidstvo je sice vyhubeno,
a~roboti neznají princip výroby, autor ale vidí budoucnost v~citovém
vzplanutí mezi dvojicí Robotů a~v~nezničitelnosti podstaty života.

Spolu s~bratrem \jmena{Josefem}{Josef Čapek} píše alegorickou satirickou
komedii \dilo{Ze života hmyzu.} Dala by se přirovnat k~bajce, druhy hmyzu zde
totiž zastupují negativní lidské vlastnosti (motýli - přelétavost,
chrobáci - hrabivost, jepice - egocentrismus, vůdce mravenců -
válkychtivost). Vítězství dobra a~povrchnost lidského snažení
symbolizuje tulák, který se z~pozorovatele stane účastníkem, když
nakonec s~výkřikem odporu \uv{Ach, ty hmyze! Ty pitomý hmyze!} rozdupe
mravence i~s~vůdcem.

Společným dílem obou bratří je i~\dilo{Adam Stvořitel.} Přes skeptický
názor, že svět není ideální, se autoři utěšují tím, že lepší se vytvořit
nedá, můžeme jen napravovat chyby.

Se \emph{Shawovým} dramatem \dilo{Zpět k~Methusalemovi} polemizuje Čapek
ve hře \dilo{Věc Makropulos.} Na rozdíl od něj tvrdí, že
\uv{šedesátiletý život je přiměřený,} pokud je prožit aktivně, že
dlouhověkost je zbytečná, pokud není život čím naplnit.

Stejně jako \emph{V+W} i~Čapek reaguje svou dramatickou tvorbou
na~politickou situaci v~Evropě. Výrazně protimilitaristická je hra
\dilo{Bílá nemoc.} Ukazuje v~ní státeček, na jedné straně fanatizovaný
\emph{Maršálem}, ženoucím všechny do expanzívní \uv{vítězné války,}
na~druhé straně sužovaný zákeřnou \emph{Bílou nemocí} -- malomocenstvím.
Doktor \emph{Galén} vynalezne lék proti ní, hodlá jím \emph{Maršála}
uzdravit, jen když válku ukončí. Když \emph{Maršál} nakonec souhlasí,
Galén je na cestě k~němu ušlapán zfanatizovaným davem.

O~poznání komornější je další protiválečná hra, \dilo{Matka.} Od té
postupně odchází do války manžel a~všichni synové. O~jejich tragickém
osudu se dovídá z~rozhlasu a~z~fiktivního rozhovoru s~jejich dušemi.
Posledního, nejmladšího \emph{Toniho}, chce ušetřit. Po vyslechnutí
zpráv o~válečných hrůzách, o~bombardování škol a~zabíjení dětí, mu však
sama podává pušku do ruky se slovy \uv{Jdi!}

Protiválečně zaměřené jsou i~Čapkovy romány, zejména \dilo{Válka s~Mloky.}

\subsection*{Josef Skupa (1892 - 1957)}
Klasikem českého loutkového divadla se stal Josef Skupa, který právě
v~době mezi světovými válkami vytváří svou originální dvojici
\emph{Spejbla a~Hurvínka}. 