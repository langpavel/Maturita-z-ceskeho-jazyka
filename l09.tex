\chapter{Světový a~český romantismus}
\section{Vznik a~hlavní rysy}
Romantismus vzniká začátkem 19. století jako reakce na změny ve
společnosti -- feudální systém je buď svržen revolucí, jako ve Francii,
nebo se mění v~osvícenský absolutismus, jako u~nás. Do života lidí
proniká kapitalismus a~s~ním i~nejistota. Mnoho z~nich se musí
přizpůsobit zcela novým podmínkám. Tam, kde se neuplatnily ideály
francouzské revoluce (a~to také ve Francii), roste nespokojenost
obyvatel.

Není divu, že se umělci ve svých námětech vracejí do minulosti, únik
z~reality hledají ve starobylých a~přírodních námětech. \emph{Také
romantický hrdina je charakteristický konfliktem.} Cítí se výjimečný,
nemůže však své záměry uskutečnit kvůli rozporu s~okolím. Proto je to
většinou vyhraněný individualista. \emph{Dává přednost svým pocitům před
rozumem},  to je znak celého romantismu, který se tak staví proti
osvícenství, přestože z~něj vychází. Život romantického hrdiny je
ovlivněn láskou, ať už k~ženě nebo vysněnému ideálu, vždy je to ale
láska nešťastná. Protože se většinou autor se svým hrdinou ztotožňuje,
najdeme i~v~životopisech romantických autorů mnoho tragických lásek.

Romantismus již jako renesance nenapodobuje klasické antické vzory,
krásu hledá zcela jinde. Ideálem přestává být pravidelnost a~řád,
nahrazuje ho originalita a~přirozenost. Rozdíl je patrný i~v~úpravě
zahrad -- anglický park proti francouzskému. Tomu odpovídá i~prostředí
romantických děl -- \emph{jezera, hřbitovy, zříceniny, případně exotické kraje}.

\section{Anglie}
\subsubsection*{George Gordon Byron (1788 - 1824)}
Předním básníkem anglického romantismu je \jmeno{George Gordon Byron}.
Pocházel sice ze staré aristokratické rodiny, odchází ale od ní a~svojí
ženy a~většinu života tráví v~Itálii.

Stejný osud jako on má i~hrdina rozsáhlého veršovaného eposu 
\dilo{Childe Haroldova pouť,} také se rozchází se svým okolím a~vydává
se na cestu Evropou. Uchvacují ho přírodní scenérie, slavná historie
jednotlivých zemí a~její kontrast se současným stavem. Epos je rozdělen
do čtyř zpěvů -- o~svobodě Španělska, Řecka, Anglie a~Itálie.

Stejné motivy nacházíme v~poemách\footnote{veršovaná lyricko-epická
povídka, je po svém tvůrci zvaná také byronská} \dilo{Kor\-zár,}
\dilo{Džaur} a~\dilo{Lara,} odehrávajících se většinou v~orientálním
prostředí.

Odpor hrdinů k~tradicím se stupňuje ve filosofických dramatech
\dilo{Manfred} a~\dilo{Kain,} kde přerůstá až
v~\pojem{titanismus}.\footnote{odpor proti Bohu} V~nedokončeném eposu
\dilo{Don Juan} využívá osudů hlavního hrdiny k~satiře na soudobou
společnost.

\subsection*{Percy Bysshe Shelley}
Titanistický názor vyjadřuje i~\jmeno{Percy Bysshe Shelley} v~básnickém
dramatu \dilo{Odpoutaný Prométheus.} Přestože námět čerpá z~antiky,
ústřední je zde opět konflikt -- mezi bohem Jupiterem a~Prométheem jako
lidskou touhou po dobru.

\subsection*{Walter Scott}
Jako tvůrce romantických balad ze skotské historie začínal \jmeno{Walter
Scott}. Brzy ale zakládá nový žánr - historický román. Prvním román
\dilo{Waverley} věnuje skotskému povstání, děj \dilo{Ivanhoe} situuje do
doby Richarda Lvího srdce. Scott dokázal minulost oživit přesvědčivým
líčením dobových zvyků a~vedlejších postav, hlavní hrdina však zůstává
romanticky idealizován.

\section{Francie}
\subsection*{Victor Hugo}
Ve Francii romantismus prosazuje prozaik, básník a~dramatik \jmeno{Victor Hugo}.
Přestože byl synem Napoleonova generála, byl vychován k~nenávisti proti
němu a~republice. Tento názor později mění, ve svých historických dílech
si ale zachovává objektivní nadhled. Hned ve svém prvním díle, dramatu
\dilo{Cromwell} se zabývá historií a~v~úvodu k~němu vyjadřuje základní body
romantismu.

V~básnických sbírkách \dilo{Východní zpěvy} a~\dilo{Zpěvy soumraku}
oslavuje boj Řeků za svobodu a~francouzskou revoluci. Místo vznešených
hrdinů zde však líčí osudy prostých a~chudých lidí.

Když odchází z~politických důvodů do dobrovolného vyhnanství, rozhodně
se neizoluje od ostatního světa. Píše zde rozsáhlý trojdílný cyklus
básní \dilo{Legenda věků,} kde popisuje vývoj lidstva. Jeho budoucnost
vidí optimisticky, hlavně ve vědě a~pokroku.

Z~Hugovy prozaické tvorby jsou nejznámější romány \dilo{Chrám Matky boží
v~Paříži} a~\dilo{Bídníci.} První z~nich popisuje život lidí kolem
pařížské katedrály od králů až po žebráky. Ve výběru postav používá
autor hojně protikladů. Všemi nenáviděný zvoník \emph{Quasimodo} je
ochoten se pro druhé obětovat, oproti tomu kněz \emph{Frollo}, ačkoliv
se tváří zbožně, je ve skutečnosti ničema.

Také Bídníky umisťuje mezi nejnižší vrstvu společnosti. Ústřední hrdina,
nespravedlivě odsouzený trestanec \emph{Jean Valjean}, se mění díky
biskupovi Mirielovi v~ušlechtilého člověka. Stává se ochráncem osiřelé
Cosetty a~poté co zbohatne, pomáhá svým jměním chudým a~bojuje proti
svému úhlavnímu nepříteli, vězeňskému dozorci a~později policejnímu
komisaři Javertovi. Smyslem tohoto díla je boj proti předsudkům
a~nespravedlivým společenským zákonům.

V~exilu vznikli i~\dilo{Dělníci moře,} román o~rybáři Gilliatovi a~jeho
boji proti nespravedlnosti a~proti přírodním živlům, které překonává pro
záchranu parníku.

V~románě \dilo{Devadesát tři,} psaném již po návratu z~vyhnanství, se
znovu vrací k~francouzské revoluci, ze které si vybírá boj revolučních
a~konzervativních sil na venkově, ve Vendée. Hlavním dějem je konflikt
mezi Gauvainem a~Cimourdainem, který --- přestože byl jeho vychovatelem
a~přítelem --- získá za revoluce odlišné názory. Když Gauvain zradí
republiku tím, že pomůže k~útěku vůdci \uv{bílých} Lantenacovi, musí
ho~Cimourdain potrestat. Kromě toho Hugo na mnoha příkladech ukazuje,
že~válka si nevybírá, a~že jí za obět padají i~nevinní lidé. Celým
příběhem totiž prolíná příběh matky, která se bez ohledu na nebezpečí
vydává hledat své ztracené děti. Příběh rozhodně není podáván
realisticky, je plný nečekaných zvratů a~nakonec i~ona matka zasáhne
do~běhu událostí.

\subsection*{Stendhal}
\jmena{Stendhalovi}{Stendhal} hrdinové se již nechovají čistě
romanticky, svůj konflikt s~okolím řeší vnějším přizpůsobením, stávají
se pokrytci.

V~románě \dilo{Červený a~černý} se dřevařský syn Sorel stává bez
přesvědčení knězem. Lepšího společenského postavení dosahuje podvody
a~intrikami s~využitím milenek. V~\dilo{Kartouze parmské} (vlastně
\emph{Kartuziánský parmský klášter}) líčí osudy italského šlechtice,
který se po návratu z~války nemůže vyrovnat se změněnými domácími
poměry, a~odchází do kláštera.

\subsection*{Alfred de Musset (1810 - 1857)}
Své subjektivní pocity vylíčil básník \jmeno{Alfred de Musset} ve svém
autobiografickém románě \dilo{Zpověď} dítěte svého věku. Své osudy nechá
prožívat Octava, který jako celá jeho generace, nenachází v~životě nic
než smutek a~beznaděj.

\section{Rusko}
\subsection*{Alexandr Sergejevič Puškin (1799 - 1837)}
Ruským představitelem romantismu je \jmeno{Alexandr Sergejevič Puškin}.
Za svou tvorbu byl celý život pronásledován, a~stála ho i~život.

V~prvních dílech, poemách \dilo{Kavkazský zajatec} a~\dilo{Cikáni}
oslavuje Byronovo téma -- svobodu. Pak se ale jeho obdiv k~Byronovi
a~jeho romantickým hrdinům ztrácí, dál je Puškin líčí spíše ironicky.

Tak je tomu v~poemě \dilo{Evžen Oněgin.} Hlavní hrdina poznává Taťánu,
ta se do něj sice zamiluje, ale Oněgin ji odmítne. Místo toho
vyprovokuje žárlivost manžela její sestry Lenského a~v~souboji ho
zabije. Když se po letech s~Taťánou setká a~zamiluje se do ní, ta na
křivdu nezapomene a~odmítne ho.

Osud karbaníka, který hledá štěstí v~hazardních hrách, ale přijde
o~všechno, vypráví poema \dilo{Piková dáma.}

V~historickém dramatu \dilo{Boris Godunov} sice Puškin líčí dávný souboj
o~moskevský trůn, řeší však zde aktuální politické otázky. Činy cara
Petra Velikého oslavuje svou historickou poemou Měděný \dilo{jezdec.}

\subsection*{Michail Jurjevič Lermontov (1814 - 1841)}
Básník \jmeno{Michail Jurjevič Lermontov} také nebyl oblíben carským
režimem a~stejně jako Puškin i~on byl vypovězen na Kavkaz.

Pohádku O~princezně Tamaře přeměňuje v~souboj dobra a~zla v~poemě
\dilo{Démon.} Zlo je podle něj způsobeno bezmocností dobra, tím
vysvětluje tragický osud ušlechtilých romantických hrdinů.

Oproti tomu se román \dilo{Hrdina naší doby} věnuje aktuálním otázkám.
Hlavní hrdina Pečorin cestuje po světě, cítí se být povznesen nad
ostatní. Jelikož nenachází smysl svého života, stává se \uv{zbytečným}
člověkem. Právě problém ztracené lidské existence je společný Lermontovi
i~Puškinovi.

\section{Polsko} \subsection*{Adam Mickiewitz (1798 - 1855)} Polský
básník \jmeno{Adam Mickiewitz} prožil většinu života ve vyhnanství
v~Rusku a~emigraci v~západní Evropě. Aktivně se totiž účastnil hnutí za
osvobození své země, která byla tehdy rozdělena mezi Prusko, Rusko
a~Rakousko. Začínal jako klasicista, brzy se však přiklání k~romantismu ve
svých \dilo{Baladách a~romancích.} V~poemě \dilo{Konrád Wallenrod} se
vrací do minulosti Litvy, a~ukazuje její boj za svobodu proti křižákům.
Do svého dramatu \dilo{Dziady} promítá kromě pohanských obřadů
a~vlastních milostných citů hlavně víru ve spasitelské poslání národního
utrpení. Epos \dilo{Pan Tadeáš} situuje mezi drobné zemany v~době
Napoleonova tažení do Polska. Ústředním dějem je příběh lásky
příslušníků dvou nepřátelských rodů -- narozdíl od Romea a~Julie však
skončí šťastně smírem.

\section{Německo}
%\subsection*{Johann Wolfgang Goethe} -- presunuto do l07.tex

\subsection*{Heinrich Heine}
Představitelem hnutí \emph{Mladé Německo} je básník \jmeno{Heinrich Heine}.
Snaží se bojovat proti absolutismu a~měšťácké morálce. Lidovou poezií se
nechává inspirovat v~básnické sbírce \dilo{Kniha písní,} svůj kritický
postoj ke společnosti vyjadřuje v~\dilo{Zimní pohádce.} 

\subsection*{Jakob a~Wilhelm Grimmové}
V~této době vydávají také bratři Grimmové svůj soubor lidových německých
pohádek.

\section{Romantismus u~nás}
\subsection*{Karel Hynek Mácha (1810 - 1836)}
V~Čechách se romantismus projevuje hlavně v~životě a~díle Karla Hynka
Máchy.

Narodil se v~Praze na Malé Straně, po studiích na gymnáziu, právech
a~filosofii odchází za prací advokáta do Litoměřic. Malebná krajina kolem
nich, a~nejenom ta --- Mácha rád cestoval po celé Evropě --- se mu stává
inspirací pro romantická díla.

Počátky jeho tvorby jsou zastoupeny básněmi, ve kterých kombinuje své
sny a~naděje s~cestopisnými zážitky a~výjevy z~historie. Příkladem může
být \dilo{Těžkomyslnost} nebo \dilo{Pouť krkonošská,} kde jako poutník
vystupuje na Sněžku a~setkává se zde se zemřelými mnichy, kteří ožívají
na jeden den v~roce.

Historický námět najdeme i~v~povídce \dilo{Křivoklad,} která měla být
původně součástí čtyřdílného souboru \dilo{Kat.} Objevuje zde rozpor
mezi králem Václavem IV. a~jeho katem. Zatímco král je se svým vysokým
postavením nespokojen, kat po něm naopak touží.

Ze dvou povídek se skládají autobiografické \dilo{Obrazy ze života mého.}
Ve \dilo{Večeru na Bezdězu} líčí lyricky opět své cestovní zážitky.
Druhá povídka -- \dilo{Márinka,} má zřetelnou písňovou formu, skládá se
ze~zpěvů a~meziher, z~básně i~prózy. Mácha se zde vrací ke své první,
tragické lásce.

Podobně jako Puškin i~on v~povídce \dilo{Cikáni} oslavuje svobodu,
přidává zde motiv msty, stejně jako v~pozdějším Máji zde hlavní
hrdina objevuje ve svém milostném sokovi svého otce.

\subsubsection*{Máj}
Nejvýznamnějším dílem je jeho dílo poslední, lyrickoepická poema
\dilo{Máj.} Vlastní příběh je poměrně jednoduchý a~je obsažen v~prvním
zpěvu. Vilém je svým otcem vyhnán z~domu a~stává se vůdcem loupežníků --
\uv{strašlivým lesů pánem.} Nevědomě se mstí svému otci za svedení své
milenky Jarmily. Za tento čin je zatčen a~odsouzen.

Ve vězení, kde Vilém čeká na popravu, se odehrávají další zpěvy
a~intermezza. Vilém přemýšlí nad smyslem lidské existence, uvažuje
o~naději na posmrtný život, dospívá však k~závěru, že ho čeká jen
\uv{věčné nic.} Svůj tragický osud si uvědomuje v~kontrastu mezi krásami
okolní přírody a~smrtí. Tu zastupuje sbor duchů, který přijímá Viléma
mezi sebe. Nakonec Mácha splývá se svým hrdinou, což nejlépe vyjadřuje
závěrečné zvolání \uv{Hynku !, Viléme !!, Jarmilo !!!}

Máchovo dílo se zprvu nesetkalo s~porozuměním kritiky. Je svými
současníky odmítán pro svůj pesimismus a~nedostatek vlastenectví. Jeho
význam objevuje teprve jeho přítel \emph{Karel Sabina}. Májovci si volí
jeho odkaz jako základ svého almanachu.