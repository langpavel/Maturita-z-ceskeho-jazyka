\chapter[Světový realismus a~naturalismus]{Realismus a~naturalismus ve světové literatuře}
%\section{Realismus}
\pojem{Realistická} díla se snaží o~pravdivé zachycení skutečnosti. Jeho prvky
se oběvují v~literatuře renesanční, klasicistické, osvícenecké a~v~dílech romantiků.
Jako umělecký směr vznikl po polovině 19.~století a~vrcholí ve 20.~století.
Realismus je podporován rozvojem společnosti, kapitalismem a~vědeckým pokrokem.
\pojem{Kritický realismus} ukazuje na nedostatky společnosti a~chce její nápravu.
Vzniká sociologie.\footnote{věda o~společnosti}

Realistická díla jsou charakterizována několika znaky. Autor se snaží
pravdivě zachytit smrt, hrdiny jsou obyčejní lidé, postavy mají společné
i~individuální rysy, charakter postav se vyvíjí, autor se neúčastní děje 
a~velkou roli hrají popisy.

\section{Anglie}
V~Anglii se realismus objevuje na počátku 18. století a~má zde nejdelší
tradici.
Lze sem zařadit i~\dilo{Robinsona Crusoa} \jmena{Daniela Defoa}{Daniel Defoe},
nebo \jmena{Jonathana Swifta}{Jonathan Swift}.

Skutečný rozvoj realismu v~Anglii nastává až v~19.~století. Od jiných
zemí se liší anglický realismus humornými prvky, pohled na skutečnost se
též liší -- díla vyznívají smírně.

Nejvýznamějším představitelem je \jmeno{Charles Dickens}. Na jeho díla
má vliv život. Kritizuje v~nich společnost prostřednictvím osudu malých
dětí vyrůstajících bez domova, v~sirotčinci. Hned jeho první román
\dilo{Kronika Pickwickova klubu} ho proslavil. \dilo{Oliwer Twist} je
příběh malého chlapce, který pracoval u~rakváře od kterého utekl ke
zlodějům. Krom toho napsal \dilo{Nadějné vyhlídky} nebo 
\dilo{David Copperfield,} což je jeho autobiografie.

% jak se to pise ??
Společnosti v~Anglii se vysmívá a~kritizuje jí \jmeno{Jhackeray}.
Jeho \dilo{Kniha o~snobech} pojednává o~přetvářce a~lidech,
kteří předstírají, že jsou vznešení.

% jak se to pise ??
Na osudu ukazuje kitický oraz Anglie \jmeno{Thomas Hardy}. Hlavní
hrdinky jsou ženy. \dilo{Tess D'ubervilles} je tragický příběh.

\section{Francie}
Ve Francii se objevuje realismus od 19.~století, ale převažuje zde
romantismus. Přechod tvoří \emph{Stendhal -- Červený a~černý}.

Za zakladatele francouzského realismu se považuje \jmeno{Honor\` e de
Balsac}. První tzv.~\uv{černé romány} (romantické, dobrodružné příběhy)
píše pro peníze. Již realistický je cyklus \dilo{Vidská komedie,} který
zahrnuje přes 120~románů. Cílem je zobrazit francouzskou společnost
v~polovině 19.~století, kde ukazuje, jak je společnost ovládána
majetkem. Jádrem cyklu jsou romány \dilo{Otec Goriot,} \dilo{Ztracené
iluze,}
\dilo{Lesk a~bída kurtizána.}

\emph{Otec Goriot} je bohatým obchodníkem, který měl dvě dcery. Obě
velice miloval a~výhodně je provdal -- jednu za šlechtice, druhou 
za bankéře. Dal jim veškerý majetek, ale ony nejsou rády a~stydí se 
za něj, protože nemá peníze. Prolíná se zde příběh studenta,
který přichází s~iluzemi\dots{} Stará se o~Goriota, ale po jeho smrti
se chová tak, jak by měl -- bez ohledu na ostatní lidi.
%\emph{Lesk a bída kurtizána}

Dovršitelem realismu ve Francii je \jmeno{Gustav Flaubert}. Trpí
epilepsií, ale mezi záchvaty mu zbývá čas na psaní. Narozdíl od Balzaca
se nesnaží zachytit celou společnost. V~jeho románech není souvislá
dějová linie, jde spíš o~zachycení psychologie člověka. Byl postaven
před soud kvůli románu \dilo{Paní Bovaryová} pro urážku mravnosti. Ema
byla vychována v~klášteře. Provdá se za lékaře \emph{Bovary}, ten je
hodný, čestný, ale nesplňuje její představy. Ema se snaží zaměstnat na
venkově, ale poté si najde milence, upadá do dluhů a~páchá sebevraždu.

Dal vzniknout novému typu historického románu, kde je snahou
pravdivě vylíčit minulost, ale důraz je kladen na psychologii postav.
\dilo{Salambo} je kněžka, sestra Hanibala, ze starého Kartága.

Napsal též román \dilo{Citová výchova,} kde mladík strácí iluze
o~společnosti a~lásce.

\section{Rusko} 
\subsection*{Nikolaj Vasiljev Gogol}
V~rusku se realismus objevuje v~tvorbě omantiků od poloviny 19.~století.
Díla \jmena{Nikolaje Vasiljeva Gogola}{Nikolaj Vasiljev Gogol} patří
k~romantismu i~kritickému realismu. Čistě romantické jsou jeho povídky
z~ukrajinské historie. \dilo{Taras Bulka} pojednává o~boji kozáků proti
Polákům. \dilo{Večery na dědince nedaleko Dikaňky} vycházejí z~ústní
lidové slovesnosti. Děj se odehrává ve vesnici a~objevují se představy
o~vlkodlacích, upírech a~čarodějnicích.
\dilo{Petrohradské povídky} poukazují na sociální nespravedlnost,
chamtivost. Objevuje se zde hořký humor, satira, ironie. 

Satirické komediální drama \dilo{Revizor} se odehrává v~malém městečku,
kam přijíždí obyčejný malý úředník. Ten je považován za obávaného
revizora, čehož využívá. Podvod je ale odhalen, protože do města přijíždí
skutečný revizor.

Satirický román \dilo{Mrtvé duše} měl mít tři části, ale zůstala pouze
jedna. Drobný úředník se rozhodne získat úspěch podlézáním a~podvodem.
Skupuje mrtvé nevolíky, protože nebudou platit daně, po sčítání lidu
mu pozemky propadnou.

\subsection*{Ivan Sergejevič Turkeněv}
Společensko--psychologické romány píše \jmeno{Ivan Sergejevič Turkeněv}.
Jeho díla jsou ovlivněna životem, mají stručný a~rychlý spád děje.
Píše také novely a~povídky, \dilo{První láska,} \dilo{Jarní vody,}
\dilo{Asja.}

Cyklus \dilo{Lovcovy zápisky} je sbírka povídek, kterou spojuje 
do jednoho celku vypravěč, který s~přítelem prochází ruským venkovem.

Román \dilo{Otcové a~děti} srovnává povahové rysy mladých a~starých.
Nejde ale o~generační spor.

\subsection*{Lev Nikolajevič Tolstoj}
Jeden z~nejslavnějších světových autorů je \jmeno{Lev Nikolajevič Tolstoj}.
Začíná psát, když se účastní Krymské války. Ve svých dílech vystupuje proti
nevolnictví, hájí zájmy rolníků, dochází k~tomu, že život každého by se měl
přiblížit k~životu rolníka, odmítá civilizaci. Má na něj vliv křesťanství
a~učení \emph{Petra Chelčického}.

Ve sbírce \dilo{Sevastopolské povídky} se vrací ke krymské válce.
Zdůrazňuje zde úlohu vojáka a~lidu.

Napsal trilogii autobiografických románů \dilo{Dětství, chlapectví, jinošství.}

Novým typem historického románu je čtyřdílná \dilo{Vojna a~mír.} Děj zobazuje
osud ruské společnosti v~době Napoleonských tažení. Domnívá se, že dějiny
nedělají velké osobnosti, ale lid. Vyzvedává úlohu lidu v~boji proti Napoleonovi.
Je psán formou románové kroniky, nemá základní děj. Objevuje se velké množství
postav, epizod a~příběhů, které dohromady dávají obraz doby.
Základem jsou příběhy tří postav -- Andreje Bolkonského, důstojníka ruské amády,
Piera Beruchova, nerozhodného zbohatlíka, a~Nataši Rostovové, mladé dívky, která
se pomalu mění vlivem okolí.

Dvoudílný román \dilo{Anna Kareninová} je založený na osudu tří rodin.
Je to kritika ruské společnosti, srovnává s~\emph{Flaubertovou Paní Bovariovou}.

\subsection*{Fjodor Michailovič Dostojevskij}
\jmeno{Fjodor Michailovič Dostojevskij} byl pro účast v~povstání odsouzen
k~smrti, ale trest mu byl změněn na vyhnanství na Sibiř. Pobyt ve vyhnanství
ho změnil a~pozonost soustřeďuje na sociální nespravedlnost a~utrpení.
\dilo{Zápisky z~mrtvého domu} je jediným jeho autobiografickým románem,
vrací se zde k~pobitu na Sibiři.

\dilo{Zločin a~trest} se odehrává v~Petrohradě. Hlavní postavou je nadaný
student, který kvůli nedostatku peněz zabije lichvářku. Otázka, zda
měl na to právo, nebo ne, ho přivede na Sibiř, kam ho doprovází
prostitutka Soňa.

\dilo{Idiot} je kníže Miškin, který trpí epilepsií a~duševní chorobou.
Protože chce pomáhat lidem, je pokládán za \uv{idiota.}

Román \dilo{Bratři Karamazovi} vyjadřuje jeho pohled na svět. Je to
příběh o~velkém inkvizitorovi, odehrává se ve Španělsku. Přichází
Kristus, kteý je jako kacíř upálen. Příběh se soustředí kolem tří synů
a~jednoho nevlastního. Otec je nalezen mrtev, z~vraždy je obviněn
nejstarší syn. Ten nese trest za prostředního syna. Nevlastní syn spáchá
sebevraždu. Děj není ukončen.

\subsection*{Anton Pavlovič Čechov}
\jmeno{Anton Pavlovič Čechov} je zakladatelem moderního lyrizovaného
dramatu. Je autorem výborných humoristických a~psychologických povídek.
Zobrazuje uskou společnost a~protestuje proti ní. Vystupují lidové
postavy, postavy ze všech vrstev i~zbytečných lidí.

Humoristická povídka \dilo{Chameleon} pojednává o~psovi, který pokousal
trhovce. Policie neustále hledá majitele a~mění názor. Když se zjistí,
že pes patří manželce generála, pokutu platí trhovec.

V~povídce \dilo{Pavilon č. 6} se primář v~pavilonu pro duševně choré
snaží zlepšit podmínky svých svěřenců. Nakonec je považován za blázna
a~ocitá se mezi nimi.

V~Čechovových dramatech se na dějišti téměř nic
neodehrává,\footnote{tzv. dramata nudy} chybí dramatický konflikt.
Důležitější, než příběh je podtext, který má hlubší význam. Hrdinové
jsou dobří, nadaní, čestní a~hledají smysl života. Můžeme jmenovat
\dilo{Tři sestry,} \dilo{Strýček Váňa.} Ozdobou statku Roněvské a~jejího
bratra Gajeva je \dilo{Višňový sad,} který se stává majetkem kupce
Lopachina. Ten chce nádherný sad zničit a~rozprodat ho na parcely. 
Obraz kvetoucího višňového sadu a~jeho zánik je symbolem ruské šlechty.

\section{Polsko}
V~polsku se projevuje \pojem{pozitivismus}. Vznikají historické romány,
které pravdivě zachycují minulost, vycházejí ze studia pramenů.
\jmeno{Henryk Sienkiewicz} píše formou románové kroniky.
Trilogie \dilo{Křižáci} se odehrává na přelomu 14. a~15.~století,
objevuje se \emph{Jan Žižka}. Další trilogie \dilo{Ohněm a~mečem}
se odehrává v~době obrany Polska před vpádem Tatarů.
\dilo{Quo Vadis} je z~antického Říma.

\section{Naturalismus}
Naturalismus navazuje na kritický realismus, na jeho vznik má vliv
rozvoj věd, jako sociologie či genetika. Vzniká ve Francii, autor je
v~pozadí, neobjevuje se v~ději.

\subsection*{Emil Zola}
\jmeno{Emil Zola} vytvořil teorii naturalismu, zdůrazňuje vliv vědeckého
prostředí a~dědičnosti. První naturalistické dílo se jmenuje
\dilo{Tereza Raquinová.}

Teorii naturalismu aplikoval na cyklus románů \dilo{Rougon--Macquartové}
s~podtitulem \uv{Přírodovědný a~sociální dějepis jedné rodiny za 2.~císařství.}
Na rodokmenu, který si vytvořil, chce ukázat vliv dědičnosti a~prostředí.
Vytvořil asi 20 románů.

\dilo{Zabiják} je alkohol. Na jedné rodině ukazuje, jak alkohol dokáže zničit
vztahy i~postavení. \dilo{Germinal} čerpá z~dělnického prostředí.
Napsal též trilogii \dilo{Tři města.}

\subsection*{Anatole France}
Dalším představitelem naturalismu je básník, prozaik, esejista
\jmeno{Anatole France}.
\dilo{Historie našich dnů} zachycuje dobu kolem \emph{Drayfusovi aféry}.
\dilo{Ostrov tučňáků} je alegorie a~satira na francouzskou společnost.