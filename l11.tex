\chapter[Vývoj českého realismu]{Vývoj českého realismu 19. století}
První projevy realismu jsou již v~\emph{Máchově Márince}, nebo  v~tvorbě
\emph{J.~K.~Tyla} (venkovská realistická próza). Výrazněji se objevuje
až okolo poloviny 19.~století. Kritický realismus vzniká jako směr
v~80.~letech 19.~století.

\subsection*{Karel Havlíček Borovský}
Zakladatelem realismu v~české literatuře je \jmeno{Karel Havlíček
Borovský}.  Byl to spisovatel, kritik, psal satiry, představitel
\emph{austroslavismu}.\footnote{všechny slovanské národy mají tvořit jeden národ v~čele s~Ruskem}

Narodil se v~Borové u~Přibyslavi, studoval v~Německém Brodě (později
Havlíčkův Brod), studoval filosofii, chtěl být knězem,
ale na~\emph{Šafaříkovo} doporučení jel do Ruska, odkud posílal dopisy --
\dilo{Obrazy z~Rus} je kritika ruských poměrů.

Byl novinářem a~přispíval do \emph{Pražských novin} a~\emph{Národních novin}.
Nejlepší z~časopisu \emph{Slovan} shrnul do \dilo{Kutnohorské epištoly.}

Byl pokrokovým literárním kritikem, odmítal \uv{co je české, to je hezké.}
Teorii literární kritiky sepsal do spisu \dilo{Kapitoly o~kritice.}

O~jeho zatčení a~deportaci do \emph{Brixenu} napsal \dilo{Tyrolské elegie}

V~exilu napsal velké satirické skladby \dilo{Král Lávra} a~\dilo{Křest svatého Vladimíra.}  

Motiv staré irské pohádky o~králi s~oslíma ušima, satira na absolutismus
a~kritika omezenosti a~hlouposti panovníka je skladba \emph{Král Lávra},
kde kontrastuje neustálé opakování královy dobroty a~šlechetnosti s~tím,
že nechá popravit každého, kdo ví o~jeho oslích uších.

\subsection*{Božena Němcová}
\jmeno{Božena Němcová} je pseudonym pro \emph{Barboru Panklovou}. Vliv
na ni měla babička, která s~nimi několik let žila v~Ratibořicích. Vedle
\emph{Tyla} byla zakladatelkou české realistické prózy s~vesnickou
tématikou. Prolíná realistické a~romantické prvky. Ukazuje na

společenské rozdíly, naproti tomu venkov idealizuje. Ojevuje se u~ní typ 
\emph{dobrého člověka} -- mravně dobrý, obětavý, příklad pro okolí\dots{}
Záporné postavy jsou sloužící, lidé, kteří se povyšují nad svoji vrstvu.

Píše povídky metodou obrazů ze života. Nejde o~příběh, ale o~vykreslení
prostředí a~vztahů mezi lidmi. Málo děje, který je soustředěn kolem
jedné postavy -- \dilo{Karla,} \dilo{Divá Bára,} \dilo{Dobrý člověk}
nebo \dilo{Pan učitel.} 

Nejrealističtější dílo Němcové je \dilo{V~zámku a~v~podzámčí,} ale i~zde
idealizuje. Zobrazuje zde kontrast mezi zámkem (bohatí lidé) a~chudými
lidmi v~podzámčí, kde vládne bída, hlad a~nemoci.

Byla sběratelkou národní lidové slovesnosti, upravovala povídky
a~pohádky  a~sepisovala je. \dilo{Pyšná princezna,} 
\dilo{Princezna se~zlatou hvězdou na~čele,} \dilo{Sůl nad~zlato,}
\dilo{Čertův švagr,} \dots{}

\section{Historická tématika}
Historická próza má u~nás dlouhou tradici, počínaje kronikou a~národním
obrozením. Realistické pojetí je ovlivněno \emph{Tolstojem} a~\emph{Flaubertem}.
Většina má výchovný charakter, chce ukázat současníkům slavnou minulost.
Vychází se ze studia historických pramenů.

\subsection*{Alois Jirásek}
Nejvýznamějším představitelem české historické prózy je \jmeno{Alois Jirásek}.
Věnuje se jak historické, tak vesnické tématice, píše prózu a~drama.

Z~husitského období je to trilogie \dilo{Mezi proudy} -- z~dob počátků
reformního hnutí v~době vlády \emph{Václava IV.}, Vystupuje zde řada
postav, jako \emph{Jan Hus, Žiška}, základem jsou ale lidové postavy.
Končí vydáním \emph{Kutnohorského dekretu 1409}.

Trilogie \dilo{Proti všem} je stručnější. Zobrazuje husitské revoluční
hnutí od založení Tábora až do roku \emph{1421} -- vyhnání Adamitů.
Základem jsou historiké postavy \emph{Zikmund, Žižka}, ale důležitější
jsou postavy hejtmana \emph{Ctibora z~Hvozdova}, jeho dcery \emph{Zdeny}
a~kněze \emph{Bydlinského}.

\dilo{Bratrstvo} zobrazuje konec a~pád husitské slávy, bratříčky
na~Slovensku, kteří nebojují za myšlenku, ale za hmotné zájmy.

Zachycuje pobělohorskou dobu, dobu největšího úpadku českého národa.
Román \dilo{Temno} dal název celému období. Odehrává se ve vrcholu
rekatolizace (\emph{1723---1729}), je zde vykreslena barokní kultura.
Je zde zobrazeno i~svatořečení \emph{Jana Nepomuckého}.

Národní obrození Jirásek zobrazuje jako lidové hnutí. Vyzvedává úlohu
obrozenců, jak šíří český jazyk a~literaturu. \dilo{F.~L.~Věk} je
pětidílný román. Havní postavou je obrozenec \emph{Hek}, kterého si Jirásek
přejmenoval. Vystupují zde \emph{V.~Thám, Jungmann, Kramerius}.

\subsection*{Zikmund Winter}
\jmeno{Zikmund Winter} si všímá života lidí ve městě, nesnaží se zobrazit celou
dobu, ale zaměřuje se na postavy lidí, ukazuje jakým způsobem mění doba
život člověka. 

\dilo{Rakovnické obrázky} jsou zbeletrizované, mírně upravené,
kulturně-histoické obrázky, příběhy z~archívu. Nemají příběh,
ale snaží se zobazit dobu ve které lidé žili.

Povídka \dilo{Pražské obrázky} vykresluje též psychologii postav,
ale oběvuje se určitý příběh. Většina vyznívá tragicky, ale
objevuje se i~humor.

\dilo{Rozina sebranec} je příběh dívky, která je vychována v~klášteře,
zamiluje se do Itala. Ten ji ale kvůli jejímu původu nechá a~ona se vdá
za mnohem stašího řemeslníka, který se k~ní chová jako k~majetku. Ona je
mu za to nevěrná. Jeho to ale dopálí a~obědná si nájemného vraha, který
ji zabije. Nakonec umírají všichni, jen její manžel zůstává naživu.

\section{Vesnická tématika}
V~období realismu u~nás převažuje vesnická tématika. Většina autorů
pochází z~venkova, protože je vesnice pro ně dobře známou věcí, tak o~ní
píší. Na venkově převažuje idealistická představa, že lidé na venkově
jsou lepší. Nebo se ukazuje, že vztahy na venkově jsou ovládány majetkem.

\subsection*{Teréza Nováková}
\jmeno{Teréza Nováková} přikládá psychologii postav též velkou roli.
Důležitější je, jak hlavní postavy vnímají prostředí. Většinu života
prožila v~\emph{Litomyšli} a~v~\emph{Proseči}. Z~pražského prostředí
čepá v~díle \dilo{Maloměstský román,} kde zachycuje životní osudy
\emph{Boovského Zdeničky}. Později píše drobné prózy z~okolí Litomyšle,
v~kterých zachycuje sociální problémy na venkově. \dilo{Drobová polévka}
(zabijačková) patří k~\pojem{dokumentárnímu realismu}.\footnote{postavy
a~jejich jména jsou skutečné} Napsala sbírku povídek \dilo{Úlomky žuly.}

Mapsala několik monografických románů. 
\dilo{Jan Jílek} byl český evangelík, který když zjistil, že jeho
předkové trpěli za víru, odchází a~žije mezi vystěhovalci.
\dilo{Jiří Šmatlán} byl tkadlec z~Poličky. Celý život hledal
\uv{vopraudivou praudu.} Zachycuje jeho přeměnu z~katolíka
na evangelíka a~následně socialistu.
\dilo{Na Librově gruntě} sepisoval sedlák Libra kroniku kraje.
\dilo{Děti čistého živého} byla sekta. Zachycuje vznik, rozklad a~zánik sekty. 
\dilo{Drašar} byl \emph{Josef Justin Michl}, který se narodil v~Poličce,
studoval v~Litomyšli a~v~Praze na kněze, kde se seznámil 
s~\emph{Václavem Hankou} a~\emph{Františkem Palackým.}
Chtěl se stát pokračovatelem \emph{Jung\-manna}, ale moc neuspěl.
Konec života tráví sám, v~opovržení, v~Březinách u~Poličky.

\subsection*{Karel Václav Rais}
\jmeno{Karel Václav Rais} je představitelem kritického realismu. Děj
jeho románů se odehrává v~Podkrkonoší a~na Českomoravské vysočině. Byl
učitelem v~Trhové Kamenici a~v~Hlinsku, později ředitelem školy na
Vinohradeh. Psal historické povídky pro mládež výchovného charakteu.
Nepsal povídku \dilo{Výměnkáři,} kde se zabývá problematikou výměn
a~nabídek často jen zastírajících zisk nějakou výhodou.

Zobrazuje také člověka, který by byl příkladem. Dílo 
\dilo{Zapadlí vlastenci} je poděkování kněžím a~učitelům.
Učitelský pomoník Čermák odjíždí do Pasek, ale vůbec se mu tam nechce
(kvůli nejistému platu). Nakonec se zamiluje do Albínky z~fary, dostává
svoji školu a~nakonec se usadí.

\dilo{Západ} pojednává o~stařičkém faráři Kalousovi, který celý svůj
život věnoval okolí.

\dilo{Kalibův zloči} se místy blíží k~naturalismu. Kaliba byl dobrák,
kterému po smrti matky rozebrali majetek. Když si uvědomí, že ho
jeho žena podvádí a~zneužívá ho s~tchýní jenom pro peníze, zbije ji.

\section{Městská tématika}
Méně častá je v~tomto období městská tématika. Autoři navazují na
\emph{Jana Nerudu}. 

Jedním z~mála autorů je \jmeno{Ignát Hermann}, který zachycuje
světmalých obchodníků a~droných řemeslníků. Píše především povídky, ale
i~romány. \dilo{U~snědeného krámu} -- Matin Žemla si otevře obchod, ale
ten končí krachem. Upadá do rukou manželky a~tchyně a~proto páchá
sebevraždu. Příběh je tragický, ale objevují se v~něm prvky humoru.

\section*{Naturalismus}
Naturalismus se oběvuje především u~románů s~městskou tematikou.

\subsection*{Karel Matěj Čapek -- Chod}
\jmeno{Karel Matěj Čapek} pochází z~Chodska. Aby se jeho jméno nepletlo,
přidal si za jméno \uv{Chod.} Zpočátku píše
\pojem{lokálky},\footnote{místní zprávy -- je třeba znát důvěrně
prostředí} až později píše romány. Je představitelem naturalismu. Vybírá
si hrdiny z~okraje společnosti a~člověk je \uv{redukován na pudy.}
\dilo{Kašpar Lén mstitel,} podle kterého byl natočen film \dilo{Mstitel,}
se zabývá psychologií vzniku zločinu. Dvoudílný román pojednává o~pomstě
Léna, který se vrací po třech letech z~vojny a~zjišťuje, že dcera Mařka
se stala prostitutkou\dots{} Za vše může kupec Konopík, kterého Lén
zabije. Druhá část románu se děje u~soudu a~končí Lénovou smrtí.

Román \dilo{Turbina} pojednává o~vzestupu a~pádu českého podnikatele.
Zasahuje zde osud, příběh se mění v~tragikomickou grotesku. 









