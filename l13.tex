\chapter{Generace májovců, ruchovců, lumírovců}
\section{Májovci}
Májovci vydávají svůj první \emph{almanach Máj} roku 1858. Jeho jméno odvozují
od slavné \emph{Máchovy} básně, k~jehož odkazu se hlásí. Narozdíl od autorů
romantismu nehledají náměty v~minulosti, naopak se snaží zaměřit na
současné problémy. Českou literaturu tím chtějí pozvednout na evropskou
úroveň. S~romantismem mají společný zájem na svébytnosti a~svobodě
umělecké tvorby.

Za hlavní reprezentanty této generace považujeme \emph{Vítězslava Hálka, Jana
Nerudu, Adolfa Heyduka, Karolínu Světlou} a~\emph{Jakuba Arbesa}.

\subsection*{Jan Neruda}
\jmeno{Jan Neruda} se narodil na Malé Straně, kde jeho matka sloužila 
v~domácnosti slavného geologa \emph{Joachima Barranda}. Ten se mu
později stal inspirací pro jednu z~jeho povídek. Studuje na gymnáziu
spolu s~\emph{V.~Hálkem}, pak na filosofii a~právech. Po přechodném
zaměstnání profesora nastupuje dráhu novináře a~spisovatele. Od poloviny
šedesátých let až do konce života působí jako redaktor v~Národních
listech, pro které píše své fejetony. V~nich se věnuje aktuálním tématům
(\dilo{1. máj 1890}), ale  i~námětům z~běžného života (\dilo{Kam s~ním,}
\dilo{O~švestkových knedlících}).

Jeho prvním literárním dílem byla pesimistická básnická sbírka
\dilo{Hřbitovní kvítí,} kde protestuje mimo jiné proti nespravedlivým
majetkovým rozdílům mezi lidmi. Další sbírku \dilo{Knihy veršů} rozdělil
do tří částí. První je sociálně laděná (balada \dilo{Dědova mísa}),
další obsahuje lyriku věnovanou \uv{Otci, Matičce a~Anně,} ve třetí
vyjadřuje vlasteneckou obavu o~osud slovenského národa.

Až po dalších deseti letech vychází značně optimističtější sbírka
\dilo{Písně kosmické.} Ovlivněn dobovými astronomickými objevy se Neruda
humornou a~personifikovanou formou zamýšlí nad vztahem člověka
a~vesmíru. Je přesvědčen, že pomocí nezadržitelného pokroku lidstvo
překoná všechny problémy, mezi které patřila i~dobová situace
utlačovaného českého národa.

Ke starým legendám a~lidovým pověstem se vrací ve sbírce \dilo{Balady
a~romance,} ze kterých je nejznámější \dilo{Romance o~Karlu IV..} Pomocí
starých příběhů vyjadřuje svůj postoj k~současnému světu. Inspiruje se
také vlastním dětstvím, příkladem může být \dilo{Balada dětská.}

Zároveň vydává \dilo{Prosté motivy,} kde ve stylu lidové poezie popisuje
krásy jednotlivých ročních období.

Poslední sbírku \dilo{Zpěvy páteční,} obsahující vlastenecké motivy,
vydává posmrtně \emph{Jaroslav Vrchlický}.

Většího uznání dosáhl Neruda jako prozaik, především svými městskými
povídkami. První jejich sbírkou jsou \dilo{Arabesky,} známější jsou
\dilo{Povídky malostranské.} Neruda v~nich čerpá ze svých vzpomínek 
na rodnou Malou Stranu, do tohoto idylického prostředí umisťuje typické
pražské figurky a~jejich životní osudy. Ty jsou někdy tragické,
například povídky \dilo{Hastrman} nebo \dilo{Přivedla žebráka na
mizinu,} vyprávějící osud žebráka \emph{Vojtíška}, kterého záměrně
šířené klepy připraví o~život. Jiné zase pojímá autobiograficky
(\dilo{Jak to přišlo, že dne 20.~srpna roku 1849 o~půl jedné z~poledne
Rakousko nebylo rozbořeno}) nebo v~nich zachycuje lidské charaktery bez
určitého děje, jako je tomu v~úvodní (\dilo{Týden v~tichém domě})
a~závěrečné (\dilo{Figurky}) povídce.

I~v~románě \dilo{Trhani} se věnuje tématu lidí z~nejnižších vrstev.
Obrazy ze života dělníků pracujících na stavbě železnice spojuje
tragická postava \emph{trhana Komárka}.

Nerudovu dramatickou tvorbu tvoří především situační komedie podle
francouzského vzoru. Jeho díla \dilo{Ženich z~hladu,} \dilo{Prodaná
láska} či pokus o~tragédii \dilo{Francesco di Rimini,} inspirovanou
\emph{Dantovým Peklem}, však nevzbudila velký ohlas.

\subsection*{Vítězslav Hálek}
Vůdčí osobností mezi Májovci byl \jmeno{Vítězslav Hálek}, podobně jako
Neruda byl členem redakce \uv{Národních listů.} V~počátcích své poezie
se hojně inspiruje Máchou a~Byronem, píše poemu \dilo{Alfred} a~lyrické
sbírky \dilo{V~přírodě} a~\dilo{Večerní písně,} ke kterým ho inspiruje
jeho láska k~ženě Dorotce.

Erbenovými baladami se nechal ovlivnit v~\dilo{Pohádkách z~české
vesnice.} V~příbězích jako \dilo{Husar,} či \dilo{Dražba} zobrazuje
radostné i~smutné chvíle života venkovanů. Svůj kladný vztah ke
Slovensku Hálek vyjádřil v~poemě \dilo{Děvče z~Tater.}

V~próze se zabývá hlavně tzv. \emph{vesnickými povídkami}. Jejich
námětem je jsou obvykle generační vztahy, rodiče bránící svým dětem žít
život podle svých představ, jako je tomu v~\dilo{Muzikantské Lidušce}
nebo v~povídce \dilo{Na statku a~v~chaloupce.} Opačný problém, tedy
zanedbávání starých rodičů, řeší v~povídkách \dilo{Náš dědeček} nebo 
\dilo{Na~vejminku.} Uvádí ale i~kladné příklady, například obětavého
\dilo{Poldíka Rumaře.}

Hálkova dramata jsou převážně historická. Napsal šest veršovaných
tragédií z~našich i~světových dějin. Patří mezi ně \dilo{Král Vukašín,}
\dilo{Záviš z~Falkenštejna} a~\dilo{Král Rudolf,} který neprošel
tehdejší cenzurou.

\subsection*{Adolf Heyduk}
K~prvním básníkům almanachu Máj patřil \jmeno{Adolf Heyduk}. Úspěchu
dosáhl svou první sbírkou \dilo{Básně,} zvláště jejím oddílem
\dilo{Cikánské melodie,} kde podobně jako Mácha oslavuje svobodu. Další
sbírky přírodní lyriky (\dilo{Lesní kvítí,} \dilo{Ptačí motivy}) jsou
inspirovány jeho rodným krajem -- jižními Čechami -- nevzbudily větší
ohlas.

\subsection*{Karolína Světlá}
\jmeno{Karolína Světlá} si svůj pseudonym zvolila podle rodné vsi
manžela \emph{Petra Mužáka}, svého domácího učitele. Do~Podještědí
situuje i~děj svých vesnických povídek. V~nich obvykle vystupují
emancipované ženské hrdinky, například v~povídkách \dilo{Kantůrčice,}
\dilo{Frantina} nebo \dilo{Kříž u~Potoka.} Ve~\dilo{Vesnickém románu} se
projevují autorčiny konzervativní mravní zásady, když nechá hlavního
hrdinu \emph{Antoše Jírovce} snášet žárlivou a~sobeckou ženu, místo aby
od ní odešel. Úvahy o~víře a~mravních zásadách se objevují také
v~povídce \dilo{Nemodlenec.}

\emph{Povídky z~Podještědí} se stávají inspirací \jmena{Elišce
Krásnohorské}{Eliška Krásnohorská}, která podle nich vytvořila
\dilo{Hubičku,} libreto pro \emph{Bedřicha Smetanu}.

Děj svých povídek ale umisťuje i~do města, přesněji do Prahy. V~povídce
\dilo{Černý Petříček} líčí staropražské prostředí svého mládí, v~románě
\dilo{Zvonečková královna} se sice snaží o~zachycení doby, hlavním
motivem však zde je život v~přepychu.

\subsection*{Jakub Arbes}
Smíchovský rodák \jmeno{Jakub Arbes} přispěl do tvorby Májovců svými
romanety. Tento specifický literární druh můžeme charakterizovat jako
novelu se záhadným nebo detektivním námětem, který má překvapivé
rozuzlení, někdy fantastické, někdy podložené vědeckým zdůvodněním. To
dokázal Arbes dobře vysvětlit, vystudoval totiž tehdejší \emph{Vysoké
učení technické}.

První romaneto vydává pod názvem \dilo{Ďábel na skřipci,} brzy následují
další: \dilo{Svatý Xaverius,} \dilo{Sivooký démon,} \dilo{Ukřižovaná,}
nebo \dilo{Newtonův mozek.}

Ve svých románech zpracovává sociální témata. Román
\dilo{Štrajchpudlíci} (\emph{Stávkokazi}) líčí stávku dělníků na
Smíchově. Podobně v~díle \dilo{Kandidáti existence} se zabývá myšlenkou
na změnění světa, jakýmsi druhem socialismu. Sociálně kritický je
i~román \dilo{Mesiáš.}

\section{Ruchovci}
Autoři soustředění kolem \emph{almanachu Ruch}, vydaného o~deset let
později (1868), narozdíl od Májovců usilují o~svébytnou a~nezávislou
českou literaturu. S~tím souvisí silná orientace na české dějiny,
zvláště u~Svatopluka Čecha.

\subsection*{Svatopluk Čech (1846 - 1908)}
\jmeno{Svatopluk Čech} začíná svou tvorbu husitskými náměty, v~almanachu
Ruch otiskuje svoji první báseň \dilo{Husita na Baltu.} Tento námět dále
uplatňuje v~básnických skladbách \dilo{Adamité} a~\dilo{Žižka,}
a~v~dramatu \dilo{Roháč na Sioně.} Postupně se od historie dostává
k~současným mezinárodním otázkám, řeší je v~eposech \dilo{Evropa}
a~\dilo{Slavie.} 

Svůj vztah k~rodnému českému venkovu Čech vyjádřil v~básni \dilo{Ve stínu lípy}
a~v~poemě \dilo{Lešetínský kovář.} První představuje soubor veselých i~vážných
příběhů, které si vyprávějí vesničané při posezení pod starou lípou,
druhá příběh kováře, který brání svůj majetek proti Němcům (proto byla
tato báseň cenzurována).

Satiře se věnuje jak v~poezii, v~komickém eposu \dilo{Hannuman}
přirovnává lidskou společnost k~opičímu království, tak v~próze, svými
romány o~panu \emph{Broučkovi}. Jeho \dilo{Pravý výlet pana Broučka do Měsíce}
a~\dilo{Nový epochální výlet pana Broučka, tentokrát do 15. století} 
mu dávají příležitost kriticky zobrazit přízemnost a~zbabělost pražského
měšťáka, která zvlášť vyniká v~porovnání s~vlastenectvím husitských
bojovníků.

\subsection*{Eliška Krásnohorská}
\jmeno{Eliška Krásnohorská} se zabývala tvorbou libret ke Smetanovým
operám. Na~motivy Světlé píše \dilo{Hubičku,} dále pak \dilo{Tajemství}
a~\dilo{Čertovu stěnu.} Kromě toho se věnuje přírodní lyrice, dívčím
románům a~překladům evropských romantiků -- \emph{Puškina, Mickiewitze}
a~\emph{Byrona}.

\section{Lumírovci}
\emph{Lumírovci} navazují na Májovce ve své snaze o~světový význam české
poezie. Usilují také o~její originalitu a~tvůrčí svobodu, odsuzují
průměrnost soudobých měšťanských uměleckých žánrů. Jméno je odvozeno od
časopisu \emph{Lumír} (veden J. V. Sládkem). Mezi Lumírovce můžeme
zařadit \emph{Josefa Václava Sládka, Jaroslava Vrchlického}
a~\emph{Julia Zeyera}.

\subsection*{Josef Václav Sládek}
\jmeno{Josef Václav Sládek} se narodil ve Zbiroze jako syn zednického
mistra. Za~studiem přírodních věd odjíždí do Ameriky, kde si musí těžce
vydělávat na živobytí, pracuje např. na stavbě železnice. Po návratu
domů působí určitou dobu v~\emph{Národních listech}, účastní se i~tvorby
almanachu \emph{Ruch}, nakonec se stává redaktorem časopisu \emph{Lumír}.

Jeho první básnická sbírka \dilo{Básně} sice obsahuje intimní lyriku,
ale narozdíl od prvotin Nerudy je mnohem optimističtější. Sládek totiž
svůj osud konfrontuje s~ostatním národem. Také sem promítá své zážitky
z~pobytu v~Americe.

Také druhá sbírka \dilo{Jiskry na moři} kombinuje intimní lyriku (stesk
nad smrtí své ženy) s~vlasteneckými a~politickými básněmi. Přitom se
inspiruje lidovou poezií, podobně jako \emph{V.~Hálek} v~\emph{Pohádkách z~naší
vesnice}.

V~dalších sbírkách \dilo{Světlou stopou,} \dilo{Na prahu ráje,}
\dilo{Ze~života} a~\dilo{Sluncem a~stínem} se dále vytrácí intimní
lyrika a~nahrazuje ji stále víc zájem o~společenské otázky. Jsou také
o~poznání optimističtější, Sládek totiž našel štěstí ve svém druhém
manželství. Také původně jednoduchý a~přístupný jazykový styl se mění
v~náročnější podle vzoru Vrchlického.

To ale neplatí o~Sládkově tvorbě pro děti, kterou píše stále formou jim
přístupnou. Jeho díla \dilo{Zlatý máj,} \dilo{Skřivánčí písně}
a~\dilo{Zvony a~zvonky} se snaží děti zaujmout svým objevným pohledem
na~svět.

Neutěšená politická situace v~osmdesátých letech ho vede k~námětům
z~lidové poezie. V~selském lidu, jeho práci a~lásce k~rodné zemi vidí
šanci pro národ. Proto vznikají \dilo{Selské písně a~české znělky.}
Sedláka považuje za ztělesnění české národní povahy, obdivuje se jeho
svobodě a~nezávislosti.

Ve stejném duchu pokračuje sbírkami \dilo{Starosvětské písničky}
a~\dilo{Písně smuteční,} kde se inspiroval lidovými pohřebními písněmi.
Silně vlastenecké jsou \dilo{České písně,} kterými brání národní svobodu
i~jazyk.

Poslední soubory reflexívní lyriky \dilo{V~zimním slunci} a~\dilo{Za
soumraku} píše již pod vlivem nemoci, zamýšlí se v~nich nad svým
životem.

\subsection*{Jaroslav Vrchlický}
\jmeno{Jaroslav Vrchlický} pocházel z~Loun, studuje na gymnáziu v~Klatovech
a~v~Praze na Karlově universitě. Přehled o~evropské kultuře získal
následnými cestami po Evropě, během kterých pracuje i~jako vychovatel
v~šlechtických rodinách.

Do literatury vstupuje svou sbírkou milostné lyriky \dilo{Z~hlubin.}
I~další jeho sbírky z~počátků tvorby jsou plné radosti ze života, oslavy
mládí a~lásky -- \dilo{Eklogy a~písně,} \dilo{Poutí k~Eldorádu.}

Pak ale přichází do jeho tvorby krize. Cítí se unaven životem a~také
jeho rodinný život se hroutí. Odráží se to v~pesimistických básnických
sbírkách \dilo{Hořká jádra,} \dilo{Okna v~bouři,} nebo \dilo{Zaváté
stopy.} Nakonec se však ve své lyrické poezii opět vrací
k~optimistickému pohledu na svět, ve sbírce \dilo{Strom života} oslavuje
přírodu a~věčný koloběh života.

Podle vzoru \emph{Hugovy Legendy věků} píše epický cyklus o~historii
lidstva \dilo{Zlomky epopeje.}. Narozdíl od ní se nesnaží o~úplnost, ani
o~chronologickou souslednost, jsou to skutečně jen zlomky dějin. Dějiny
také jednoznačně nehodnotí, spíše se jimi inspiruje. Přestože věří ve
šťastnou budoucnost lidstva, nevyhýbá se kritice současnosti, kterou
přirovnává k~historickým příkladům. Podobné téma zpracovává také
v~epické skladbě \dilo{Bozi a~lidé.} V~\dilo{Selských baladách} se
věnuje protišlechtickým povstáním a~oslavuje selské vůdce, například
v~\dilo{Baladě o~Janu Kozinovi.}

Velkého úspěchu dosáhl Vrchlický jako dramatik. Platí to hlavně o~jeho
veselohrách, z~nichž se dosud hraje \dilo{Noc na Karlštejně.} Zápletka
této hry je daná zákazem pobytu žen na hradě Karla~IV. a~jeho
pochopitelným nedodržováním. Králova manželka totiž jeho zákaz poruší.
Jeho historické tragédie čerpají z~našich dějin -- \dilo{Drahomíra,}
\dilo{Bratři a~Knížata} tvoří trilogii. Další trilogii \dilo{Hippodamie}
věnoval antickému námětu. Skládá se z~částí \emph{Námluvy Pelopovy, Smír
Tantalův} a~\emph{Smrt Hippodamie}. \emph{Zdeněk Fibich} ji svou
scénickou hudbou upravil na melodram.

Vrchlický se také zabýval překlady světové poezie i~prózy. Českým
čtenářům přiblížil mimo jiné \emph{Shakespeara, Andersena, Byrona} 
nebo \emph{Goetha}.

\subsection*{Julius Zeyer}
Julia Zeyera můžeme zařadit mezi novoromantiky, hlavně díky jeho zálibě
v~našich i~cizích dějinách, mýtech a~pověstech. Narodil se v~zámožné
pražské německo--francouzské rodině. Po neúspěšných studiích se věnuje
jen literatuře. Cestuje po cizích zemích, v~Rusku je šlechtickým
vychovatelem.

Ze staročeských pověstí vychází sbírka epických básní \dilo{Vyšehrad.}
Evropskými rytířskými bájemi, hlavně francouzskými, se zabývá
v~\dilo{Románě o~věrném přátelství Amise a~Amila} a~v~\dilo{Karolinské
epopeji.}

Vrcholem Zeyerovy prozaické tvorby jsou romány \dilo{Jan Maria Plojhar}
a~\dilo{Dům u~tonoucí hvězdy.} V~obou líčí osudy krajanů, kteří utíkají
před společností do evropských velkoměst. Ve \dilo{Třech legendách
o~krucifixu} líčí příběhy třech ukřižovaných.

Ve stejné době jako Vrchlický se i~Julius Zeyer pouští do dramatické
tvorby. Také jeho dramatická díla vycházejí z~českého dávnověku.
Příkladem je \dilo{Libušin hněv} nebo \dilo{Neklan.} Větší ohlas
vzbudila jeho romantická scénická pohádka \dilo{Radúz a~Mahulena.}
Je situována do bájné slovenské krajiny a~doprovázená hudbou Josefa
Suka. 