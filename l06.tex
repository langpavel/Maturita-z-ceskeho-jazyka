\chapter[Česká pobělohorská literatura, baroko]{Česká literatura doby pobělohorské, barokní kultura, Jan Ámos Komenský} %6. otázka
Roku \rok{1620} bylo poraženo české stavovské povstání v~bitvě na Bílé hoře, což vedlo k~útlaku
české kultury až do konce 18. století.
Poprava 27 českých pánů na Staroměstském náměstí roku \rok{1621}, po které následovala konfiskace majetku,
byla odstrašujícím příkladem pro \uv{neposlušné} Čechy. Byla zakázána činnost nekatolických kněží ve městech.
Obnovené zřízení zemské (\rok{1627}) ruší platnost všech privilegií království a~jediným povoleným náboženstvím
je katolické. Nekatolíci musí přestoupit na katolictví, nebo odejít z~Čech a~Moravy.
Němčina se stává úředním jazykem vedle češtiny. V~důsledku toho mnoho nekatolíků odchází z~našeho území.
Jezuité pálí \uv{zakázané knihy,} ale na druhé straně budují školství pro všechny chlapce.
S~rostoucím vlivem němčiny upadá český jazyk a~literatura. Čeština setrvává na venkově, ale venkované
většinou neumějí číst a~psát.

Do 60. let 17. století u~nás doznívá humanismus, ale potom až do konce 18. století dochází k~největšímu
úpadku česky psané literatury. Jirásek tuto dobu nazval dobou \textit{Temna}. 

Literatura se v~tomto období dělí na \pojem{exulantskou}, \pojem{oficiální rekatolizační}
a~\pojem{lidovou a~pololidovou tvorbu}. Exulantská literatura navazuje na humanistické tradice, 
oficiální je ovlivněna barokem a~prosazuje katolické náboženství.

\section{Exulantská literatura}
Exulantská literatura navazuje na literaturu předbělohorskou, tedy na literaturu humanistickou.
Protože vznikala v~zahraničí, příliš neovlivnila současnou domácí tvorbu. Jednota bratrská 
se přesouvá do Polska.

V~zahraničí vznikají spisy, které mají informovat o~naší vlasti, například \dilo{O~státě českém}
autora \jmena{Pavla Stránského}{Pavel Stránský}.
Dějiny církve do roku 1620 v~Čechách popisuje \jmeno{Pavel Skála ze Zhoře} v~desetisvazkovém spise 
\dilo{Historie církevní.} 
\jmeno{Jiří Třenovský}, autor kancionálu, emigroval na Slovensko.

\subsection*{Jan Amos Komenský}
Jako autor naučné pedagogické literatury proslul \jmeno{Jan Amos Komenský}, biskup Jednoty bratrské.
Po~studiích v~Německu působí v~Přerově jako učitel, odtud získává své pedagogické poznatky, později 
působí jako kazatel ve~Fulneku.

Probíhající třicetiletou válku kritizoval ve spisu \dilo{Listové do~nebe.} Ta ho nakonec 
z~Fulneku vyhnala tak rychle, že nestačil zachránit svou knihovnu a~některé rukopisy shořely. Po roce 1628 je vydáním
nařízení proti nekatolíkům donucen emigrovat do~polského Lešna.
Zde píše svůj \dilo{Labyrint světa a~ráj srdce,} rozsáhlou jinotajnou skladbu, ve~které popisuje svou životní
cestu světem -- městem ho provází \textit{Všezvěd Všudybud} -- lidská zvědavost, \textit{Mámení} mu nasazuje 
růžové brýle -- alegorie na zvyk přijímat cizí názory bez rozmyslu. Při pohledu do~ulic -- zaměstnání -- vidí
všude podvodníky a~chamtivce toužící po zisku.

Problematikou výchovy se zabývá ve~spisu \dilo{Velká didaktika} -- \dilo{Didaktica magna.}
\dilo{Informatorium školy mateřské} se zabývá výchovou v~rodině. Výuku jazyků se snažil podpořit učebnicí latiny
\dilo{Brána jazyků otevřená,} ale i~obrázkovým slovníkem \dilo{Orbis Pictus} -- \dilo{Svět v~obrazech,} který se díky 
ilustracím ze všedního života stává vzorem pro jazykové učebnice. Během své cesty do~Uher dramatizuje 
Bránu jazyků a~vytváří tak \dilo{Školu hrou,} ve~které přináší soubor divadelních her využitelných při výuce. 
Ke~konci svého života pobývá v~Naardenu a~Amssterodamu, kde se zabývá myšlenkou nápravy lidské společnosti. 
Dílo \dilo{Všeobecná porada o~nápravě věcí lidských} však nedokončil.

\section{Baroko}
V~17.---18. století se ve střední Evropě a~v~katolických zemích objevuje baroko v~reakci na dobu neustálého
násilí a~válek (například třicetileté). Umění se odvrací od přírody a~vnějšího světa a~zaměřuje se na vlastní
nitro. Oběvuje se naturalistická konkrétnost nad utrpením a~bezmocností lidí, duchovnost a~mysticismus.
Dramatické napětí vychází z~protikladu ducha a~hmoty, využívá se kontrastu světla a~stínu.
V~sochařství a~malířství se promítá vnitřní napětí do pohybu postav -- vzpínajících se napětím, někdy až křečovitě
zkroucených. Objevuje se touha po poznání a~splynutí s~bohem. V~tomto období působí \jmeno{Matyáš Braun} -- sochy na Kuksu.
V~dramatu se objevuje kontrast mezi bohem a~člověkem. Smrt a~utrpení je znázorňováno naturalisticky.
Hra světla a~stínu se oběvuje i~v~architektuře. 

V~literatuře se objevují prvky kontrastu, používá se sugestivní líčení násilí války i~smrti na proti bohu -- 
dobru a~spravedlnosti. Patří sem i~Jan Amos Komenský se svou alegorií Labyrint světa a~ráj srdce.
Pro baroko je typické časté užití metafor a~přirovnání. 

\subsection*{Itálie}
\jmeno{Torquato Tasso} napsal náboženský hrdinský epos \dilo{Osvobození Jeruzaléma,} který pojednává
o~křížové výpravě a~víře v~boha.

\subsection*{Anglie}
Biblickým příběhem o~Adamovi a~Evě se inspiroval angličan \jmeno{John Milton}. Ve svém duchovním eposu
\dilo{Ztracený ráj.}

\subsection*{Německo}
Německá tvorba byla ovlivněna utrpením lidí ve třicetileté válce. Její hrůzy zachycuje 
\jmeno{Hans Jakof Christffel von Grimmelshausen}, který ji zažil již v~dětství.
Humoristický román \dilo{Dobrodružný Simplicius Simplilisimus} o~mladíkovi, který prožívá třicetiletou 
válku v~níž přijde o~rodiče, je hledáním cesty k~bohu.
Pocity zmaru a~nicoty ve svých dílech vyjadřuje dramatik \jmeno{Andreas Gryphius}.

\subsection*{Naše území}
Na našem území probíhá rekatolizace, kterou provádějí Jezuité. Literatura je psána především česky, 
znovu se oběvují legendy (\dilo{O českých svatých}) a~rozvíjí se duchovní básnictví.

Poezii s~mystickým charakterem píše básník \jmeno{Bedřich Brídl}. V~díle \dilo{Co bůh? Člověk?} se zabývá
nicotností života člověka oproti dokonalému světu božímu.
Písně psal \jmeno{Adam Michna z~Otrokovic}. Je autorem kancionálu \dilo{Česká mariánská muzika,}
vánoční koledy \dilo{Chtíc, aby spal,} ukolébavky \dilo{Hajej můj andílku} nebo milostné písně
\dilo{Pod našimi okny.}

Objevuje se i~vědecká literatura, především historická. Protože ji píší Jezuité, je husitství opomíjeno.
\jmeno{Bohuslav Balbín} se zabývá kulturními dějinami v~latinském spisu \dilo{Učené Čechy,} kde nabádá
k~vlastenectví. \dilo{Rozprava na ochranu jazyka Slovanského, zvláště pak Českého} je kniha,
kde popisuje dřívější rozkvět českého jazyka a~literatury, srovnává je se současným stavem. Zabývá se příčinami
úpadku jazyka, vyjadřuje víru, že se český jazyk znovu rozvine a~získá si své místo ve společnosti.
%dokončit