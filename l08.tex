\chapter{České národní obrození
}
\section{Vznik a~původ
}
České národní obrození je vyvoláno novým myšlenkovým proudem,
\emph{osvícenstvím}. To se k~nám rozšířilo z~Francie, kde ho prosazovali
takzvaní encyklopedisté, autoři naučného slovníku Encyklopedie aneb
Racionální slovník věd, umění a~řemesel, Denis Diderot a~D'Alambert, ale
i~Voltaire a~Rousseau.

\emph{Osvícenství se vyznačuje snahou o~postupnou likvidaci feudálního systému a~reformu společnosti.} Tohoto cíle chce dosáhnout pomocí rozumu
a~vědy a~překonat tak náboženskou ideologii a~absolutismus. Osvícenské
ideje zasáhly i~panovníky. Reformy prováděla \emph{Marie Terezie}, která
zavedla povinnou školní docházku, osamostatnila školství od církve
a~umožnila i~prostým dětem získávat vzdělání. Její syn \emph{Josef II.}
prosadil na konci 18. století svými patenty zrušení nevolnictví
v~Čechách, zavedl svobodu náboženského vyznání a~zrušil cenzuru tisku.

Obrozenecká literatura se vyznačuje silným protiněmeckým zaměřením
a~vlasteneckým cítěním. Klade velký důraz na racionální vědu, proto v~ní
vědecká literatura zaujímá významný podíl.

\section{Obrozenecká vědecká literatura
}
Obrozenecké vědecké hnutí se projevuje v~osvícenských vědeckých
společnostech. V~Praze byla roku 1774 založena \emph{Královská česká společnost nauk}. 
Její členové se zabývali hlavně společenskovědními problémy, zejména
jazykovědnými a~historickými. Jejich bádání mělo silný vliv na rozvoj
národní literatury.

\subsection*{Historie
}
Kritické dějepisectví se snažilo rozebírat historické prameny 
a~nezaujatě jejich obsah hodnotit. Řádový kněz \emph{Gelacius Dobner}
se~zabýval rozborem Hájkovy kroniky, kterou sám vydal, a~odhalil její
nevěrohodnost. Své spisy však stále píše latinsky.

Oproti tomu \emph{František Martin Pelcl}, první profesor českého jazyka 
na~pražské universitě, píše ke~konci svého života již česky. Stejně jako
Dobner se i~on zasloužil o~historickou vědu a~svou \emph{Novou kronikou českou}
chce nahradit kroniku Hájkovu pravdivějším líčením minulosti.

\subsection*{Jazykověda
}
Zakladatelem \pojem{slavistiky}, neboli vědy studující jazyk
a~literaturu slovanských národů, byl Dobnerův žák \jmeno{Josef Dobrovský}.
Narodil se v~Ďarmotech v~Uhrách, studoval na gymnáziu v~Klatovech
a~Německém (Havlíčkově) Brodě a~na pražské filosofické a~teologické
fakultě. Stal se načas ředitelem kněžského semináře, pak pracuje jako
vychovatel v~rodině hraběte Nostice.

Také on kriticky zkoumá staré české literární památky a~své poznatky
shrnuje do \dilo{Dějin české řeči a~literatury.} Sám však o~budoucnosti
českého jazyka silně pochybuje, všechna svá díla píše buď německy, nebo
latinsky. Normu pro českou gramatiku zpracoval v~\dilo{Podrobné mluvnici české.}
Kniha měla zásadní význam pro ustálení spisovné češtiny. Aby
vyvrátil pochybnosti o~méněcennosti českého jazyka v~porovnání
s~ostatními, vytváří dvoudílný \dilo{Německo - český slovník.} Za jeho
ústřední slavistické dílo můžeme považovat \dilo{Základy jazyka staroslověnského,} 
přestože v~ní staroslověnštinu mylně považuje za~prapůvodní slovanský
jazyk. (Ve~skutečnost však jde jen o~makedonský dialekt z~okolí Soluně.)
Do básnické teorie zasáhl svým výkladem
\dilo{O~prozódii české}\footnote{prozódie je nauka o~stavbě verše,}
ve~kterém určuje jako základ novodobé poezie \pojem{sylabotónický systém}. 
Ten je kombinací sylabického verše, založeného na počtu slabik
a~verše tónického, vycházejícího z~přízvuku. Kromě toho existuje ještě
časoměrný verš, který pravidelně střídá krátké a~dlouhé slabiky.

K~vědeckému okruhu obrozenců můžeme počítat i~tzv. obránce jazyka.
Nejradikálnější z~nich byl \jmeno{Karel Ignác Thám}, bratr známého
divadelníka, autor \dilo{Obrany jazyka českého proti zlobivým jeho utrhačům.}
Stejně jako ostatní obhajuje právo používání češtiny v~úřadech, církvi
i~lidové literatuře. Tyto snahy však měly i~odvrácenou stranu v~podobě
různých oprávců, kteří v~úmyslu češtinu zdokonalit vymýšlejí nejrůznější
vykonstruované neologismy jako náhradu za přejatá slova. V~této oblasti
neblaze proslul \jmeno{Jan Václav Pohl}.

\section{Beletrie počátku obrození
}
Obrozenci využívají beletrii hlavně jako prostředek lidové osvěty.
Centrem vydavatelských snah byla \emph{Česká expedice}, knihkupectví 
a~nakladatelství \jmena{Václava Matěje Krameria}{Václav Matěj Kramerius}. 
Kromě týdeníku \dilo{Pražské noviny} vydává široké spektrum
lidové četby -- \emph{Staročeské cestopisy}, \emph{Trójanskou kroniku},
\emph{Ezopovy bajky} a~také první český překlad \emph{Robinsona Crusoe}. 

\section{Novočeská poezie
}
Snaha o~povznesení české literatury na světovou úroveň se projevila
nejdříve v~poezii. \jmeno{Václav Thám} vydává roku 1785 sborník
\dilo{Básně v~řeči vázané,} do kterého zařazuje ukázky starší české
poezie, překlady cizích autorů, i~novodobé pokusy současníků.

O~deset let později vydává svůj almanach také \jmeno{Antonín Jaroslav
Puchmajer} pod jménem \dilo{Sebrání básní a~zpěvů.} Tentokrát do něj
kromě sebe nechává více přispívat současné autory, jako byli
\emph{bratři Nejedlí}, \emph{Šebestian Hněvkovský}, nebo \emph{Josef Jungmann.}
Almanach doplňuje překlady, bajkami a~teoretickými výklady. Narozdíl od
almanachu Thámova více využívá pravidel prozódie.

\section{Druhá obrozenecká generace
}
I~přes ztížené podmínky způsobené vládou kancléře \emph{Metternicha}
obrozenecké hnutí dále sílí, hlavně díky přílivu další generace české
inteligence.

\subsection*{Josef Jungmann}
Obdobný význam jako Dobrovský pro začátky obrození má pro toto období
\jmeno{Josef Jungmann}. Narodil se v~Hudlicích u~Berouna v~rodině chudého
řemeslníka, studuje v~Praze filosofii, pak nastupuje na místo
gymnazijního profesora v~Litoměřicích. Do národního obrození zasahuje
svými statěmi, např. \dilo{Rozmlouvání o~jazyku českém.} Vyjadřuje
v~nich odmítavý postoj k~poněmčování škol a~úřadů a~také své nové pojetí
národa a~vlastenectví. Nepovažuje totiž za Čechy ty, kteří sice u~nás
žijí a~vlastenecky cítí, mluví však cizím jazykem. Tím odmítá tzv.
\pojem{zemské vlastenectví}.

Podobně jako Dobrovský i~Jungmann pracuje na rozsáhlém slovníku,
tentokrát ale pro překlad opačným směrem. Jeho pětidílný \dilo{Slovník
česko -- německý,} se stal základem dalšího rozvoje české literatury
a~svým významem předčil úroveň polského \emph{Lindova slovníku}, který mu
byl zpočátku vzorem.

Nezapomíná ani na výuku češtiny na školách. Sestavuje proto učebnici
literatury \dilo{Slovesnost,} kterou doplňuje ukázkami od Jana Husa až
do současnosti.

Stejně jako Dobrovský se i~Jungmann pokusil o~sepsání našich literárních
dějin. Narozdíl od něj se však v~\dilo{Historii literatury české}
nezaměřuje na starší tvorbu, naopak, považuje obrození za začátek nové
epochy.

\subsection*{Rukopisné padělky
}
Růst zájmu o~vlastenecké hnutí ve vzdělanějších vrstvách, jako jsou
měšťané nebo šlechta, vyvolal potřebu náročnější
\emph{krásné}\footnote{krásná literatura neboli beletrie} literatury.
Obdiv k~dávným hrdinským zpěvům vedl k~nálezu \dilo{Rukopisů
královédvorského a~zelenohorského.} První nalezl roku 1817 univerzitní
profesor \jmeno{Václav Hanka} ve věži kostela ve Dvoře Králové. Byly to
listy pergamenu, údajně se zlomky starých básní ze 13. století. O~rok
později byl do Národního muzea doručen druhý rukopis, dokonce prý z~9.
století, který líčil \emph{Libušin soud}. Václav Hanka pravděpodobně
na~těchto podvrzích spolupracoval s~\jmena{Josefem Lindou}{Josef Linda}.
Účelem rukopisů bylo patrně dokázat vysokou kulturní úroveň našich zemí
již v~dávných dobách, a~podpořit tak české vlastenectví.

Pochybnosti o~jejich pravosti projevil \emph{Josef Dobrovský}, tuto domněnku
však definitivně potvrdil až po tzv. rukopisných bojích \emph{T. G. Masaryk}.

\subsection*{František Palacký}
Vedle Dobrovského a~Jungmanna je dalším významným vědcem
\jmeno{František Palacký}, rodák z~Valašska. Po studiích v~Trenčíně
a~Bratislavě se živí jako vychovatel v~šlechtických rodinách. Po
příchodu do Prahy se věnuje práci historika. Na doporučení Dobrovského
se zabývá genealogií\footnote{sestavování rodokmenů} Štenberků a~jiných
šlechtických rodů. Poté, co vydá staročeské kroniky pod názvem
\dilo{Staří letopisové čeští,} pokračuje v~soustavné práci na českých
dějinách. Prozkoumal řadu evropských a~domácích archívů, a~studoval zde
listinné materiály.

Své poznatky uložil do svého stěžejního díla 
\dilo{Dějiny národa českého v~Čechách a~na Moravě.} V~pěti dílech
popisuje českou historii až do roku 1526, tedy nástupů Habsburků na
český trůn. Dál nepokračoval proto, že by události již nemohl popisovat
objektivně. Patrně pod vlivem \emph{Rukopisů} zde dějiny mírně
idealizuje, líčí je jako boj slovanských demokratických principů proti
germánskému feudalismu. Za nejslavnější období tedy považuje husitství.

\subsection*{František Ladislav Čelakovský}
Takzvanou ohlasovou poezií, napodobující a~čerpající z~lidové
slovesnosti, se zabýval \jmeno{František Ladislav Čelakovský}. Pocházel
ze Strakonic, ze školy v~Českých Budějovicích byl vyloučen pro četbu
Husovy \emph{Postily}. Filosofii pak dál studuje v~Linci a~Praze. Když
se zde uchytí jako redaktor, je opět zbaven místa, tentokrát kvůli
článku proti ruskému carovi. Jinak ruský národ považoval za oporu
ostatních slovanských národů, proto se nejprve soustřeďuje na ruskou
lidovou tvorbu. Přestože látku pro \dilo{Ohlas písní ruských} čerpal
z~ruských bylin,\footnote{ruská lidová epická píseň} některé příběhy
doplňuje dle vlastních představ.

Deset let připravuje \dilo{Ohlas písní českých,} kde narozdíl od ruských
bohatýrských bylin převládají spíše lyrické nebo žertovné skladby,
charakterizující českého sedláka. Pokusil se též o~baladu -- 
\dilo{Toman a~lesní panna.} Rozdíl mezi ruskou a~českou poezií sám
přirovnává k~rozdílu mezi typickými krajinami obou zemí. Rozlehlé ruské
hvozdy a~mohutné řeky oproti českým drobným lesíkům, říčkám a~potokům.

\section{Obrozenecké divadlo
}
České divadlo za obrození se obtížně prosazovalo vedle divadla německého
a~italské opery. Obrozenci se mu však soustavně věnovali pro jeho
bezprostřední vliv na diváka. V~pražském divadle \emph{V~Kotcích} uvedli
první českou hru roku 1771. Byl to překlad německé \emph{Kr\" ugerovy}
jednoaktové veselohry \dilo{Kníže Honzik.} Později se hrálo česky
v~\emph{Nosticově} -- Stavovském divadle, za vydatné podpory císaře
Josefa II..

\subsection*{Počátky obrození, divadlo Bouda
}
Prvním čistě českou scénou se stalo \emph{Vlastenecké divadlo}, známé
spíše jako \emph{Bouda}. Sídlilo v~dřevěné budově na Václavském náměstí.
Kolem \jmena{Václava Tháma}{Václav Thám} se zde shromáždilo několik
autorů, kteří psali historické hry z~českých dějin, dále oblíbené frašky
a~rytířské hry. Hojně překládali z~němčiny i~ze světových klasiků
(Shakespeare). Sám Václav Thám byl tvůrcem vlasteneckých dramat
zobrazujících události raných českých dějin. Oblíbeny lidovým
obecenstvem byly hra \dilo{Břetislav a~Jitka,} nebo \dilo{Vlasta
a~Šárka,} pojednávající o~dívčí válce.

\subsection*{1. polovina 19. století
}
Ve dvacátých letech působí ve Stavovském divadle \jmeno{Jan Nepomuk
Štěpánek}. Pod jeho vedením se hrají hlavně rytířské a~strašidelné
příběhy, vesměs přeložené z~němčiny. Jeho dobu překonala jen veselohra
\dilo{Čech a~Němec,} kde se kromě jazykové komiky vyskytuje myšlenka
tolerance mezi národy. 

Dalším a~dodnes hraným obrozeneckým dramatikem je \jmeno{Václav Kliment
Klicpera}, rodák z~Chlumce nad Cidlinou. Byl profesorem na gymnáziu
v~Hradci Králové a~posléze ředitelem pražského akademického gymnázia.
Na~obou místech rád hrával divadlo se studenty a~ochotníky. Začínal
vlasteneckými hrami, jako \dilo{Blaník,} pak postupně přechází
k~veselohrám kritizujícím lidské nedostatky, lakotu, plané vlastenectví,
byrokracii a~hloupost, jako je tomu ve \dilo{Veselohře na mostě} nebo
\dilo{Potopě světa.} Klicperovy komedie nepostrádají vtip a~optimismus,
často využívají záměny osob, například v~\dilo{Rohovínovi Čtverrohém}
a~\dilo{Hadriánovi z~Římsů,} což je zároveň parodie na rytířské hry.
\dilo{Ptáčník} líčí osud lékárníka, kterého se snaží přátelé přesvědčit,
aby zanechal své záliby v~lovení ptactva a~věnoval se raději svému
povolání. Ve svém prozaickém díle se Klicpera věnuje historickým
námětům, hrdinou povídky \dilo{Točník} je král \emph{Václav IV.}.

\subsection*{Josef Kajetán Tyl}
Klicperovým žákem jako student i~jako dramatik byl \jmeno{Josef Kajetán
Tyl}. Studoval tedy také v~Hradci a~v~Praze. Svou uměleckou dráhu začal
historickými povídkami z~české minulosti. Ty však se skutečností mají
málo společného, hlavní důraz Tyl klade na napínavý děj. Příkladem mohou
být povídky \dilo{Rozina Ruthardová} nebo \dilo{Dekret Kutnohorský.}

Hrou z~prostředí ševcovské cechovní slavnosti, \dilo{Fidlovačka, aneb
žádný hněv a~žádná rvačka,} chtěl získat i~měšťany a~řemeslníky pro
obrozenecké hnutí. Památnou je zejména tím, že v~ní poprvé zazněla píseň
\jmena{Františka Škroupa}{František Škroup} \dilo{Kde domov můj,} která
znárodněla a~stala se naší hymnou. Mezi jeho sociální hry patří
\dilo{Paličova dcera,} vyprávějící příběh Rozárky, jejíž otec je odsouzen
za žhářství, jehož se dopustil ze zoufalství a~msty. Rozárka, která se
pak stará o~své osiřelé sourozence, se nakonec dočká odměny od bohaté
tety.

Vrcholem Tylovy tvorby jsou jeho pohádkové hry, především
\dilo{Strakonický dudák,} příběh jihočeského dudáka \emph{Švandy}, který
podlehne vyprávění o~bohaté cizině a~vydá se do světa vydělat
\uv{tisíce} českou muzikou. Tam ale propadne kouzlu slávy a~nebýt
\emph{Dorotky}, jeho milé, a~\emph{Rosavy}, polednice a~jeho matky,
která však ztělesňuje českou přírodu, byl by na svou vlast ani
nevzpomněl. I~přes svůj pohádkový námět je Tylova hra stále aktuální
a~uváděná na současných scénách.

Pohádkový námět, který se navíc prolíná s~revolučními idejemi roku 1848,
se objevuje také v~další báchorce o~svéhlavé mlynářce --
\dilo{Tvrdohlavá žena.}

Revoluční rok 1848 Tyla inspiroval k~vytvoření historických her
\dilo{Kutnohorští havíři,} \dilo{Jan Hus,} \dilo{Jan Žižka z~Trocnova}
a~\dilo{Bitva u~Sudoměře.} Jejich hlavním smyslem ale není zachytit
pravdivě minulost, spíše se snaží vyjádřit aktuální myšlenky, současné
požadavky a~cíle, bojuje tímto způsobem za národní demokratická práva.

\section{Národní obrození na Slovensku
}
\subsection*{Pavel Josef Šafařík}
Palackého přítel a~spolupracovník \jmeno{Pavel Josef Šafařík} se věnoval
studiu slovanských národů. Považoval je za jednotné, jednotlivé jazyky
jen za dialekty. Jejich nejstarší dějiny (do roku 1000) zachytil ve
\dilo{Slovanských starožitnostech,} kde zároveň dokázal, že i~Slované
patří k~evropské kultuře.

\subsection*{Jan Kollár}
Tvorbu \jmena{Jana Kollára}{Jan Kollár} nejvíce ovlivnily roky studia
v~saské Jeně. Zažil zde nejenom o~poznání větší svobodu pokrokových
myšlenek než v~Rakousko-Uhersku, ale potkává zde svojí životní lásku,
kterou nazývá \emph{Mína}. První sbírka sonetů jí věnovaných vyšla pod
názvem \dilo{Básně.} Po dvou letech ale musí \emph{Mínu} opustit a~ta se
mu postupně stává víc a~víc představou a~symbolem vlasti, podobně jako
Dantova \emph{Beatrice}. K~jeho sonetům přibývají další, a~v~další
sbírce \dilo{Slávy dcera} již Mínu opěvuje jako dceru bohyně
\emph{Slávy}. Skladbu rozdělil do částí nazvaných po řekách, podél
kterých žili nebo žijí Slované - \emph{Sála}, \emph{Labe}
a~\emph{Dunaj}. Další části, \emph{Lethé} a~\emph{Acheron} věnuje
přátelům a~nepřátelům slovanských národů.