\chapter[Moderní umělecké směry mezi světovými válkami]{Moderní umělecké směry a~poezie mezi dvěma světovými válkami} 
V~reakci na 1.~světovou válku a~situaci po ní vzniká v~tomto období
mnoho uměleckých směrů.

%\begin{table}[btp] 
\begin{center}
\textsc{\Large
Moderní umělecké směry\\
vznikající mezi válkami
}
\end{center}
\vspace{3mm}
\begin{description}
\hrule
\vspace{-1mm}

\item[fauvismus] výtvarný směr, vrcholí na počátku 20.~století, vznik ve Francii,
ostrá barevnost, barva v~rozpou se skutečností, jde o~vyjádření tvůrce
\item[futurismus] umělcký směr ve všech druzích umění, vniká na počátku
20.~století, reakce na dosavadní umění a~společnost, obdiv k~budoucnosti,
technice, civilizaci, umění vyjadřuje pohyb a~dynamiku, v~malířství
se obraz rozkládá na pohybové fáze, v~literatuře se experimentuje,
odmítá intrpunkci, velká a~malá písmena, dosavadní větnou stavbu
\item[kubismus] výtvarný směr počátku 20.~stoltí, převádění tvarů 
na geometrické tvary, základním tvarem krychle, snaha zobazit předmět
ze všech pohledů, cílem je proniknout pod povrch věcí, literatura rozložuje
slohové reality na slova, slova se opět znovu skládají
\item[dadaismus] z~francouzštiny, dětský výraz, \emph{Tristan Tzara},
vznikl jako reakce na 1.~světovou válku, výrazný odpor proti válce
kultuře, válka i~svět je absurdní, proto musí být i~umění absurdní,
poezie ruší pravidla skladby, princip náhod, náhodné skládání slov
\item[surrealismus] vznik ve 20.~letech ve Francii, nadrealismus,
vychází z~učení \emph{Z.~Freuda}, automatický text, zachycování myšlenek
bez vědomé kontroly, psaní bez předem připraveného textu, nemůže vyloučit
aktuální obsah, \emph{V.~Nezval, K.~Biebl}
\item[poetismus] český umělecký směr, básníci přesvědčeni, že literatura je
vzdálená, skutečný rozvoj v~druhé polovině 20.~století -- končí éra proletářské
poezie, představa, že umění má především bavit autora i~adresáta,
principem je hra (se slovy, představami, zvukovou stránkou, grafickou podobou),
obrovské uvolnění fantazie, vyloučení aktuálního a~sociálního obsahu,
témata, náměty z~exotického prostředí, cirkus, variet\`e, motiv
cesty, postavy námořníků, klaunů, černochů, \emph{J.~Sei\-fert, \hbox{K.~Teige}, V.~Nezval, \hbox{K.~Biebl}}
\hrule 
\end{description}
%\caption{Moderní umělecké směry počátku 20.~století} \label{tab:modernismerydva}
%\end{table}

\section{Světová poezie 1.~poloviny 20.~století}
\subsection*{Francie}
\subsection*{Guillaume Apollinaire}
První použil pojmu \emph{surrealismus} francouzský básník \jmeno{Guillaume
Apollinaire}. Apollinaire se narodil v~Římě, dětství ale prožil v~Monaku
a~nakonec se usadil v~Paříži. Počátky jeho tvorby však patří ještě
symbolismu.

Píše obrazové básně \dilo{Kaligramy,} svojí úpravou připomínají obraz.
Básně založené na principu asociace,\footnote{podvědomé vybavení
představ} \dilo{Alkoholy,} obsahují \dilo{Pásmo,} které dalo vzniknout
metodě pásma -- volného spojování pocitů, dojmů, nálad, vzpomínek\dots{}

Představitel hybridního směru kubofuturismu, který spojuje prvky kubismu
a~futurismu, je \jmeno{Guillaume Apollinaire}. Ruší vztahy mezi jednotlivými
větami, zobrazuje skutečnost ze všech pohledů.

Jednou z~jeho prvních sbírek je \dilo{Práchnivějící kouzelník,}
nejslavnějším dílem se stala sbírka \dilo{Alkoholy,} kde uplatňuje svůj
osobitý pohled na svět, vytvořený řetězením nejrůznějších dojmů,
spojením minulosti se současností. Mnoho dalších autorů ovlivnil cyklus
\dilo{Pásmo.} Návštěva Prahy ho inspirovala k~napsání \dilo{Pražského
chodce.} Vytvořil také novou básnickou formu -- \dilo{Kaligramy,} kde
vytvářel z~veršů nejrůznější obrazce. Surrealismus uplatnil i~v~dramatu
\dilo{Prsy Tireziovy.}

\subsection*{Ruská futuristická tvorba\\Vladimir Vladimirovič Majakovskij}
Ruským představitelem futurismu je \jmeno{Vladimir Majakovskij}.
Futurismus sloučil se socialistickými revolučními idejemi -- rozdíl od
pojetí italského vysvětlil ve stati \dilo{Futurismus dnes.}
Za~1.~světové války psal lyrické poemy, jako \dilo{Oblak v~kalhotách.}

Pro revoluci a~SSSR agituje v~dílech \dilo{Levý pochod} a~\dilo{Správná
věc.} Satirou na hymny je ironický cyklus \dilo{Hymny,} obsahuje hymnu
na zdraví, soudy, učence\dots{}

Spolupracoval s~režisérem \emph{Ejzenštejnem}, založil s~ním moskevské
divadlo MCHAT (píše satirické hry \dilo{Štěnice} a~\dilo{Ledová sprcha})
a~vytvořil pro něj scénář k~filmu o~vzpouře námořníků během prvního dne
VŘSR,\footnote{výťezné říjnové socialistické revoluce} \dilo{Křižník Potěmkin.}

\jmeno{Vladimír Majakovskij} používal zvláštní grafickou úpravu,
používal motiv velkoměsta, války, sociální nespravedlnosti, ale
i~motiv budoucnosti

\section{Surrealismus}
Surrealisté se snaží spontánně zachytit lidské podvědomí a~bezprostředně
zaznamenat představy a~sny v~podobě automatického textu. Odmítají logiku
v~umění, a~tím chtějí plně osvobodit tvůrčí fantazii.

\subsection*{André Breton}
Od dadaismu (sbírka \dilo{Zastavárna}) přešel k~surrealismu \jmeno{André
Breton}. Svoje opovržení konvenční stavbou poezie vyjádřil ve sbírce
\dilo{Bída poezie.} Surrealistická jsou jeho další díla \dilo{Spojité
nádoby} a~\dilo{Vzduch vody.} Program surrealistického hnutí vyložil ve
dvou \dilo{Manifestech surrealismu} (1.--~1924, 2.--~1930). V~druhém
z~nich se projevuje příklon surrealistů k~levici, pod dojmem španělské
a~ruské revoluce.

\subsection*{Paul Eluard}
Stejně jako André Breton i~jeho přítel \jmeno{Paul Eluard} začínal
dadaismem a~přešel k~surrealismu sbírkami \dilo{Bezprostřední život,}
\dilo{Kritika poezie,} \dots{} Kromě toho věnuje svoji poezii
společenské a~protimilitaristické kritice, zejména za španělské občanské
války (slavná je jeho báseň \dilo{Svoboda}). Protiválečná
a~protifašistická jsou jeho díla \dilo{Básně pro mír,}
\dilo{Veřejná růže} nebo \dilo{Vítězství Guerniky,} ke které ho
podnítilo bombardování španělské vesničky.

2.~světové války se účastní aktivně jako partyzán i~jako autor sbírek
\dilo{Otevřená kniha I. a~II.}

\section{Česká poezie 1.~poloviny 20.~století}
Ovlivněna společností a~moderními uměleckými směy, 1.~světovou válkou,
zklamáním z~politiky, událostmi v~Rusku (únorová a~říjnová revoluce,
poválečná hospodářská krize). V~polovině 20.~století nastává období
stabilizace -- změna české poezie. Vzniká poetismus, roku 1920 vzniká
\jmeno{Devětsil} -- spolek avantgardních umělců (\emph{Teige, Seifert,
Wolker, Nezval, Biebl, Vančura, Burian, V+W, Ježek\dots{}}~)

Ve~30.~letech nastupuje fašismus, nastává hospodářská krize, to vede ke
vzniku nových uměleckých směrů. \emph{Angažovaná tvorba} navazuje na
proletářskou poezii, \emph{surrealismus} je významější
a~\emph{spirituální a~meditativní poezie} se zabývá otázkami lidské
existence.

Koncem 30.~let dělení mizí, poezie reaguje na fašismus a~na ohrožení
republiky. V~období okupace je poezie oblíbena, píše se česky.
Sdružují se témata, převažuje intimní, historické (bez husitství) téma,
jako symbol národa se oběvuje Praha. Aktuální témata se schovávají
a~vycházejí až po válce.

\section{Proletářská poezie}
Proletářská poezie vzniká ve 20.~letech. Oběvuje se reakce na
1.~světovou válku. Hlásí se velké množství autorů (\emph{Wolker, Hora,
Hořejší, Neumann}) 
\emph{Jiří Wolker} napsal stať \dilo{Proletářská poezie,} kde shrnuje základní
znaky: revoluční optimistické vyznění, svět musí být změněn ve pospěch
utlačovaných, autor se dívá na pohled dělnické třídy, nemluví sám za sebe,
vyjadřuje názory dělníků.

\subsection*{Stanislav Kostka Neumann}
\label{chap:SKNagit}
Krom poezie buřičské a~anarchistiké (viz. str. \pageref{chap:SKNbur})
psal \jmeno{Stanislav Kostka Neumann} i~agitační poezii. Ta má burcovat k~boji proti kapitalismu,
vyvolává nenávist.

Po~1.~světové válce, kterou prožil částečně na frontě (protiválečná
próza \dilo{Válčení civilistovo}), se angažuje v~komunistické straně. Orientuje
se na proletářskou poezii svou sbírkou \dilo{Rudé zpěvy,} kde nekriticky
vkládá do SSSR naděje na uskutečnění světové revoluce.

V~\dilo{Sonátě horizontálního života} varuje před fašismem, ukazuje
obraz španělské války. V~próze \dilo{Anti--Gide} se zastává Stalina.

Za okupace reagoval na Mnichovský diktát a~bránil demokracii.
Protestoval v~dílech \dilo{Bezedný rok} a~\dilo{Zamořená léta.}

\subsection*{Jiří Wolker}
\jmeno{Jiří Walker} je nejtypičtějším představitelem proletářské poezie.
Napsal dvě básnické sbírky -- \dilo{Host do domu} a~\dilo{Těžká hodina.}

\uv{\emph{Host do domu}} není proletářská sbírka, jde spíše o~básně ovlivněné
vitalismem -- oslava života, radost z~jeho krásy, pokorná láska k~lidem,
maličkostem přírodě. V~básni \dilo{Pokora} je on tím nejmenším článkem
světa. Též napsal \dilo{Poštovní schránku.} Oběvují se personifikace,
motiv očí a~srdce -- oči vidí svět, srdce prožívá lásku, která vše řeší.
Oběvuje se motiv těžkosti života, samota a~smutek -- \dilo{Svatodušní
svátky.} Oběvují se sociální motivy, vzpomínky na dětství, přátele, sny
o~lepším světě v~samostatné skladbě \dilo{Svatý kopeček.}

\uv{\emph{Těžká hodina}} je sbírka proletářské poezie. 
Důležitá je úvodní a~poslední báseň. Úvodní \dilo{Těžká hodina}
zobrazuje protiklad dětských snů a~toha jejich naplnění v~dospělosti.
Obává se o~jejich naplnění. V~básni \dilo{Moře} se autor stotožňuje
s~básníky, na uskutečnění snů už není sám. Moře představuje lidi, kteří
ho obklopují.  Walkerovy sociální balady mají pochmurný děj a~tragický
konec -- \dilo{Balada o~očích topičových,} \dilo{Balada o~nenarozeném
dítěti} nebo \dilo{Balada o~snu,} ten se dá zabít jenom jeho realizací.

\subsection*{Jindřich Hořejší}
\jmeno{Jindřich Hořejší} překládal z~francouzštiny, psal poezii.
Jeho poezie je ovlivněna 1.~světovou válkou. Oběvuje se odpor proti
válce, motiv velkoměsta a~lásky. V~dalších sbírkách se oběvuje milostná
lyrika. Z~jeho tvorby můžeme uvést díla \dilo{Hudba na náměstí,}
\dilo{Korálový náhrdelník} a~\dilo{Den a~noc.}

\subsection*{Josef Hora}
\jmeno{Josef Hora} překládal z~ruštiny, přeložil \emph{Evžena Oněgina}.
Psal poezii -- vitalismus a~impresionismus. Motiv revoluce, touha po
harmonii a~rovnováze se oběvuje v~díle \dilo{Strom v~květu.}
Z~proletářské poezie -- \dilo{Pracující den,} \dilo{Srdce a~vřava světa,}
\dilo{Bouřlivé jaro.}
Přiklání se k~intimní a~přírodní lyrice, používá motiv ubíhajícího času
-- \dilo{Itálie,} \dilo{Tvůj hlas.}
Ve 30.~letech reaguje na španělskou válku a~nástup fašismu. Obrací se
k~domovu a~tradicím. Vztach k~rodné zemi vyjádřil v~\dilo{Máchovské
variaci,} kde dává hold \emph{Máchovi.}
V~\dilo{Domově} reaguje na bezprostřední ohrožení republiky.
Jan Kubelík ho inspiroval k~básni \dilo{Jan Houslista.}

\subsection*{Jaroslav Seifert}
\dilo{Jaroslav Seifert} dostal Nobelovu cenu za literaturu.
V~jeho proletářské poezii se oběvuje soucit s~chudými, chce svět bez
bídy, nenávisti, ideálně zbyde pouze štěstí a~láska. (\dilo{Město
v~slzách,} \dilo{Samá láska})

Jeho poetistická díla jsou anekdotická, s~bizardními nápady.
(\dilo{Na vlnách TSF}) V~dalších sbírkách se oběvuje motiv smutku a~rozpornosti života.
(\dilo{Slavík zpívá špatně,} \dilo{Poštovní holub})

Později, ve 20.---30.~letech se nehlásí už k~žádnému směru, reaguje na
fašismus a~ohrožení republiky. Vzpomíná na dětství a~na domov.
(\dilo{Jablko z~klína,} \dilo{Jaro s~bohem})
Reakcí na Mnichov a~okupaci s~hořkostí a~hněvem je \dilo{Zhasněte světla} a~\dilo{Píseň o~rodné zemi.}
Motiv domova a~Prahy se oběvuje v~\dilo{Světlem oděné} a~\dilo{Kamenném mostu.}
Návratem k~tradicím je jeho \dilo{Vějíř Boženy Němcové.}

\subsection*{Konstantin Biebl}
\jmeno{Konstantin Biebl} píše proletářskou poezii podobnou Walkerově,
oběvuje se melancholie a~smutek. (\dilo{Věrný hlas,} \dilo{Zloděj
z~Bagdádu}) Napsal lyricko--poetický cestopis \dilo{S~lodí, jež dováží čaj
a~kávu.} Básnická skladba \dilo{Nový Ikaros} -- hlavní hrdina je básník,
prostřednictvím motivu Ikarova letu se setkáváme s~jeho životem i~soudobým světem.
Surrealistické \dilo{Zrcadlo} je reakcí na mnichovský diktát a~okupaci,
báseň \dilo{Bez obav} je reakcí na válku.

\subsection*{Vítězslav Nezval}
\jmeno{Vítězslav Nezval} prosazoval moderní umělecké směry. Jeho poezie
je ovlivněna fantazií a~kontrastem se světovou. Používá překvapivé
metafory, přirivnání a~lehké optimistické verše.  Napsal básnické sbírku
\dilo{Most,} \dilo{Pantomima,} které jsou souhrnem různých děl
a~experimentů (jazykových, zvukových a~grafických). \dilo{Papoušek na
motocyklu} je poetistická báseň, která má působit jako lehká hra, práce
nesmí být vidět. Sbírka \dilo{Abeceda} obsahuje básňě inspirované
písmeny abecedy. (\uv{A} Nazváno buď chatrčí\dots{}) \dilo{Podivuhodný
kouzelník} je psán metodou pásma. Básník (přirovnávaný ke kouzelníkovi)
prochází různými obdobími. Pásmo je oslavou tvůrčí práce.

Ke konci 20.~let se jeho tvorba začíná měnit -- objevuje se motiv
smutku. Sbírka \dilo{Básně noci} obsahuje několik básní a~pásem --
\dilo{Akrobat,} \dilo{Edison.}

\emph{Edison} je psán metodou pásma, inspirací mu byly osobní problémy
T.~A.~Edisona, oběvuje se zde opakování, myšlenkou je oslava tvůrčí
práce, kdy člověk zapomene na smutek a~stesk\dots{}

Ve 30.~letech píše surrealistická díla -- \dilo{Žena v~množném čísle,}
\dilo{Praha s~prsty deště,} \dilo{Absolutní hrobař.} Zároveň píše
anonymně, sbírky jsou melodické s~vázaným veršem, reagují na sociální
problémy. Ve sbírce \dilo{52~hořkých balad věčného studenta Roberta
Davida} projevuje odpor k~měšťáctví. \dilo{Sbohem a~šáteček} je básnický
cestopis po Itálii a~Francii. Projevuje se v~něm stesk po domově, ale
zároveň okouzlení přírodou a~sluncem.

Reakce na dobu a~společenské události vyjadřuje v~\emph{Matce naděje.}
\dilo{Matka naděje} je spojení jeho nemocné matky a~osudu národa.
%\dilo{Historický obaz}

Za okupace píše alegorické verše inspirované jeho rodným krajem a~láskou
k~domovu. (\dilo{Pět minut za městem}) Píše také divadelní hry --
\dilo{Milenci z~kiosku,} překlad \emph{Pr\`evostova} románu \emph{Manon
Lescaute.} 

\section{Poezie 30.~let}
Objevuje se \emph{spiritualismus} -- duchovní poezie. Jedinými jistotami
je život a~smrt. Autoři reagují na mnichovský diktát, projevují lásku
k~vlasti a~varují před nebezpečím zneužití moci.
\subsection*{František Halas} 
Básníkem mezigeneace, tj. mezi proletářskou literaturou a~2.~světovou
válkou, je \jmeno{František Halas}. Je to rozporný básník typem a~tím,
co píše. Jeho poezije je nemelodické a~disharmonická, po formální
stránce ovlivněná poetismem -- metafory, pohrávání s~představami.
Liší se tématy -- pocit rozčarování, melancholie, slepoty a~smrti.
Dětství a~smrt jsou jediné dvě jistoty -- \dilo{Kohout plaší smrt.}

Ve 30.~letech se jeho tvorba mění a~píše angažovanou poezii. Ve sbírce
\dilo{Dokořán} ale převažují básně předcházejících let. Reakce na
Mnichov se oběvuje ve sbírce \dilo{Torzo naděje} (\emph{torzo} je
symbolem téměř beznaděje). Je psána melodickým a~rytmickým veršem.
Naději básník získává přimknutím k~domovu a~k~českým tradicím. Oběvuje
se motiv Prahy, jako symbol českého národa -- báseň \dilo{Praze.}
Opakují se tvrdé souhlásky, které navozují pocit nebezpečí, naděje je
malá, ale má se pro ni bojovat. Báseň \dilo{Zpěv úzkosti} pojednává
o~zradě Anglie a~Francie. Báseň \dilo{Mobilizace} reaguje na Mnichov,
protiklad nadšení a~zklamání z~Mnichova.

V~období okupace se obrací k~tradicím. \dilo{Naše paní Božena Němcová}
pojednává o~jejím osudu, spojuje ji s~osudem náoda, jazykem a~kulturou.
Spolu s~\emph{Holanem} napsal \dilo{Láska a~smrt.}

\subsection*{Vladimír Holan}
\jmeno{Vladimír Holan} je myšlenkově i~čtenářsky nejnáročnějším autorem.
Zabývá se otázkami smyslu života, bytím a~nebytím.

Pokouší se o~\pojem{absolutní báseň} -- řeč je nezávislá na významové hodnotě slova.
V~tomto odobí experimentu s~jazukem napsal například \dilo{Kamení, přicházíš.}

Ve 30.~letech píše angažovanou poezii, je ovlivněn událostmi ve
Španělsku a~Mnichově. Sbírka \dilo{Havraním brkem} -- v~básni
\dilo{Září~1938} odsuzuje mnichovskou zradu, věří v~budoucnost,
v~básni \dilo{Poslední testament} je vidět pocit smutku a~beznaděje.

\subsection*{Jan Zahradníček}
\jmeno{Jan Zahradníček} je katolický autor, píše poezii ovlivněnou
symbolismem a~dekadencí. V~jeho básních se oběvuje pocit osamělosti,
žal, nicota, bolest, utrpení a~odraz jeho života -- \dilo{Pokušení
smrti.} Poté se obrací na přírodní lyriku s~motivem rodného kraje
a~práce. \dilo{Jeřáby} je sbírka osobní lyriky.

Po Mnichově se u~něj oběvuje touha, aby se už utrpení lidstva
neopakovalo. Varuje před katastrofou, nakonec oslavuje život --
\dilo{Korouhve.} Za války opěvuje lásku k~národu a~projevuje odpor proti
utlačovatelům -- \dilo{Stará země,} \dilo{La Salleta,} \dilo{Čtyři
léta.} \dilo{Znamení noci} je reakcí na situaci po válce, varuje před
zneužitím moci.

\section{Poezie za okupace}
Za okupace se oběvuje téma jistoty domova a~dětství. Vzniká sborník
\dilo{Jarní Almanach Básnický,} vzniká \emph{Skupina~42}, která se snaží
zobrazit skutečný svět.

\subsection*{Jiří Orten}
\jmeno{Jiří Orten} píše meditativní, melodickou, intimní a~hluboce
prožitou poezii. Zpočátku byl důvěřivý, měl něžný vztah k~lidem
a~~věcem -- \dilo{Čítanka jara.}

Poté hledá smysl života a~prožívá pocit ohrožení, motivem je naděje --
\dilo{Cesta k~mrazu,} \dilo{Jeremiášův pláč,} \dilo{Ohnice.}

O~smrti a~nejcennějších hodnotách světa medituje v~žalozpěvech nad
ztraceným dětstvím, domovem, láskou ve sbírce \dilo{Elegie.}