\chapter[Evropská renesance a~humanismus]{Renesance a~humanismus v~evropské literatuře}
Renesance vzniká v~Itálii na přelomu \rok{12. a~13.} století a~je ideovým návratem k~antické kultuře.
\section{Itálie}
\subsection*{Dante Alighieri}
V~Itálii byl prvním představitelem básník \jmeno{Dante Alighieri}. Narodil se v~rodině drobného florentského %OK??
šlechtice, dosáhl vysokého vzdělání, hlavně v~oblasti antického a~soudobého italského básnictví. Za~to, že hájil
městské svobody a~vzpíral se papežovým nárokům, byl odsouzen do~vyhnanství a~tak strávil posledních 20~let života
ve~Veroňe a~v~Ravoňe. Zpočátku se zabývá dvorskou milostnou lyrikou a~rytířskými náměty a~zpracovává je do~\pojem{sonetů.}

Po odchodu z~Florencie napsal epos \dilo{Božská komedie} (inspiroval například Kollárovu 
Slávy dceru nebo Komenského Labyrint světa a~ráj srdce). 
Epos má tři části \pojem{Peklo, Očistec a~Ráj} každá má 33 zpěvů. Celkem tedy s~předzpěvem obsahuje zpěvů 100.
Dante v~něm v~doprovodu básníka \textit{Vergilia} prochází nejprve lesem, ohrožován třemi šelmami -- Smyslností, Pýchou a~Lakotou.
Při průchodu peklem dramaticky popisuje muka odsouzenců, mezi které umisťuje i~své politické odpůrce.
Před branou Ráje ho musí Vergilius opustit a~na~místo průvodce nastupuje \textit{Beatrice}.

\subsection*{Francesco Petrarca}
Francesco Petrarca byl stejně jako Dante florentský rodák a~také jeho vykázali do~vyhnanství, které prožil ve~Francii.
Jeho tvorbu ovlivňuje jeho celoživotní láska k~\textit{Lauře} a~touha po slávě, které chtěl dosáhnout svými latinskými
spisy inspirovanými antikou -- eposy, životopisy slavných Římanů a~traktáty. Nejvíce ho však proslavily \textit{sonety}, 
neboli znělky. \pojem{Sonet} se skládá ze 14 veršů rozdělených do~dvou strof po 4~verších a~dalších dvou 
po~3~verších. Neměnná byla i~myšlenková stavba. Kromě Petrarcy si sonet oblíbili Shakespeare, nebo u~nás Sova a~Hrubín.
Napsal \dilo{Sonety Lauře} -- 366 italských básní a~sonetů, všechny věnoval Lauře.

\subsection*{Giovanni Boccaccio}
Renesanční próza je zastoupena novelami, na jejichž vzniku má zásluhu \jmeno{Giovanni Boccaccio}. Jeho nejslavnějším dílem
je \dilo{Dekameron} -- skládá se ze 100 novel zasazených do~jedné rámcové novely. V~úvodním příběhu se dozvídáme o~morové
epidemii ve~Florencii, před kterou utíká společnost 7 šlechtiček a~3 mladých šlechticů na sídlo, kde si v~idylickém prostředí
10~večerů vyprávějí převážně milostné příběhy.


\section{Francie}
\subsection*{Fran\c cois Villon}
Také do~francouzské renesanční poezie proniká reálnější pohled na svět. Projevuje se například v~poezii 
\jmena{Fran\c coise Villona}{Fran\c cois Villon}. Studoval na pařížské universitě, pak ho okolnosti svedly k~nevázanému životu. 
Jednou v~hádce zabije kněze, prchá před spravedlností, toulá se a~krade. Několikrát je zatčen a~odsouzen k~smrti,
pak propuštěn díky vlivným přátelům. Pak po něm mizí stopy (1463).

Dobrodružný život se mu stal zdrojem jeho poezie. V~cyklu balad
\dilo{Velký testament} (Velká závěť) se zamýšlí  nad svým životem,
vyjadřuje lítost nad promarněným mládím a~omlouvá své hříchy chudobou 
a~osudem. \linebreak[3] Ve~starším \dilo{Malém testamentu} (Malé závěti) se formou
ironických a~satirických básní loučí se svými přáteli.

Své balady podává svrchovaným výjimečným způsobem, po něm nazvaným \pojem{Villonova balada}. Ta se skládá ze tří
strof o~7---12 verších a~čtvrté polovičního rozsahu. Každá z~nich je zakončena jednoveršovým \pojem{posláním},
které se stále opakuje a~vyjadřuje základní myšlenku básně.

Jeho pohnutý život zaujal mnoho pozdějších autorů, například Jarmila Loukotková -- Navzdory básník zpívá, nebo
Werich a~Voskovec -- Balada z~hadrů.

\subsection*{Fran\c cois Rabelais}
Největším prozaikem v~renesanční Francii byl všestranně vzdělaný kněz, lékař a~spisovatel \jmeno{Fran\c cois Rabelais}.
Proslul čtyřdílným satirickým románem \dilo{Gargantua a~Pantagruel} (pátý díl připojen neznámým autorem).
Prostřednictvím popisu historie královského rodu obrů paroduje rytířské romány. U~hrdinů jsou zveličeny tělesné 
rozměry, síla, ale i~žravost. Zároveň v~románu najdeme satirický pohled na běžné životní situace, na církev a~mnichy, 
jejichž život znal.

\section{Španělsko}
\subsection*{Miguel de Cervantes Saavedra}
Parodií na rytířské romány je i~rozsáhlý román špaňelského spisovatele \jmena{Miguela de Cervantese}{Miguel de Cervantes} 
\dilo{Důmyslný rytíř Don Quijote de La Mancha.} Vypráví příběh chudého zemana, který se, ovlivněn četbou rytířských knih, 
vydává jako rytíř za dobrodružstvím. Brání svým mečem dobro a~čest i~v~naprosto nesmyslných situacích, a~proto ho okolí
má spíše za legrační figurku. Jeho snaha, ač marná, přesto není směšná, odráží se v~ní touha po spravedlivějším světě
a~po svobodě -- jeden z~renesančních ideálů. V~postavě \textit{Sancho Panzy} dokázal Cervantes spojit vznešené ideály
se~zdravým selským rozumem.

\subsection*{Lope de Vega}
K~nejplodnějším autorům patří španělský dramatik \jmeno{Lope de Vega}. Počet jeho her odhadujeme na 2000,
z~nichž se nám dochovala jen necelá čtvrtina. Nejúspěšnější byly jeho hry s~historickým námětem, držící se 
typického schématu -- \uv{komedie pláště a~dýky} -- konflikt lásky a~cti. Konflikt mezi tyranským šlechticem a~poddaným 
je~obvykle spravedlivě rozsouzen králem, většinou ve~venkovanův prospěch. Z~jeho díla můžeme uvést hru \dilo{Fuente Ovejuna}
(Ovčí pramen), která zobrazuje lidové protišlechtické povstání.

\section{Anglie}
\subsection*{William Shakespeare}
Za~největšího světového dramatika je považován \jmeno{William Shakespeare}. O~jeho životě víme tak málo, že to někdy
vedlo k~dohadům, jestli to nebyl pseudonym celého kolektivu autorů. Narodil se ve~Stratfordu a~jako velmi mladý se oženil.
Ve~21 roku odchází do~Londýna, kde brzy vyniká jako dramatik.

V~prvním období své tvorby píše zejména komedie a~historické hry. Patří sem \dilo{Komedie plná omylů,} 
\dilo{Zkrocení zlé ženy,} \dilo{Mnoho povyku pro nic,} \dilo{Kupec benátský,} 
\dilo{Veselé paničky Winsdorské} a~především komedie \dilo{Sen noci svatojánské.}
Ve~všech těchto komediích se spletitým dějem vítězí nakonec láska a~spravedlnost.

Hrdiny jeho historických her jsou angličtí a~římští panovníci \dilo{Jindřich VI.,} \dilo{Richard III.,} 
\dilo{Jindřich IV.} a~\dilo{Julius Caesar,} kde se Shakespeare zaměřil hlavně na postavu císařova vraha \textit{Bruta}.
Do~této doby můžeme zařadit jednu z~jeho prvních tragedií -- \dilo{Romeo a~Julie.} Vypráví o~milencích ze dvou znepřátelených
Veronských rodů, kteří se snaží svou situaci vyřešit útěkem. Díky nedorozumění se jim to nezdaří a~celá hra končí
sebevraždou obou milenců.

V~druhém tvůrčím období po roce 1600 proniká do~Shakespearových her pesimismus, proto v~této době psal tragédie.
\dilo{Macbeth,} \dilo{Král Lear,} \dilo{Antonius a~Kleopatra,} \dilo{Othello} a~zejména \dilo{Hamlet.}
Hamlet, kralevic dánský, se snaží pomstít smrt svého otce. Poté, co se dozví, že vrahem je králův bratr \textit{Claudius},
upadá do~těžkých duševních stavů a~uvažuje o~smyslu života -- odtud je slavná otázka \textit{být či nebýt}.
Claudius začíná tušit, co proti němu Hamlet chystá a~vykáže ho do~Anglie. Hamlet se ale vrací a~v~souboji zabije 
\textit{Polonia} -- otce své milé \textit{Ofélie}. Ofélie se v~zoufalství utopí a~její bratr \textit{Laertes}
vyzve Hamleta na souboj. Když Hamlet zjistí, že byl podveden, král Laertovi podstrčil otrávený meč, ve~vzteku je oba 
zavraždí. Sám potom podlehne. Hlavním tématem je kritika bezohledné touhy po moci, která se před ničím nezastavuje.

Také komedie v~tomto období jsou spíše vážnějšího rázu. Jmenujme \dilo{Jak se vám líbí,} \dilo{Večer tříkrálový}
a~\dilo{Konec vše napraví.}

Postupem času se jeho pesimismus vytrácí a~smiřuje se se životem. Z~tohoto období pochází \dilo{Zimní pohádka}
nebo \dilo{Bouře.}

Shakespeare také psal verše a~prózu a~často je prokládá. Používá \textit{blankvers}, 
což je nerýmovaný pětistopý jamb.

V~žádných Shakespearových dílech nenajdeme znaky antického dramatu (osudovost, ani středověkou podřízenost víře).
Jeho příběhy jsou ryze světské a~hrdinové rozhodují o~svých osudech sami, to z~něj dělá autora čistě renesančního.