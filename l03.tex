\chapter{Literatura doby husitské}
Na přelomu 14. a~15. století byla situace na našem území značně napjatá. Církev vlastnila
zhruba polovinu půdy celého feudálního systému. Na~své bohatství přišla vybíráním desátků, 
prodáváním odpustků a~většinou falešných ostatků svatých.

Je pochopitelné, že takové praktiky církve začínaly vadit lidem, kteří nevěřily na odpuštění 
za peníze. Patřil mezi ně například \jmeno{Matěj z~Janova}, který jemně kritizoval církev ve~svém
spisu \dilo{Pravidla starého a~nového zákona} nebo \jmeno{Tomáš Štítný ze Štítného}. Byl to
český zeman, je autorem díla \dilo{Knížky šestery o~obecných věcech křesťanských,} kde pojednává
o~porušování desatera přikázání i~ze~strany církve.

Lidé našli svého mluvčího v~kapli Betlémské, kde kázal \jmeno{Mistr Jan Hus}. Vystudoval Karlovu
universitu a~byl jejím rektorem. Jeho vzorem byl Jan Viklef, anglický reformátor. Vystupoval proti
prodeji odpustků a~církevním podvodům. Ve~spisu \dilo{O~církvi} hlásá, že hlavou církve je Bůh, nikoli papež.
\dilo{Výklad Viery, Desatera a~Páteře} (víra, desatero a~otčenáš) pojednává o~významu těchto modliteb.
Jeho \dilo{Knížky o~svatokupectví,} zejména jejich část -- pojednání \dilo{Rozprava o~odpustcích} vzbudila 
v~Praze roku 1412 mezi prodejci odpustků vlnu znepokojení, proto na něj papež uvalil klatbu a~Husovi je zakázáno 
v~Praze kázat. Proto odchází k~přátelům na Kozí Hrádek a~na~hrad Krakovec, kde dál káže jako předtím v~Betlémské kapli.
Svá kázání sepisuje do~souboru \dilo{Postila} (což je souhrn kázání a~traktátů). V~roce 1414 je Zikmundem 
pozván do~Kostnice, aby se obhájil před konsilem, ten ho ale nařkne z~kacířství a~Hus je uvězněn. 
Ve~vězení píše dopisy, které jsou pašovány ven a~sesbírány do~souboru \dilo{Listy.} 

\rok{6. 7. 1415} je Mistr Jan Hus upálen.

Kromě ostatního psal i~\pojem{traktáty} například již jmenovaný \dilo{O~církvi.}
\dilo{O~českém pravopise,} kde popisuje systém české diakritiky a~zavádí \pojem{nabodeníčka}.

Jeho životopis zpracoval Petr z~Mladoňovic pod názvem \dilo{Zpráva o~Mistru Janu Husovi v~Kostnici}

Kromě Husa vedli český lid i~další kazatelé, jako byl \jmeno{Jeroným Pražský}, \jmeno{Jakoubek ze Stříbra}
a~\jmeno{Jan Želivský} -- radikální vůdce pražské chudiny.

Husitská literatura je dílo především lidové. Vznikaly chorály a~písně. Z~této doby pochází \dilo{Jistebnický kancionál,} 
v~kterém najdeme chorál \dilo{Ktož jsú boží bojovníci,} který povzbuzoval vojáky k~boji a~nepříteli
naháněl hrůzu, nebo \dilo{Povstaň, povstaň, veliké město pražské.}

\jmeno{Vavřinec z~Březové} byl inspirován husitskými boji. Napsal latinskou \dilo{Báseň vznešené koruny české} a~taktéž
latinská je \dilo{Husitská Kronika,} která pojednává o~letech 1414---1422.

Naopak \jmeno{Petr Chelčický} boje a~roztřídění společnosti odsuzoval. \linebreak[3]
Ve~spisu \dilo{O~trojiem lidu} odsuzuje rozdělení společnosti na církev, 
šlechtu a~poddané -- všichni lidé by si měli být rovni. Své radikální
názory na církev jako instituci zbytečnou shrnul do~děl \dilo{Postila}
a~\dilo{Sieť viery.}

Z~názorů Petra Chelčického vychází \jmeno{Jednota bratrská}, kterou založil mnich Řehoř Krajčí. Vzniká 
v~Kunvaldu u~Žamberka. Zpočátku odmítá vzdělání, chce rovnost lidí a~živí se fyzickou prací.
V~16.---17. století se stává nositelem kultury a~vzdělanosti. Jejími členy jsou i~Jan Amos Komenský a~Jan Blahoslav\dots{}

Jiří z~Poděbrad rozvíjí diplomatické styky se západem Evropy a~proto vznikají cestopisy.
Z~cesty do~Francie je to například \dilo{Deník panoše Jaroslava.} Inspirací pro Aloise Jiráska bylo
\dilo{Putování pana Lva z~Rožmitálu a~z~Blatné z~Čech až na konec světa,} jehož autorem je \jmeno{Václav Šašek z~Bířkova}.