\chapter[Krize konce 19. a~počátku 20. století]{Reakce básníků na myšlenkovou, společenskou a~politickou krizi konce 19.~a~počátku 20.~století}

Konec 19.~století je doba velice rozporuplná. Rozvíjí se věda, technika,
mění se společnost. Oběvují se představy o~krásném životě, ale i~nálady
katastrofismu a~anarchismu. Propukají války a~oběvují se rozpory
ve~společnosti a~krize myšlení.

Na myšlení lidí měly vliv názory filozofů. \jmeno{Henry Bergson}
zastával názor, že svět je jen představa a~umění zachycuje proces snění.
\jmeno{Arthur Schopenhauer} měl na svět pesimistický pohled. Dějiny
nemejí žádný smysl, člověk je od přírody zlý, chová se pudově a~svět
spěje ke katastrofě. \jmeno{Friedrich Nietzsche} --- učení od
nadčlověku, který může žít mimo zákon.

Umělci mají pocit, že nejsou společností pochopeni, bouří se proti ní
a~proti dosavadní kultuře. Buď tvrdí, že umění je ovlivněné politikou,
nebo se od reality odvrací úplně. Vzniká \pojem{lartpourlartismus}
tj.~\uv{umění pro umění.} Vznikají moderní umělecké směry, viz.
tabulku~\ref{tab:modernismery}.

\begin{table}[hbtp] 
{\textsc{\Large \begin{center}{Moderní umělecké směry\\počátku 20.~století}\end{center}}}
\vspace{3mm}
\begin{description}
\hrule
\vspace{-1mm}
\item[impresionismus] vznik v~70.~letech, název mu dal obraz
\emph{Cloude Moneta Imprese} (východ slunce)
\item[symbolismus] autoři se snaží vyjádřit vztah ke skutečnosti
prostřednictvím symbolů, pod každou skutečností je skrytý význam, 
který se snažíme rozeznat, používá se volný verš, hudebnost verše
\item[dekadence] \uv{úpadek,} vychází ze symbolismu, liší se tématicky
-- smutek, marnost, smrt.
\item[expresionismus] city, vztah ke světu, emoce, ostré a~křiklavé barvy
\item[secese] \uv{odchod,} umělci opouštějí akademickou půdu, chtějí
se přiblížit k~životu; inspirace přírodou; barva bílá, zelená, fialová
\end{description}
\hrule
\caption{Moderní umělecké směry na počátku 20.~století} \label{tab:modernismery}
\end{table}

\section{Předchůdci moderního umění}
Svým dílem předznamenali vznik moderních uměleckých směrů. Zakladatelem
detektivky a~moderní hororové literatury  je spisovatel \jmeno{Edgar
Allan Poe}.  Je také autorem \dilo{Filozofie básnické skladby,} kde
tvrdí, že autor si stanoví, čeho chce dosáhnout, co chce vsugerovat --
myšlenky, nálady, pak si stanoví téma a~čím toho chce dosáhnout. 
\dilo{Havran} je smutná, děsivá báseň. Tématem je smrt mladé dívky,
havran představuje symbol smrti a~opakuje refrén.

\section{Symbolismus}
Symbolismus je jedním z~uměleckých směrů tzv. \emph{literární moderny}, který
vzniká nejprve ve Francii v~poslední čtvrtině 19.~století. Je reakcí na
tehdejší realismus a~naturalismus. Obraz skutečnosti se snaží podat
prostřednictvím přenášení významu z~jedné věci na druhou -- použitím
symbolů. Ty, jako konkrétní předměty, často vyjadřují určitý abstraktní
pojem. S~tím souvisí velký rozvoj básnické obraznosti -- náladu a~dojem
vyvolávají u~čtenáře jednotlivá slova a~jejich souzvuk (hudebnost
poezie). Symbolismus nerespektuje konvence a~hranice literárních žánrů --
do poezie přináší např. volný verš. Jeho představiteli ve Francii je
trojice tzv. prokletých básníků.

\begin{quote}
\vspace{3mm}
\hrule
{\Large \textbf{Prokletí básníci}}

\emph{Prokletí básníci} jsou skupinka autorů, kteří byli ve své době
nepochopeni společností a~bouří se proti ní dílem i~způsobem života. 
\end{quote}

\subsection*{Charles Baudelaire}
Prvním z~nich byl \jmeno{Charles Baudelaire}. Ten, jako všichni tři, žil
bohémským a~provokativním životem. Také jeho rozsáhlá sbírka \dilo{Květy
zla} vzbudila velké pohoršení. Podle vlastních slov se snaží najít krásu
tam, kde ji ostatní básníci nehledali (\uv{vytěžit krásu ze zla}). Témata
svých básní čerpá z~prostředí bídy, chudiny, odpudivých jevů, jako třeba
v~básni o~koňské mršině. Řeší v~nich poslání umění a~umělce, inspiruje
se také svým ambivalentním postojem k~ženám, které ho přitahují
a~zároveň odpuzují. U~nás známe tuto sbírku v~překladu \emph{J.~Vrchlického.}

Kromě poezie se také zabývá kritikou a~překlady, např.
\emph{E.~A.~Poea}, který v~mnohém symbolismus ovlivnil.

\subsection*{Paul Verlaine}
Na \dilo{Baudelaira} zpočátku navázal mladší \jmeno{Paul Verlaine}
sbírkami \dilo{Saturnské básně} a~\dilo{Galantní slavnosti.} Pak se ale
v~jeho tvorbě objevují básně na rozdíl od Baudelaira mnohem více
osobité, náladové a~spontánní. Ke sbírkám jako \dilo{Písně beze slov} ho
inspirovalo přátelství s~\emph{Rimbaudem}, se kterým sdílel život bohéma
a~tuláka. Když ho jednou v~opilosti postřelí, ve vězení pak píše
náboženskou sbírku \dilo{Moudrost,} kde se obrací k~pokání a~víře.

\subsection*{Jean Arthur Rimbaud}
Ještě spontánnějším básníkem byl \jmeno{Jean Arthur Rimbaud}. Své
představy o~novém světě, bez konvenčnosti, ukládá do sbírky básní
v~próze i~ve verších \dilo{Iluminace}. Proslulá je jeho báseň
\dilo{Opilý koráb,} kde sebe jako básník přirovnává k~lodi zmítané
bouří, jako metaforou poezie. 

Veškeré své dílo vytvořil do 20.~roku života, pak vede dobrodružný život
vojáka a~tuláka po Evropě i~Africe.

\section{Impresionismus}
\emph{Impresionismus} vzniká opět ve Francii, nejprve jako směr
malířský. Snaží se vyvolat v~divákovi nebo čtenáři bezprostřední dojem
okamžiku. K~tomu impresionističtí malíři (\emph{Monet, Degas, Renoir,
Van~Gogh,} sochař \emph{Rodin}) bohatě využili komposice světla a~barev.

\subsection*{Fráňa Šrámek (1877 - 1952)}
Literární impresionismus se uplatňuje především v~poezii, u~nás v~díle
\jmena{Fráňy Šrámka}{Fráňa Šrámek}. Ten je ale zpočátku ovlivněn
symbolismem a~anarchismem, ve sbírkách \dilo{Života bído, přece tě mám
rád} a~\dilo{Modrý a~rudý} vyjadřuje svůj protimilitaristický
a~protirakouský postoj (za ten byl na vojně o~rok déle).
Ve~vzpomínkových sbírkách \dilo{Básně} a~\dilo{Nové básně} se vrací do
rodného kraje (Sobotka), opět k~antimilitarismu se vrací ve sbírce
\dilo{Ještě zní.}

Do impresionismu můžeme zařadit Šrámkovu lyrickou sbírku \dilo{Splav,}
kde spojuje pohanské přírodní mýty s~tématem milostného vztahu. Podobný
námět mají i~romány \dilo{Křižovatky} (skeptické), \dilo{Tělo} (vliv
války na lásku a~krize jen tělesné orientace) a~\dilo{Stříbrný vítr,}
s~postavou studenta Jana Ratkina který svůj rozpor s~pokryteckými
společenskými konvencemi překonává díky \uv{stříbrnému větru} -- věčné síly
mládí.

Impresionismus se objevuje i~ve Šrámkových dramatech. Ve hře \dilo{Léto}
oslavuje mladou lásku jako protiklad strojeným měšťáckým mravům.
Konflikt mládí a~stáří se objevuje i~v~jeho poválečných dílech (\dilo{Měsíc
nad řekou}).

\section{Dadaismus}
Jedním ze zakladatelů dadaismu byl za 1.~světové války Rumun
\jmeno{Tristan Tzara}. Hnutí dada vzniklo ve Švýcarsku jako protest
proti všem dosud uznávaným hodnotám. Principem byla náhodná \uv{výroba}
uměleckých děl např. z~ústřižku novin, řazením jednotlivých slov bez
logického smyslu. Následovníkem je pak \emph{surrealismus}.

\section{Futurismus}
\emph{Futurismus} vzniká na začátku 20.~století jako součást anarchistického
odporu proti dosavadním normám. Obdivuje techniku a~civilizaci a~chce
i~umění odpovídající budoucímu průmyslovému věku. Chce \uv{osvobodit slova}
zrušením interpunkce a~využívá zvukomalbu k~vyjádření zvuků strojů,
letadel\dots{} Iniciátorem futurismu byl Ital \jmeno{Filippo Tomasso Marinetti}
svým Manifestem futurismu.

\section{Česká moderna} 
Mimo jiné jako výraz odporu proti vládnoucím \emph{Mladočechům} byl roku
1895 vydán \emph{Manifest České moderny}. Jako kritici se pod ním
podepsali F.~X.~Šalda a~F.~V.~Krejčí, jako prozaici V.~Mrštík
a~J.~K.~Šlejhar a~z~básníků J.~S.~Machar, A.~Sova a~O.~Březina. Těmto
mladým umělcům je ale společný jen nesouhlas s~dosavadním uměním, kvůli
různým cílům se rozpadá.

\subsection*{Josef Svatopluk Machar}
Iniciátorem manifestu byl \jmeno{Josef Svatopluk Machar}. Pocházel
z~Kolína, studoval filosofii v~Praze. Pracoval jako úředník ve Vídni
a~po válce jako generální armádní inspektor.

Hned ve své prvotině, trojdílném \dilo{Confiteoru I---III}
(\emph{Confiteor, Bez názvu, Třetí kniha lyriky}) vyjadřuje své pojetí
poezie jako osobní zpověď básníka, tím odmítá službu vlasteneckým
a~nadosobním ideálům. Svou poezii se rovněž snaží psát hovorovým
jazykem, jednoduše a~bez patosu.

Milostné zklamání mu bylo podnětem k~vytvoření pesimistických \dilo{Čtyř
knih sonetů.} V~nich tradiční pojetí sonetu jako milostné lyriky
rozšiřuje o~vyjádření svého zklamání nad moderní civilizací a~pocitů
zbytečnosti života. 

Sbírka politických básní \dilo{Tristium Vindobona} (\emph{žalozpěvy
z~Vídně}) není vlastenecká, na obrozenecký boj se dívá s~věcnou skepsí.

V~souboru povídek \dilo{Zde by měly kvést růže} vypráví Machar několik
příběhů žen a~dívek. Líčí jejich zoufalství vyplývajícího z~jejich
nerovného postavení ve společnosti, kdy jejich romantické ideály obvykle
končí v~nerovném manželství.

Podobně veršovaný román \dilo{Magdaléna} odsuzuje nejen prostitutku, ale
i~okolní pokryteckou společnost. Společenskou satirou jsou i~\dilo{Boží
bojovníci.}

Stejně jako \emph{Hugo} nebo \emph{Vrchlický} i~Machar se rozhodl
zpracovat dějiny lidstva v~cyklu nazvaném \dilo{Svědomí věků.} V~něm si
všímá vývoje společenské morálky a~náboženství od antiky až po
současnost. Mezi silnými osobnostmi minulosti hledá autor ztělesnění
svého ideálu. Začíná zamyšlením nad možností zneužití náboženství
v~antické Golgatě. Antický námět najdeme i~ve sbírkách \dilo{V~záři
helénského slunce} a~\dilo{Jed z~Judey,} kde Machar vyslovil názor, že
křesťanství přineslo do optimistického pohanského náboženství jen
odříkání. Další sbírky věnuje středověku (\dilo{Barbaři}), renesanci
(\dilo{Pohanské plameny}) i~francouzské revoluci (\dilo{Roky za století}).

\subsection*{Antonín Sova (1864 - 1928)}
Mezi impresionisty můžeme zařadit jihočeského básníka z~Pacova
\jmena{Antonína Sovu}{Antonín Sova}. Po právnických studiích v~Praze se
nakonec stal úředníkem pražské Městské knihovny. Jeho samotářský život
jen prohlubovalo částečné ochrnutí.

Atmosféru svého rodného kraje zachytil v~intimních sbírkách \dilo{Květy
intimních nálad} a~\dilo{Z~mého kraje.} Spíše než o~přesnost popisu se
tu snaží o~vystižení nálady a~dojmu z~krajiny. Ve druhé sbírce se navíc
zamýšlí nad její husitskou minulostí.

Od~impresionistických krajinných námětů se Sova obrací k~vyjádření
zklamání moderního člověka a~úsilí o~nápravu měšťácké společnosti.
Vypovídají o~tom sbírky \dilo{Soucit a~vzdor} a~\dilo{Zlomená duše,} 
ve které částečně autobiograficky líčí etapy ze života člověka formou sedmi
monologů.

Stejná snaha o~vytvoření harmonické společnosti, doplněná symbolickými
vizemi se objevuje ve sbírkách \dilo{Vybouřené smutky,} \dilo{Údolí
nového království} a~\dilo{Dobrodružství odvahy.} Objevuje se tu obraz
poutníka, vyhnaného okolím na \uv{horu snů} a~hledajícího cestu zpět
k~lidem.

Sbírku \dilo{Zpěvy domova} psal Sova za války, proto se tu objevuje
motiv lásky k~domovu a~národním tradicím, zejména husitství na venkově.

Ve svých posledních sbírkách se znovu vrací k~rodnému kraji -- \dilo{Básníkovo
jaro,} \dilo{Drsná láska.}

Lyrická je i~Sovova próza, píše zejména psychologické romány. \dilo{Ivův
román} je portrétem samotářského studenta, \dilo{Pankrác Budecius,
kantor} je o~těžce zkoušeném venkovském muzikantu a~učiteli.

\subsection*{Otokar Březina}
Nejvýraznějším českým symbolistou je \jmeno{Otokar Březina}. Přestože působil
jako venkovský učitel, dosáhl značného filosofického a~jazykového
vzdělání a~publikoval v~předních časopisech (\emph{Rozhledy, Moderní Revue}).

Na své smutné dětství a~mládí v~chudých poměrech vzpomíná v~první sbírce
\dilo{Tajemné dálky.} Jeho pesimistická samota je způsobená smrtí obou rodičů.

Ve sbírce \dilo{Svítání na západě,} kde je západ symbolem smrti, Březina tvrdí,
že pochopení smyslu života je možné až po vysvobození z~jeho strastí~--~smrti.

Nejabstraktnější jeho sbírkou jsou \dilo{Větry od pólů,} snaží se zde
oprostit od hmotného světa a~hledá odpověď na otázku, kdo vlastně ovládá
svět. Nachází smysl života v~soucitu s~lidmi, v~lásce, ale zároveň
v~bolesti.

Poslední dvě symbolické sbírky, \dilo{Stavitelé chrámu} a~\dilo{Ruce,}
vyjadřují úsilí umělců, básníků, kteří staví \uv{dokonalejší chrám
světa.} Kromě toho se tu objevuje představa spojení každého člověka
neviditelnými pouty s~vesmírem a~jeho řádem.

Březinovu tvorbu uzavírají knihy náboženských a~filosofických esejů
\dilo{Hudba pramenů} a~nedokončené \dilo{Skryté dějiny.}

\section{Generace buřičů}
\subsection*{Viktor Dyk}
S~označením \uv{buřiči} se poprvé setkáváme u~Viktora Dyka, spisovatele,
novináře, ale i~politika. Své ideje protispolečenské vzpoury uplatňuje
ve všech literárních žánrech.

Začíná básnickými sbírkami \dilo{A~porta inferi,} \dilo{Síla života}
a~\dilo{Marnosti} s~vlivy symbolismu a~dekadence.

Svůj postoj definuje v~knize \dilo{Buřiči a~smíření,} na postavách tří
zločinců, ukázkou báseň \dilo{Milá sedmi loupežníků,} o~dívce, která se
ze msty za zabití přítele zbaví svých druhů.

Satirou na politiku i~národní charakter jsou sbírky \dilo{Satiry
a~sarkasmy} a~\dilo{Pohádky naší vesnice} (báseň \emph{Smutná píseň
vesnického šprýmaře}). Oslavou lidské objevitelské touhy je epická
skladba \dilo{Giuseppe Moro.} Za války se Dyka zmocňuje obava
o~budoucnost národa -- ve sbírkách \dilo{Anebo} a~\dilo{Okno.} Jeho
posledním dílem je \dilo{Devátá vlna,} sbírka intimní a~meditativní poezie.

Bouřlivý politický vývoj roku 1897 se Dyk pokusil zachytit
v~nedokončeném cyklu \dilo{Akta působnosti Čertova kopyta.} Z~jeho
povídek je známý \dilo{Krysař,} variace na středověký námět. Předlohu,
pověst o~krysaři v~Hameln, tu doplnil ještě o~milostný motiv.

Dykova dramata mají rysy romantismu a~symbolismu, stejně jako ve svých
prozaických a~básnických dílech zde řeší konflikt snu a~skutečnosti. Děj
\dilo{Revoluční trilogie} se odehrává za \emph{Velké francouzské revoluce},
Cervantesův námět zpracovává ve hře \dilo{Zmoudření Dona Quijota,} kde
bere hrdinovo závěrečné prohlédnutí za ztrátu smyslu života.

%\subsection*{Jiří Karásek ze Lvovic}
%Spoluzakladatelem Moderní revue byl \jmeno{Jiří Karásek,} kritik, básník a~sběratel uměleckých děl. Svými sbírkami \dilo{Zazděná okna,} \dilo{Sodoma} a~\dilo{Kniha aristokratická} opěvuje samotu a~lásku. Společnost zde ale provokuje svou orientací na morbidní a~perverzní jevy (byl údajně homosexuál).

\subsection*{Karel Hlaváček}
Mezi buřiče patří také \jmeno{Karel Hlaváček}. Můžeme ho také zařadit do
symbolismu a~dekadence. Dekadentní jsou jeho sbírky \dilo{Pozdě k~ránu}
a~\dilo{Mstivá Kantiléna.} Své výtvarné nadání uplatnil při ilustrování svých
vlastních děl. Posmrtně byly vydány \dilo{Žalmy,} básně s~myšlenkou blízké
smrti a~náboženskými tématy.

\subsection*{Karel Toman}
Dalším autorem symbolismu a~dekadence je básník \jmeno{Karel Toman},
redaktor \emph{Národních listů}, knihovník a~cestovatel. Do těchto směrů se
zařadil už prvotinou \dilo{Pohádky krve.}

Milostnou lyriku obsahují sbírky \dilo{Torzo života}
a~\dilo{Melancholická pouť.} Ve sbírkách \dilo{Sluneční hodiny}
a~\dilo{Verše rodinné a~jiné} je patrný básníkův vývoj od individualismu
k~širší realitě, obdivu k~přírodě a~solidaritě s~chudými. K~vytvoření
cyklu dvanácti básní \dilo{Měsíce} ho inspirovaly lidové tradice.
V~závěrečných sbírkách \dilo{Hlas ticha} a~\dilo{Stoletý kalendář} se
opět ztotožňuje s~tuláky a~chudými, které chápe jako symbol svobody.

\subsection*{František Gellner}
Provokativní je i~protispolečenský postoj \jmena{Františka
Gellnera}{František Gellner}. Ironický výsměch a~kritika maloměšťáctví
je patrná už z~názvů sbírek \dilo{Po nás ať přijde potopa!}
a~\dilo{Radosti života.} V~posmrtně (padl ve válce) vydané sbírce
\dilo{Nové verše} je vidět určité zklidnění, ironie přechází ve smutek.

Gellner se pokusil i~o~prózu, román \dilo{Potulný národ} umístil do
bohémského prostředí Paříže.

\subsection*{Stanislav Kostka Neumann}
\label{chap:SKNbur}
\jmeno{Stanislav Kostka Neumann} začínal jako anarchista, byl jako jeden
z~členů \emph{Omladiny}\footnote{hnutí za nezávislost na Rakousko-Uhersku} 
zatčen a~uvězněn. Přispíval do časopisů \emph{Moderní revue, Nového kultu}
a~vydává vlastní \emph{Anarchistickou revue}. Vstoupil také do spolku \emph{Devětsil}.

% % % % % % % % % % % % % % % % % % % % % % % % % % % % % % % % % % % %
Anarchistické jsou i~jeho první sbírky -- \dilo{Jsem apoštol
nového žití,} \dilo{Satanova sláva mezi námi,} \dilo{Apostrofy hrdé
a~vášnivé} nebo \dilo{Sen o~zástupu zoufajících,} kde se utiskovaný lid
marně snaží najít Boha, a~tak se nakonec obrací k~Satanovi. Během svého
pobytu na venkově, u~Brna, vydává působivou sbírku přírodní a~milostné
lyriky \dilo{Kniha lesů, vod a~strání,} kde vyjadřuje své okouzlení být
součástí přírodního dění. Oproti tomu \dilo{Nové zpěvy} jsou oslavou
technického pokroku, velkoměsta a~tvůrčí práce.

Od proletářské poezie (viz str. \pageref{chap:SKNagit}) postupně přechází k~odporu proti německé okupaci
sbírkami \dilo{Srdce a~mračna,} \dilo{Sonáta horizontálního života,} 
\dilo{Bezedný rok} a~\dilo{Zamořená léta.}

\subsection*{Petr Bezruč}
Těžko zařaditelný autor \jmeno{Petr Bezruč} má nejen časově, ale i~svou
společenskou kritikou, blízko k~buřičům. Pocházel z~Opavy z~rodiny
profesora a~národního buditele, po gymnáziu a~nedokončených studiích na
právech pracuje jako poštovní úředník.

Počátečním podnětem k~psaní poezie byla jeho milostné a~zdravotní krize.
Zpočátku své básně anonymně posílá do časopisu \emph{Čas} a~jejich menší
soubor vyšel pod názvem \emph{Slezské číslo.} Podstatně rozšířené byly
pak vydány ve sbírce \dilo{Slezské písně.} Balady v~ní obsažené můžeme
rozdělit na intimní, sociální a~národnostní.

Ve spíše symbolických intimních baladách se inspiruje vlastním osudem
a~osamělostí (\dilo{Červený květ}) a~touhou po rodinném štěstí
(\dilo{Jen jedenkrát}).

V~sociálních baladách se Bezruč cítí být věštcem, bardem slezského lidu
a~jeho vůdcem proti utlačovatelům. Právě proto, že své básně vydával
anonymně, mohl si dovolit nepřátele konkrétně jmenovat a~děj balad
umístit do skutečných míst. Sociální problémy dokáže vyjádřit rychlým
a~dramatickým dějem balady -- konkrétním tragickým osudem (\dilo{Maryčka
Magdonova,} \dilo{Kantor Halfar,} \dilo{Markýz Géro,} \dilo{Bernard
Žár}). Silného dojmu dosahuje kontrasty, opakováním, dramatickými
dialogy a~použitím nářečních výrazů. Balady \emph{Maryčka Magdonova}
a~\emph{Markýz Géro} též zhudebnil \emph{Leoš Janáček}.

Netradiční je i~Bezručův pohled na vlastenectví v~národnostních baladách
(\dilo{Den Palackého}), nesouhlasí s~oficiální českou politikou, která
podle něj na lid zapomíná.

Jedna z~mála Bezručových samostatných básní je \dilo{Stužkonoska modrá.}
Celoživotní hledání vzácné můry symbolizuje básníkův smutný život.