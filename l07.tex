\chapter{Světová literatura 17. a~18. století}

\section{Klasicismus}
Klasicismus je umělecký směr, který se rozvíjel v~17.---18. století
ve Francii. Vzniká jako reakce na baroko a~na absolutismus (což 
je neomezená vláda panovníka). Ve Francii byl absolutismus silný -- vládl
Ludvík XIV.. Objevuje se představa, že vše musí být podřízeno nějakým
zákonům. \jmeno{Descartes} klade důraz na tvorbu lidského rozumu -- nový
směr \pojem{racionalismus}. Malířství a~sochařství napodobuje antiku,
obrazy jsou klidné, umírněné.

V~literatuře se objevuje přímost a~vyrovnanost. Objevují se dva typy 
žánru -- vyšší, kde vystupují pouze lidé z~vyšších vrstev, a~nižší. Mezi
vyšší patří tragédie, epos a~óda, mezi nižší pak komedie a~podobné.
Převládá poezie, v~dramatu se dodržuje zákon tří jednot.\footnote{místa,
času a~děje} Námět je čerpán z~antiky, tématem bývá rozpor mezi povinností
a~ctí a~láskou a~city. Objevuje se postava panovníka, který má funkci
deus ex machina.\footnote{zásah stroje vyslaného bohem}

\subsection*{Vyšší žánry}
Autorem hrdinských tragedií je \jmeno{Pier Corneille}. Námět si bere většinou
z~historie. Příkladem klasicistní tragedie je \dilo{Cid.} Don Rodrigo,
zvaný \emph{Cid} se zamiluje do \emph{Ximeny}. Mezi rody ale dojde k~hádce
a~\emph{Cid} se musí rozhodnout mezi ctí nebo láskou. Vítězí čest 
a~\emph{Cid} svede úspěšně boj proti otci své milované. Nastává
série sporů, kterou rozřeší až král. V~závěru se \emph{Cid} a~\emph{Ximera}
mohou vzít, až uplyne její smutek. Typický hrdina je čestný, spravedlivý,
pomstychtivý a~odhodlaný.

Autorem psychologické tragédie je \jmeno{Jean Racine}. Jeho dílo \dilo{Faidra}
vzbudilo ve své době pobouření. Je to příběh ženy, která se zamilovala 
do svého nevlastního syna, ale protože ji odmítne, obviní ho a~on umírá.
Faidra pak spáchá sebevraždu.

\subsection*{Nižší žánry -- próza}
V~tomto období se objevují též bajky. Hlavním představitelem je
\jmeno{Jean de La Fontain}, který napsal celkem 12~knih bajek.
Náměty si bere od Ezopa.

Jedním z~nejslavnějších dramatiků, hned po \emph{Shakespearovi} je
\jmeno{Moli\` ere}. Napsal kolem třiceti her a~byl z~nejoblíbenějších
v~Paříži. Většinou jsou to frašky a~komedie s~hudebním a~tanečním
doprovodem. Vystupují postavy ze všech vrstev, většinou jsou to
měšťané, dává do kontrastu lidské chyby a~vady společnosti se zdravým
rozumem. Jeho hra \dilo{Tartuffe} ukazuje, jak \textit{špiclové} 
z~církevních organizací pronikají do rodin. \dilo{Lakomec} vše podřizuje
penězům. \dilo{Zdravý nemocný} je hypochondr, který chce provdat
dceru za lékaře, aby měl léčení zdarma. Nejsmutnější jeho hra
je \dilo{Misantrop} -- nepřítel lidí. Ve skutečnosti je to nepřítel
pokrytectví a~přetvářky.

\subsection*{Itálie}
V~Itálii vzniká klasicismus z~ústní lidové slovesnosti. Ironicky 
zobrazuje všední problémy. Herci zakládají své vystoupení
na~improvizaci, vystupují stejné postavy. Jedním z~autorů byl
\jmeno{Carlo Goldoni}, který situuje své hry do Benátek. Z~jeho díla
známe hry \dilo{Sluha dvou pánů} nebo \dilo{Poprask na laguňe,} 
kde vystupují chudé postavy.

\section{Osvícenství}
Rozvíjí se v~18. století v~Anglii a~Francii, navazuje na Renesanci.
Osvícenství zdůrazňuje úlohu osvíceneckého rozumu, odmítá vše, co
odporuje zdravému rozumu. Zavrhuje absolutismus. Je to směr filosofický
a~literární, zdůrazňuje rovnost mezi lidmi, svobodu myšlení, stal se
ideologií francouzské buržoazní revoluce. Prvním pokusem o~shromáždění
veškerého lidského vědění byla \dilo{Velká francouzská encyklopedie,}
která vznikala v~letech 1751---1765. Ovlivnila myšlení lidí a~její ideje
se rychle šíří. 

Všestranný autor (filosof, historik, dramatik, básník a~publicista) byl
\jmeno{Voltaire}. Napsal mnoho divadelních her. Filosofická povídka
\dilo{Candide} bojuje proti dobovému optimismu a~tvrzení, že všechno zlé
je k~něčemu dobré. Parodií na mýtus o~panně Orleánské\footnote{Johanka z~Arku}
je směšnohrdinský epos \dilo{Panna.}

Známý autor této doby je \jmeno{Denis Diderot}, matematik, filosof
materialista, literární kritik, podílel se na vydání
\emph{Encyklopedie}. Rozbíjí formy klasicismu, rozvíjí koncepci
realismu. Románem \dilo{Jeptiška} bojuje proti umisťování dívek
do~klášterů. Dopisní formou je zde popsána marná snaha jeptišky dostat
se ven z~kláštera, proti ní jsou úřady i~rodiče.

Formou dialogu je psán \dilo{Rameaův synovec,} který zpodobňuje
vypočítavého příživníka.

Román o~lásce, štěstí a~neštěstí v~lásce, o~sluhovi a~jeho pánu
\dilo{Jakub fatalista} inspiroval \emph{Jana Kunderu}.

Vznikají i~jiné žánry, jako \emph{směšnohrdinský epos} 
nebo \emph{travestie}, která nicotné věci vykládá vznešeně.

\section{Preromantismus -- sentimentalismus}
Rozvíjí se ve francouzské literatuře od poloviny 18.~století, vzniká
jako reakce na absolutismus. Tvoří přechod mezi klasicismem
a~romantismem. Důraz je kladen na cit, odvrací se od skutečnosti.
Navrací se k~přírodě --- člověk, který žije v~přírodě je dobrý
a~nezkažený. Ústní lidová slovesnost odráží duši národa, proto je
napodobována, sbírána\dots{}

Používá se ich\footnote{1. pád čísla jednotného} forma, forma deníkových
zápisů nebo romány v~dopisech. Často se oběvuje motiv nešťastné lásky.
Hrdinové se bouří proti společenským konvencím, mají jinou představu
o~životě a~jak by měl vypadat vztah mezi lidmi. Děj se často odehrává
v~exotické oblasti.

\subsection*{Francie}
\jmeno{Fran\c cois René de Chateaubriand} je jedním z~autorů, který děj
svého románu o~nešťastné lásce umístil do vzdálené země. \dilo{Atala} je
jméno dívky, kterou beznadějně miluje indián Šalta. Používá básnický
jazyk, důraz klade na city, pocity a~nálady.

% jak se píše Prevost, Mannon Peníze jako nepřítele lásky ukazuje
\jmeno{Antoine Fran\c cois Pr\`evost} v~románu \dilo{Mannon Lescaute.} Je
to příběh rytíře, který se má stát knězem, ale zamiluje se do dívky
\emph{Mannon}, pro kterou jsou nejdůležitější peníze. Je odsouzena
k~deportaci, des Gieux odchází s~ní a~opouštějí Evropu. (Později
přepracoval V.~Nezval.)

\jmeno{Jean Jacque Rousseau} chce návrat k~přírodě, tvrdí, že všichni
lidé jsou si rovni. V~eseji \dilo{O~původu společenské nerovnosti} vidí
příčiny sociální nerovnosti v~soukromém vlastnictví. Tvrdí, že lidé jsou
od přírody dobří, ale kazí je osobní vlastnictví. Esej
\dilo{O~společenské smlouvě} prosazuje myšlenku, že lid má právo změnit
formu vlády. Pedagogický román \dilo{Emil} se zabývá výchovou chlapce.
Dítěti se nemají vnucovat příkazy, ale má se učit chybami. Je autorem
také románu v~dopisech \dilo{Nová Heloisa} o~tragické lásce, 
kde zobrazuje kontrast mezi životem v~přírodě a~ve společnosti.

\subsection*{Anglie}
V~anglické literatuře se objevuje \pojem{sentimentální román}, který
popisuje příběhy ctnostných hrdinů. Romány většinou končí šťastně,
inspirovali \emph{červenou knihovnu}. Jejich představitelem je
\jmeno{Samuel Richardson}, který je autorem románu \dilo{Pamela.}

Vedle toho vznikají další romány, které mají blíže k~osvícenství.
Vztah člověka k~přírodě zachytil \jmeno{Daniel Defoe} ve známém
románu \dilo{Robinson Crusoe.}

Satiru, utopii a~filosofický román psal \jmeno{Jonathan Swift}.
Kritizoval anglické poměry a~uváděl představu budoucí společnosti.
Je~autorem románu \dilo{Gulliverovy cesty.} 

\subsection*{Německo}
\jmeno{Johann Wolfgang Goethe} Prosazuje tvůrčí svobodu, je členem
hnutí \emph{Sturm und Drang} (\uv{Bouře a~vzdor}). Tyto myšlenky
vyjadřuje v~básni \dilo{Prométheus} nebo formou dopisů v~románu
\dilo{Utrpení mladého Werthera.} Dále se snaží náměty z~antiky
vychovávat měšťanskou třídu k~jejímu novému postavení ve společnosti.
V~tomto duchu píše veršované tragédie \dilo{Ifigenie na Tauridě}
a~\dilo{Torquato Tasso.} Ve~svém myšlenkově nejhlubším díle, dvoudílné
tragédii \dilo{Faust,} líčí osud nespokojeného doktora, který v~touze
po~poznání uzavře smlouvu s~ďáblem \emph{Mefistofelem}. Ten, aby ho
od~tohoto záměru odvrátil, ho zahrne světskými požitky. Faust ale
propadá zoufalství nad smrtí své lásky Markétky, která je odsouzena
za~vraždu svého a~Faustova dítěte, kterou zaviní Mefistofelos. Smysl
života nachází ve snaze pomáhat ostatním lidem. Přestože tím dosáhl
uspokojení a~měl by tedy propadnout peklu, jeho ušlechtilý záměr 
ho zachraňuje.  

Představitelem osvícenství a~klasicismu je \jmeno{Lessine}. Byl
dramaturgem německého národního divadla, psal měšťanská dramata.
\dilo{Emillia Galloti} je tragédie o~měšťanské dceři o~niž usiluje
panovník. Aby ji získal, zabije jejího snoubence. Ale ona se raději  nechá
zabít svým otcem.

\jmeno{Friedrich Schiller} vyjadřuje svým dílem touhu po svobodě.
Ve~svých dílech vyjadřuje lidské city a~protestuje proti násilí.
Je autorem básně \dilo{Óda na radost,} kterou později zhudebnil
\emph{Ludvík von Beethoven}.
Největší částí jeho tvorby je ale drama. Napsal spoustu her
inspirovaných historií, jako třeba \dilo{Valdštejn} o~zavraždění
hraběte Josefa Valdštejna, \dilo{Marie Stuartovna} nebo 
\dilo{Svatá Jana} o~Janě z~Arku.
