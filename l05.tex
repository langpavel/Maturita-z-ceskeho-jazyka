\chapter{Česká literatura renesanční a~humanistická} % 5. otázka
Renesance začíná do~českých zemí pronikat již za vlády Karla IV., zejména Petrarcovi latinské spisy. Její vývoj je 
však umlčen husitskými válkami, a~tak se u~nás objevují až v~2. polovině 15. století překlady \jmena{Hynka z~Poděbrad}{Hynek z~Poděbrad}
(přeložil například Boccacciův \dilo{Dekameron}). V~Čechách mluvíme především o~humanismu, protože se u~našich
autorů neprojevuje okouzlení životem

S~renesanční snahou o~vzdělání člověka souvisí rozvoj \textit{naučné literatury}. Sem můžeme zařadit 
\dilo{Kroniku českou} \jmena{Václava Hájka z~Libočan}{Václav Hájek z~Libočan}. Její věcná nepřesnost je vyvážena 
čtenářskou poutavostí, proto byla ještě dlouhou dobu populární. Mezi naučnou literaturu patří i~cestopisy. Popis cesty 
do~svaté země popisuje poutavou formou s~množstvím věcných informací \jmeno{Kryštof Harant z~Polčic a~Bezdružic}

S~rozšířením Gutenbergova vynálezu knihtisku roku \rok{1445} je spjat i~vznik našich nakladatelství založených
\jmena{Danielem Adamem z~Veleslavína}{Daniel Adam z~Veleslavína} a~\jmena{Jiřím Malentrichem}{Jiří Malentrich}.

Rozšíření vzdělávání mezi obyčejnými lidmy prosazoval \jmeno{Jan Blahoslav}, učitel a~biskup Jednoty bratrské.
Ve~svém spise \dilo{Filipika proti misomusům} obhajuje potřebu vyššího vzdělání jako podmínku pokroku.
O~vytříbenost jazyka se snaží v~\dilo{Gramatice české,} v~díle \dilo{Vady kazatelů} upozorňuje na správné 
rétorické vyjadřování. Též přeložil \textit{Nový zákon} do \dilo{Bible kralické,} jejíž jazyk byl považován 
za nejlepší až do doby obrození.

Historií se zabýval univerzitní profesor \jmeno{Jan Campanus Vodňanský}, po bitvě na Bílé hoře se stal rektorem.
Zveřejňoval dějiny českých zemí a~na motivy jeho života napsal \textit{Zikmund Winter} dílo \textit{Mistr Kampanus}.

Chrudimský rodák \jmeno{Viktorin Kornel ze Všehrd} přeložil \dilo{Knihy o~napravení padlého} a~\dilo{Knihy devatery}
latinsky píšícího autora \textit{Jana Zlatoústého}. 