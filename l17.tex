\chapter{1.~světová válka ve světové a~české literatuře} %[Odraz 1.~světové války]

Téma 1.~světové války se neustále vrací, protože znamenala pro lidi
veliký otřes, především morální. Zpočátku si mysleli, že vedou poctivou
obrannou válku, ale ukázalo se, že to je válka útočná. To ovlivnilo
život všech lidí a~též autorů.

Téma se vrací ve dvou vlnách. Bezprostřední reakce na válku se objevují
ve 20.~letech, varují před další válkou. Ve 30.~letech se reakce na
válku vracejí, chýlí se k~další válce a~autoři před ní varují. Reakce se
oběvuje ve všech literaturách krom Ruské (kvůli revoluci).
Nejlepší díla vznikají ve Francii, Německu, Americe a~u~nás.

\section{Francie}
Francouzská literatura je k~válce velmi kritická. Hledá její příčiny
a~odsuzuje ji. Ve Francii následkem války vznikla deziluze,\footnote{ztráta iluzí, představ}
protože francouzi byli přesvědčeni o~spravedlivosti jejich války.

\delic

%\subsection*{Henry Barbusse}
\jmeno{Henry Barbusse} byl básník, prozaik, píše pod vlivem symbolismu.
Byl přesvědčen, že vedou spravedlivou válku.  Jeho román \dilo{Oheň} byl
první protiválečný román, psaný formou deníku vojenské četby, pásmo scén
bez ústředního hrdiny, vypravěč události nehodnotí, založeno na
kontrastu mezi drastickými scénami na bojišti a~mezi boji, nebo
v~zázemí, kde je klid.

\delic

\jmeno{Romain Rolland} byl životem i~dílem přesvědčený humanista,
odpůrce války, účastní se boje proti fašismu. Napsal spoustu životopisů
zpracovaných umělecky. 

\dilo{Jan Kryštof} je románový cyklus psaný formou řeka -- všechny
romány spojuje řeka Rýn, neustále se vrací ve všech čtyřech dílech.
Příběh se odehrává v~2.~polovině 19.~století. Fiktivní příběh geniálního
hudebníka -- předobrazem Beethovena. Zabývá se otázkou vztahu k~hudbě,
kritizuje francouzský i~německý způsob života. Hlavní hrdina touží po
světě a~svobodě.

Novela \dilo{Petr a~Lucie} je inspirovaná 1.~světovou válkou, protestuje
v~ní proti nelidskosti války, válka je pouze pozadí. 
%dopsat

\delic

\jmeno{Antoine de Saint--Exup\`ery} byl žurnalista, letec sestřelený
v~2.~světové válce. Píše prózu na pomezí mezi reportáží a~dokumentem,
zároveň píše lyrizovanou prózu, která vykesluje atmosféru.
Tyto reportáže jsou shrnuty do cyklu \dilo{Pode mnou země.}

Filosofická pohádka \dilo{Malý princ} odráží otázky existence člověka.
Jako pilot havaroval v~poušti. Setká se tam s~princem, který žije na
poušti jenom s~růží. Vezme ho na Zem, tam se princ seznámí s~liškou.
Vezme pilota na různé planety, které zosobňují negativa.

\delic

Román \jmeno{Andr\`e Gida}{Andr\`e Gide} -- \dilo{Penězokazi}
předznamenává krizi románů. Hlavní hrdina je spisovatel, homosexuál,
který píše román \emph{Penězokazi}, aniž by věděl o~čem bude.
Prolíná se jeho osud s~dalšími osudy 3 chlapců.

\section{Německá}
V~německé literatuře je téma 1.~světové války nejrozšířenější, protože
Německo válku začalo, ale také ji prohrálo. Válka je zobrazena jako něco
absurdního, nesmyslného a~zbytečného.

\jmeno{Erick Maria Remarque} patřil k~tzv. \emph{ztracené generaci},
generaci mladých lidí, kteří se narodili v~90.~letech, neznali nic
jiného, než válčení, a~proto nemohli najít místo v~životě. Romány
reagují na soudobé otázky, tj. 1.~světová válka, nástup fašismu
a~2.~světovou válku. Jeho díla velice dobře vystihují dobu (postavy,
politické poměry). Romány jsou šablonovité, opakují se stejné motivy.
\dilo{Na~západní frontě klid} je román, který zachycuje osud studenta
Pavla Braunera společně s~jeho spolužáky. Ti se nechají zfanatizovat
jejich učitelem. Když se přihlásí do armády, zjišťují, že velké cíle
jsou jen vzdálené ideje jejich profesora. Pavel umírá právě když hlásí,
že je na západní frontě klid.

\delic

\jmeno{Arnold Zweig} ze začátku píše psychologické romány ovlivněné
impresionismem, vykresluje přírodu a~nálady lidí. Jeho tvorba se mění
během 1.~světové války. V~cyklu \dilo{Velká válka bílých mužů} zobrazuje
válku, její nesmyslnost a~absurditu (8 dílů). \dilo{Spor o~seržanta
Grišu} je první část. Vychází ze skutečného příběhu. Hlavní hrdina je
ruský zajatec Grišov, který se chce vátit zpět do Ruska. Zastaví se
u~jedné ženy, která mu dá doklady jiného muže. Přestože všichni vědí, že
to není špion, je na základě cizích dokladů zadržen a~popraven. Další
díly jsou například romány \dilo{Žena} nebo \dilo{Bitva u~Verdenu.}

\delic

\jmeno{Thomas Mann} je autorem románu \dilo{Buddenbrookovi,} který
zachycuje vzestup a~úpadek jedné rodiny. Obsahuje autobiografické rysy.
Jako hlavní hrdiny si často vybírá umělce, protože rozpory ve
společnosti prožívají hlouběji, než ostatní. Oběvuje se zde protest
proti fašismu. V~románu \dilo{Doktor Faustus} se zabývá otázkou umění
a~vztahu umělce ke světu. Je to parafráze na faustovský motiv. Oběvuje se
zde typický styl \emph{Thomase Manna} -- ironie, parodie, montáž (autor
kombinuje citáty z~dokumentů, slovníků, skládá je do románového děje).
Hlavní postavou je hudebník \emph{Adrian Levekuhn}, který je ochoten
zřeknout se lásky za inspiraci a~upsat se ďáblu. Tvorba vrcholí jeho
skladbou \uv{Poslední,} která je negací Beethovenovy deváté symfonie.
Děj se odehrává na pozadí dějin 20.~století a~nástupu Fašismu a~války.

Románová tetralogie \dilo{Josef a~jeho bratři} je historický příběh,
který vychází z~Bible, o~Josefovi.

\delic

\jmeno{Heinrich Mann} kritizuje maloměšťáctví a~morálku. \dilo{Profesor
Neřád} je příběh profesora \emph{Raata}, který byl nazývaný svými
studenty a~spolupracovníky \emph{Neřád}. Tyranizuje, špehuje studenty.
Zamiluje se však do prostitutky, ožení se s~ní. Od toho okamžiku začíná
jeho morální úpadek.

\delic

\jmeno{Lion Feuchtwanger} je židovský spisovatel. Je jedním
z~nejvýznamějších autorů historické prózy. Jeho romány jsou čtivé, mají
zvláštní atmosféru. Vycházejí z~historických pramenů, ale próza má
aktualizované vyznění -- současnost, protiválečné vyznění. Romány
začínají starověkem. Vybírá si témata jako francouzskou revoluci,
židovskou tématiku a~postavení židů ve společnosti. Trojdílný cyklus
\dilo{Židovská válka} (\emph{Židovská válka, Synové, Zaslíbená země}).

Po válce se stává hlavním tématem Francie. O~Francii před vypuknutím
revoluce pojednává román \dilo{Lišky na vinici.}











