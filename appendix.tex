\onecolumn
\chapter{Seznam literárních otázek} 
\begin{enumerate} 
\item \textbf{Nejstarší památky světové literatury, kulturní odkaz starověku} \\ 
vznik literatury, vztah k~lidové slovesnosti, literatura mezopotámská, 
egyptská, indická, židovská -- bible, řecká (mytologie, homérské eposy, 
vznik dramatu --- Aischylos, Sofokles, Euripides, Aristofanes ---
a~dalších žánrů), římská (drama, další žánry, ale hlavně poezie --
Horatius, Vergilius, Ovidius)

vysvětlit: mýtus, mytologie, hrdinský epos, drama -- komedie, tragédie, 
lyrika 

četba: Staré řecké báje a~pověsti, ukázky z~čítanky (Homér, Sofokles, 
Vergilius)

\item \textbf{Počátky české literatury od nejstarší doby do doby předhusitské, románská a~gotická kultura} \\ 
vznik literatury na našem území, období staroslověnské kultury, vznik
nejstarších legend, literatura latinská -- Kosmas, vznik  nejstarších
duchovních písní -- Hospodine, pomiluj ny a~Svatý Václave, umět je
srovnat, vznik česky psané literatury -- období laicizace a~demokratizace
-- Alexandreis, Dalimilova kronika -- srovnat s~Kosmovou, vznik lyriky, 
satiry (srovnat je vzájemně)

vysvětlit: legenda (a~proč se nejstarším říká životy), svatováclavská
tradice, duchovní píseň, laicizace, rytířský epos, dvorská
lyrika, demokratizace, satira, 

četba: ukázky z~čítanky (Proglas, stsl. legendy, Hospodine, pomiluj
ny, Svatý Václave, Alexandreis, Dalimilova kronika, Podkoní a~žák, 
Hradecký rukopis\dots )

\item \textbf{Literatura doby husitské od jejích počátků do 70. let 15. století, ohlas husitství v~literatuře a~umění} \\ 
charakteristika krize středověké společnosti a~církve, předchůdci
Husovi - Štítný, Hus a~jeho dílo, literatura doby husitské - rozbor
písně Kdož jsú boží bojovníci, P. Chelčický a~vznik Jednoty bratrské, 
ohlas husitství v~literatuře a~ve výtvarném umění

vysvětlit: traktát, postila\dots  

četba: ukázky z~čítanky (Štítný, Hus -- dopisy, Kdož jsú boží bojovníci, 
Chelčický: Sieť viery pravé)
     
\item \textbf{Renesance a~humanismus v~evropské literatuře} \\
proč vzniká renesance a~humanismus právě v~S~Itálii  a~kdy, vysvětlit
pojmy, rozdíl oproti středověké gotické kultuře, představitelé světové
renesance -- Dante, Petrarca, Boccaccio, Villon, Cervantes, Shakespeare
(rozbor díla, srovnání s~antikou)

vysvětlit: sonet, francouzská balada, rámcová próza, pikareskní román

četba: W. Shakespeare: 1 hra, Cervantes: Důmyslný rytíř\dots, ukázky
z~čítanky (Petrarca, Boccaccio, Dante, Villon)

\item \textbf{Česká literatura renesanční a~humanistická} \\
charakteristika renesance a~humanismu, odlišnost situace oproti Evropě ---
husitství --- proto více humanismus, Hynek z~Poděbrad, knížky lidového
čtení, cestopisy, humanismus latinský a~národní -- Všehrd, péče
o~jazyk, doba Blahoslavova a~Veleslavínova, bible Kralická

četba: ukázky z~čítanky - Viktorín Kornel ze Všehrd, cestopisy, Blahoslav)
     
\item \textbf{Česká literatura doby pobělohorské, barokní kultura, Jan Amos Komenský} \\
charakteristika situace v~Čechách po porážce stavovského povstání, 
Obnovené zřízení zemské, pronásledování nekatolíků, literatura
exulantská -- Komenský, rekatolizační -- jezuité, traktáty, kázání, 
duchovní lyrika -- Bridel, Michna, pololidová a~lidová literatura, 
prvky baroka, světová a~česká barokní kultura a~literatura

četba: ukázky z~čítanky - Komenský, pololidová literatura, ústní lidová
slovesnost 

\item \textbf{Světová literatura 17. a~18. století} \\
klasicismus, osvícenectví, preromantismus, charakteristika světové
kultury a~jednotlivých uměleckých směrů -- dokázat na dílech, klasicismus
-- rozdíly a~společné znaky s~renesancí, hlavní znaky -- Corneille, 
Racine, Moliére, osvícenectví -- Voltaire, Rousseau, Diderot, 
encyklopedisté, preromantismus -- Roussea, německá literatura -- Schiller, 
Goethe

četba: Moliére -- hra, ukázky z~čítanky -- Corneille -- Cid, Schiller:
Loupežníci, Goethe: Faust

\item \textbf{České národní obrození} \\
vysvětlení názvu, příčiny vzniku a~kdy, cíle jednotlivých fází, 
1.~období -- vznik vědy, počátky novodobého českého jazyka, vliv 
osvícenectví -- Dobrovský, Kramerius, 2. období -- rozvoj literatury
a~~vědy, snaha vyrovnat se německé kultuře, vliv preromantismu -- 
Jungmann, Palacký, Šafařík, Čelakovský, Kollár, rukopisy RKZ

vysvětlit: ohlasy\dots

četba: Jirásek -- F.L.Věk, ukázky z~čítanky -- Jungmann, Palacký, 
Šafařík, Čelakovský
   
\item \textbf{Světový a~český romantismus} \\
na jakou společenskou situaci reaguje romantismus, hlavní znaky, 
nejvýznamnější představitelé -- Byron, Shelley, Scott, Hugo, Puškin, 
Lermontov; český -- proč přijat s~nepochopením, K. H. Mácha -- rozbor Máje, 
Erben jako představitel reakčního romantismu (polemika s~Máchou)

vysvětlit: lyrický epos, poema nebo básnická povídka, titanismus, 
romantické pojetí historie

četba: K. H. Mácha -- Máj, Erben -- Kytice; ukázky z~čítanky -- Byron, 
Shelley, Hugo

\item \textbf{Realismus a~naturalismus ve světové literatuře} \\
charakteristika realismu, kritického realismu a~naturalismu, proč
vzniká jako směr na poč.~19.~st., Balzac, Flaubert, Zola, Gogol, 
Turgeněv, Tolstoj, Dostojevskij, severská literatura

četba: 1 román, ukázky z~čítanky
     
\item \textbf{Vývoj českého realismu 19. století} \\
charakteristika realismu, první projevy, ale hlavně Havlíček, Němcová, 
zmínka o~májovcích, vznik kritického realismu -- příčina opoždění oproti
ostatní Evropě, tématika vesnická proč převažuje, hlavní znaky -- Rais, 
Nováková aj., historická -- proč je tak častá, Jirásek, Winter --
vzájemné srovnání, další autoři, tématika  městská -- naturalismus --
Čapek -- Chod

četba: Němcová, Havlíček, vesnická a~historická povídka (nebo Jirásek --
.F.L. Věk), ukázky z~čítanky (Rais, Nováková, Winter)
     
\item \textbf{Vývoj českého divadla a~dramatu od počátku do konce 19. století} \\
charakteristika divadla a~dramatu, počátky českého divadla -- vznik 
z~náb. her, Mastičkář, pololidová tvorba -- loutkové hry, interludia, 
národní obrození -- Stavovské divadlo, Bouda, Thám, Šedivý, J. K. Tyl, 
vznik Prozatimního divadla, divadelní úsilí májovů a~lumírovců --
Vrchlický, Zeyer, Národní divadlo, pronikání realismu -- reakce diváků, 
zákl. témata a~proč, Stroupežnický, Mrštíkové, Preissová

četba: Tyl: Strakonický dudák, Mrštíkové: Maryša, ukázky z~čítanky --
Mastičkář, Vrchlický -- Noc na Karlštejně 
    
\item \textbf{Generace májovců, ruchovců a~lumírovců a~jejich význam pro rozvoj české poezie a~prózy} \\
charakteristika společenské situace od 50. do 80.let 19. století
(Bachův absolutismus, uvolnění pol. a~kult. poměrů po Říjnovém diplomu, 
česká politika), vydávání almanachů, Máj -- název, charakter, program
májovců, nejvýznamnější -- Neruda, Hálek, jejich srovnání, Světlá, Arbes
aj., ruchovci a~lumírovci -- srovnání, nejvýznamnější Čech, Vrchlický, 
Sládek aj., přínos pro literaturu, vysvětlit almanach, sociální balada

četba: Neruda -- Povídky malostranské, 1 básnická sbírka, ukázky
z~čítanky -- Hálek, Heyduk, Vrchlický, Zeyer, Sládek, Čech
     
\item \textbf{Reakce básníků na myšlenkovou, společenskou a~politickou krizi konce 19. a~počátku 20. století} \\
charakteristika myšl. pol. a~kulturní krize přelomu století, reakce
umění, vznik moderních uměleckých směrů -- impresionismus, symbolismus, 
dekadence, secese, prokletí básníci, Česká moderna, Bezruč, 
poč.~20.~st. -- generace buřičů a~anarchistů -- Gellner, Šrámek, Toman, Dyk

četba: Bezruč: Slezské písně, ukázky z~čítanky (Verlaine, Rimbaud, 
Baudelaire, Sova, Březina, Hlaváček, Gellner, Šrámek, Toman, Dyk)
      
\item \textbf{Moderní umělecké směry a~poezie mezi dvěma světovými válkami} \\
vznik moderních uměleckých směrů  kolem 1. sv. války -- dadaismus, 
futurismus, expresionismus, civilismus, kubismus, surrealismus atd., 
představitelé světové poezie -- hl. Appollinaire, česká poezie mezi
válkami -- reakce na spol. situaci, vznik proletářské poezie, její
program -- Wolker, Hora,  Hořejší, vznik poetismu -- Nezval, Seifert, 
Biebl, surrealismu, 30. léta -- spiritualismus -- Halas, Holan\dots, reakce
na Mnichov a~okupaci -- Halas

četba: 1 básn. sbírka, ukázky z~čítanky (Applollinaire, Wolker, Hora, 
Seifert, Biebl, Nezval, Halas)     

\item \textbf{Česká próza od počátku 20. století a~její mnohovrstevnatost v~době meziválečné} \\
reakce literatury na spol. situaci, tématické, politické, umělecké
rozdělení prózy: téma 1. sv. války a~literatura legionářská -- Hašek, 
John, Medek, Kopta, Kratochvíl\dots, literatura ovlivněná expresionismem
-- Klíma, imaginativní próza -- Vančura, společenská próza -- Olbracht, 
Majerová, Pujmanová, katolická próza a~ruralismus -- Durych, Čep, Deml, 
Křelina, Humanistická a~demokratická česká meziválečná próza, 
charakteristika demokratického programu, filozofie pragmatismu
a~relativismu -- Čapek, Poláček, Bass, Langer, psychologická -- Havlíček, 
Glazarová, Řezáč

četba:  Hašek, K. Čapek, 1 román (Vančura, Olbracht, Durych apod.)
ukázky z~čítanky

\item \textbf{1. světová válka v~světové a~české literatuře} \\
charakteristika 1. sv. války a~proč tak silně ovlivnila literaturu, 2
vlny -- 20. a~30. léta, literatura francouzská -- Barbusse, Rolland, 
německá -- Remarque, americká -- Hemingway. česká -- John, Hašek, Poláček, 
Vančura atd., legionářská literatura) 

četba: Hašek, 1 světový román, ukázky z~čítanky 

\item \textbf{Podněty ze světové literatury 1. poloviny 20. století} \\
charakteristika vývoj světové prózy v~období mezi válkami, tématika 
1.~světové války -- jen připomínka, světová demokratická a~humanistická
literatura -- Mann, Feuchtwanger, Faulkner, Steinbeck, experimentální --
Joyce, Proust, pražská židovská literatura -- Kafka, Werfel\dots

četba: 1 román, Kafka, ukázky z~čítanky
      
\item \textbf{2. světová válka v~literatuře} \\ 
2. sv. válka jako hlavní téma světové literatury, Moravia, Remarque, 
B\" oll, Lenz, Grass, Mailer, Heller, Styron, Clavell apod., sovětská
literatura, téma okupace a~války v~české literatuře -- Drda, Fučík, 
Ptáčník, Valenta, Hostovský, Weil, Lustig, Fuks, Hrabal, Pavel, 
Škvorecký atd.

četba: 1 světový román a~1 český román  s~touto tématikou     

\item \textbf{Dramatická tvorba 20. století v~české a~světové literatuře} \\
světové drama -- Ibsen, Shaw atd., epické divadlo -- Brecht, české --
tvorba anarchistů + divadlo do 1. sv. války, poválečné -- vliv
expresionismu  např. Čapek, experimentální -- V+W, Burian, drama za
okupace např. Nezval, poválečné drama -- absurdní francouzské drama, 
české\dots 

četba: Čapek, V+W, 1 poválečné české drama 
   
\item \textbf{Vývoj české poezie a~prózy v~letech 1945 - 1970} \\
charakteristika doby a~jednotlivých období, básnické skupiny, reakce
poezie na osvobození, Nezval, Seifert, Holan, 
Halas, Zahradníček, Hrubín, Mikulášek, Kolář, Kainar, Holub, Skácel, 
Wernisch\dots \\ 
próza: okupační tématika jen zmínka, společenská tématika --
Otčenášek, Řezáč, Ptáčník, Vaculík, Kundera\dots, psychologická tématika
Stýblová, Mucha, Hostovský, historická tématika -- jen zmínka, budovatelský
a~tzv. profesní román -- Řezáč, Kozák, Klíma, Páral, \dots 

četba:  1 sbírka, 1 román s~tématikou války, 1 román s~jinou tématikou

\item \textbf{Vývoj české poezie a~prózy od r. 1970 po současnost} \\
charakteristika doby a~jednotlivých období, literatura oficiální, 
samizdatová, exilová, poezie -- starší autoři -- Seifert, Závada -- mladší
generace -- Žáček, samizdat -- Šiktanc, Wernisch, exil -- Diviš, 
underground -- Jirous, Topol -- písničkáři, próza -- viz předchozí otázka, 
spol. tématika -- Škvorecký, Klíma, Hrabal, Viewegh, Kundera, Vaculík\dots

četba: alespoň jeden román

\item \textbf{Světová literatura po roce 1945} \\
charakteristika doby, existencialismus -- Sartre, Camus, neorealismus --
Moravia, rozhněvaní mladí muži, beatníci -- Kerouac, magický realismus --
Ajtmatov, García Márquez, nový román, absurdní drama -- Beckett, Ionesco, 
antiutopie -- Orwell, postmodernismus -- Eco, fantazijní literatura, téma 
2.~sv.~války atd.

četba: 1 román s~tématikou války, 1 román s~jinou tématikou 

\item \textbf{Humor a~satira v~české literatuře} \\ 
co je to humor a~satira, počátky satiry -- Mastičkář, staročeské satiry, 
husitské satiry, satiry 19. století -- Havlíček, Čech, 20. století --
Hašek, Čapek, Poláček\dots, 2.~pol.~20.~st. -- Jirotka, Škvorecký\dots

četba: Havlíček, Hašek, Čapek, ukázky z~čítanky 
      
\item \textbf{Evropská a~česká historická próza a~její proměny} \\ 
obliba historie od starověku, rozvoj v~romantismu -- Scott, Hugo, Mácha, 
Tyl, rozdíl mezi romantickým a~realistickým pojetím -- Flaubert, 
Tolstoj, Jirásek, Winter, 20.~století -- Feuchtwanger, Mann atd., Vančura, 
Durych\dots, poválečná lit. -- zejména česká -- Frýd, Kratochvíl, 
Kubka, Toman, Šotola, Neff, K\" orner) 

četba: Jirásek
\end{enumerate}

\twocolumn

\chapter{Seznam mluvnických otázek}
\begin{enumerate}
\item Stylistika a~funkční styly
\item Čeština a~jazyky příbuzné
\item Vývoj jazyka
\item Čeština jako národní jazyk --- rozvrstvení jazyka
\item Slovní zásoba, rozvrstvení, slovníky
\item Změny slovního významu
\item Obohacování slovní zásoby  
\item Knižní jazykové pomůcky, slovníky
\item Odborný styl --- popis, výklad
\item Žánry a~ž-anrové formy
\item Administrativní styl --- životopis
\item Publicistika --- informační a~beletrizované útvary
\item Tvarosloví --- slovní druhy   
\item Tvarosloví --- mluvnické kategorie jmen
\item Tvarosloví --- mluvnické kategorie sloves, přechodníky
\item Tvarosloví --- mluvnické kategorie sloves, přechodníky
\item Věty podle postoje mluvčího
\item Syntax --- věta jednoduchá, větné členy
\item Syntax --- odchylky od pravidelné větné stavby
\item Aktuální členění větné výpovědi
\item Syntax --- souvětí podřadné
\item Syntax --- souvětí souřadné
\item Zvuková stránka jazyka
\item Hlavní principy českého pravopisu
\item Racionální studium textu
\end{enumerate}